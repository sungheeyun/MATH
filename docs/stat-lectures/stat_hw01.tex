

\input{D:/Multimedia/mydefs}
%\input{D:/mydefs}

\usepackage{fullpage}
\usepackage{fancyhdr}

\usepackage{graphicx}

\pagestyle{fancy}
\fancyhead[R]{Probability and Statistics for Electrical Engineering}

\addtolength{\headsep}{.5cm}



\newcommand{\mfor}{\mbox{ for }}
\newcommand{\mand}{\mbox{ and }}
\newcommand{\mforall}{\mbox{ for all }}
\newcommand{\mif}{\mbox{if }}
\newcommand{\ld}{\ensuremath{\lambda}}


\newcommand{\lgprob}[1]{LGProb~{#1}}
\newcommand{\lgfig}[1]{LGFig~{#1}}
\newcommand{\lgexam}[2]{LGExample~{#1} {\bf #2}}
\newcommand{\lgexamref}[1]{LGExample~{#1}}
\newcommand{\lgeq}[1]{LGEquation~(#1)}

\newcommand{\br}[1]{\{#1\}}

\newcommand{\randvar}{random variable}
\newcommand{\var}{variance}
\newcommand{\bernrv}{{Bernoulli \randvar}}
\newcommand{\binomrv}{{binomial \randvar}}
\newcommand{\geomrv}{{geometric \randvar}}
\newcommand{\possrv}{{Poisson \randvar}}

\newcommand{\unifrv}{{uniform \randvar}}
\newcommand{\exprv}{{exponential \randvar}}
\newcommand{\gaussrv}{{uniform \randvar}}

\newcommand{\possdist}{{Poisson distribution}}
\newcommand{\cond}{conditional}

\newcommand{\unitfcn}{unit step function}
\newcommand{\delfcn}{delta function}



\newcommand{\al}{\ensuremath{\alpha}}


\newcommand{\sA}{\ensuremath{A}}
\newcommand{\sB}{\ensuremath{B}}
\newcommand{\Ai}[1]{\ensuremath{A_{#1}}}
\newcommand{\Bi}[1]{\ensuremath{B_{#1}}}
\newcommand{\Ei}[1]{\ensuremath{E_{#1}}}
\newcommand{\Ak}{\Ai{k}}
\newcommand{\X}{\ensuremath{X}}
\newcommand{\xk}{\ensuremath{x_k}}
\newcommand{\mzeta}{\ensuremath{\zeta}}

\newcommand{\comp}[1]{\ensuremath{{#1}^c}}
\newcommand{\diff}[2]{{#1}\backslash{#2}}
\newcommand{\sspace}{\ensuremath{S}}
\newcommand{\ssx}{\ensuremath{S_X}}
\newcommand{\evcl}{\ensuremath{\mathcal{F}}}

\newcommand{\pr}[1]{\ensuremath{P[#1]}}
%\renewcommand{\pr}[1]{\ensuremath{{\bf Pr}\{#1\}}}
\newcommand{\prb}[1]{\ensuremath{P[\{#1\}]}}
\newcommand{\bpr}[1]{\ensuremath{P\left[#1\right]}}
\newcommand{\cpr}[2]{\pr{#1|#2}}
\newcommand{\cpreq}[2]{\frac{\pr{#1\cap #2}}{\pr{#2}}}
\newcommand{\cprbeq}[2]{\cpreq{\br{#1}}{\br{#2}}}
\newcommand{\appr}[2]{\pr{#1|#2}\pr{#2}}
\newcommand{\relfreq}[1]{f_{#1}}

\newcommand{\expect}[1]{\ensuremath{m_{#1}}}
\newcommand{\cexp}[2]{\expect{#1|#2}}
\renewcommand{\Expect}[1]{\ensuremath{{\bf E}[#1]}}
\newcommand{\bExp}[1]{\ensuremath{{\bf E}\left[#1\right]}}
\newcommand{\Exp}[1]{\Expect{#1}}
\newcommand{\cExp}[2]{\Exp{#1|#2}}
\newcommand{\samplemean}[2]{\langle{#1}\rangle_{#2}}
\newcommand{\std}[1]{\ensuremath{\sigma_{#1}}}
\newcommand{\VAR}[1]{\ensuremath{{\bf VAR}[#1]}}
\newcommand{\cVAR}[2]{\VAR{#1|#2}}
\newcommand{\STD}[1]{\ensuremath{{\bf STD}[#1]}}

\newcommand{\ep}[1]{\ensuremath{E_{#1}}}
\renewcommand{\ev}[1]{\ensuremath{A_{#1}}}

\newcommand{\borel}{\ensuremath{\mathcal{B}}}

\newcommand{\chs}[2]{\ensuremath{{{#1} \choose {#2}}}}

\newcommand{\bigcupto}[3]{\bigcup_{#1=#2}^{#3}}
\newcommand{\bigcupkton}{\bigcupto{k}{1}{n}}
\newcommand{\bigcupktoi}{\bigcupto{k}{1}{\infty}}

\newcommand{\bigcapto}[3]{\bigcap_{#1=#2}^{#3}}
\newcommand{\bigcapkton}{\bigcapto{k}{1}{n}}

\newcommand{\sumto}[3]{\sum_{#1=#2}^{#3}}
\newcommand{\sumkton}{\sumto{k}{1}{n}}
\newcommand{\sumiton}{\sumto{i}{1}{n}}
\newcommand{\sumktoi}{\sumto{k}{1}{\infty}}
\newcommand{\sumntoi}{\sumto{n}{1}{\infty}}
\newcommand{\sumkzton}{\sumto{k}{0}{n}}
\newcommand{\sumkztoi}{\sumto{k}{0}{\infty}}
\newcommand{\sumnztoi}{\sumto{n}{0}{\infty}}

\newcommand{\xcomma}[2]{\ensuremath{{#1}_1, {#1}_2, \ldots, {#1}_{#2} }}
\newcommand{\xcommai}[1]{\ensuremath{{#1}_1, {#1}_2, \ldots }}
\newcommand{\xplus}[2]{\ensuremath{{#1}_1 + {#1}_2 + \cdots + {#1}_{#2} }}
\newcommand{\xcap}[2]{\ensuremath{{#1}_1 \cap {#1}_2 \cap \cdots \cap {#1}_{#2} }}

\newcommand{\ceil}[1]{\ensuremath{\lceil #1 \rceil}}
\newcommand{\floor}[1]{\ensuremath{\lfloor #1 \rfloor}}
\newcommand{\kmax}{\ensuremath{k_\mathrm{max}}}

\newcommand{\posspmf}[2]{\frac{{#1}^{#2}}{{#2}!}e^{-{#1}}}
\newcommand{\binompmf}[3]{\chs{{#1}}{{#3}}{#2}^{#3}(1-{#2})^{{#1}-{#3}}}
\newcommand{\dbinompmf}{\binompmf{n}{p}{k}}


%\newcommand{\numbereveryeqns}{}

\ifdefined\numbereveryeqns
	\newcommand{\beq}{\begin{equation}}
	\newcommand{\eeq}{\end{equation}}
\else
	\newcommand{\beq}{\[}
	\newcommand{\eeq}{\]}
\fi

\newcommand{\beql}[1]{\begin{equation}\label{#1}}
\newcommand{\eeql}{\end{equation}}

\newcommand{\optional}{$\dagger$}

\newcommand{\pmf}[1]{p_{#1}}
\newcommand{\pmfk}[2]{\ensuremath{\pmf{#1}(#2)}}
\newcommand{\pmfxk}[1]{\pmfk{X}{#1}}
\newcommand{\cpmfk}[3]{\ensuremath{\pmf{#1}(#2|#3)}}
\newcommand{\cpmfxk}[2]{\cpmfk{X}{#1}{#2}}

\newcommand{\cdf}[1]{F_{#1}}
\newcommand{\cdfk}[2]{\ensuremath{\cdf{#1}(#2)}}
\newcommand{\cdfxk}[1]{\cdfk{X}{#1}}
\newcommand{\ccdfk}[3]{\ensuremath{\cdf{#1}(#2|#3)}}
\newcommand{\ccdfxk}[2]{\ccdfk{X}{#1}{#2}}

\newcommand{\pdf}[1]{f_{#1}}
\newcommand{\pdfk}[2]{\ensuremath{\pdf{#1}(#2)}}
\newcommand{\pdfxk}[1]{\pdfk{X}{#1}}
\newcommand{\cpdfk}[3]{\ensuremath{\pdf{#1}(#2|#3)}}
\newcommand{\cpdfxk}[2]{\cpdfk{X}{#1}{#2}}

\newcommand{\indi}[1]{I_{#1}}

\newcommand{\acapb}{\ensuremath{A\cap B}}
\newcommand{\acupb}{\ensuremath{A\cup B}}


\newcommand{\limninfty}{\lim_{n\to\infty}}
\newcommand{\intinftoinf}{\int_{-\infty}^\infty}
\newcommand{\intinfto}[1]{\int_{-\infty}^{#1}}

\newcommand{\ddx}{\frac{d}{dx}}

\newcommand{\gausspdf}[3]{\frac{1}{\sqrt{2\pi} {#3}} e^{-({#1}{#2})^2/2{#3}^2} }



\renewcommand{\emph}[1]{{\it #1}}

\begin{document}
\setlength{\headheight}{15pt}
\maketitle
\begin{enumerate}
	\item LG 1.1
	\ifdefined\sol
	Consider the following three random experiments:
	\begin{itemize}
		\item Experiment 1: Toss a coin.
		\item Experiment 2: Toss a die.
		\item Experiment 3:
		Select a ball at random from an urn containing balls numbered $0$ to $9$.
	\end{itemize}
	\begin{enumerate}
		\item Specify the sample space of each experiment.
		\begin{solution}
			\begin{itemize}
				\item Experiment 1: $\sspace = \{H,T\}$
				\item Experiment 2: $\sspace = \{1,2,\ldots,6\}$
				\item Experiment 3: $\sspace = \{0,1,\ldots,9\}$
			\end{itemize}
		\end{solution}

		\item Find the relative frequency of each outcome
		in each of the above experiments
		in a large number of repetitions of the experiment. Explain your answer.
		\begin{solution}
			Let $N_k(n)$ be the number of times in which the outcomes are $k$
			when an experiment is repeated $n$ times.
			Then the relative frequency of outcome $k$ is defined by
			\[
				f_k(n) = \frac{N_k(n)}{n}.
			\]
			\begin{itemize}
				\item Experiment 1:
				If it is a fair coin,
				as we repeat tossing the coin many times,
				the number of heads and that of tails
				will tend to be (asymptotically) the same.
				Hence, both $f_1(n)$ and $f_2(n)$ will get close to $1/2$.

				\item Experiment 2:
				Again, if it is a fair die,
				as we repeat tossing the die many times,
				the number of each outcome
				will tend to be the same as others.
				Hence, $f_k(n) \to 1/6$ as $n\to \infty$ for each $1\leq k\leq 6$.

				\item Experiment 3:
				Likewise,
				$f_k(n) \to 1/10$ as $n\to \infty$ for each $0\leq k\leq 9$.
			\end{itemize}
		\end{solution}
	\end{enumerate}
	\fi


	\item LG 1.4
	\ifdefined\sol
	An urn contains three electronically labeled balls with labels
	$00$, $01$, $10$.
	Lisa, Homer, and Bart are asked to characterize the random experiment
	that involves selecting a ball at random and reading the label.
	Lisa's label reader works fine;
	Homer's label reader has the most significant digit stuck at $1$;
	Bart's label reader's least significant digit is stuck at $0$.
	\begin{enumerate}
		\item What is the sample space determined by Lisa, Homer, and Bart?
		\begin{solution}
			\begin{itemize}
				\item Lisa's label reader works fine, hence $\sspace_\mathrm{Lisa} = \{00,01,10\}$.

				\item Since Homer's label reader has the most significant digit stuck at $1$,
				$00$ is read as $10$,
				$01$ is read as $11$,
				and $10$ is read as $10$.
				Thus $\sspace_\mathrm{Homer} = \{10,11\}$

				\item Similaryly, since
				Bart's label reader's least significant digit is stuck at $0$,
				$00$ is read as $00$,
				$01$ is read as $00$,
				and $10$ is read as $10$.
				Thus, $\sspace_\mathrm{Bart} = \{00,10\}$

			\end{itemize}
		\end{solution}

		\item What are the relative frequencies observed by Lisa, Homer, and Bart
		in a large number of repetitions of the experiment?
			\begin{itemize}
				\item Lisa: The three relative frequencies will approach $1/3$.
				\item Homer:
				Since the outcome is $10$ when the actual label is either $00$ or $10$,
				and the outcome is $11$ when the actual label is $01$,
				the relative frequency of $10$ will approach $2/3$
				and that of $11$ will approach $1/3$.

				\item Bart:
				Since the outcome is $00$ when the actual label is either $00$ or $01$,
				and the outcome is $10$ when the actual label is $10$,
				the relative frequency of $00$ will approach $2/3$
				and that of $10$ will approach $1/3$.
			\end{itemize}
	\end{enumerate}
	\fi

	\item LG 1.10
	\ifdefined\sol
	Suppose that the signal $2 cos (2\pi t)$
	is sampled at random instants of time.
	\begin{enumerate}
		\item Find the long-term sample mean.
		\begin{solution}
			Let $f(t) = 2\cos(2\pi t)$.
			Since $f(t)$ is periodic with period $1$,
			\ie, $f(t+1) = f(t)$ for any $t$,
			the random sampling for $t\geq0$
			is equivalent to the random sampling
			within $t\in [0,1]$.
			Since it is reasonable to assume that
			$t$ is uniformly distributed on $[0,1]$,
			the long-term sample mean will be
			\[
				\int_0^1 2\cos(2\pi t) = 0.
			\]
			(Indeed, the description of this problem is \emph{not} unambiguous at all.
			We must define the ``randomness'' in a specific way,
			\eg, it follows the Gaussian distribution, the geometric distribution, \etc)
		\end{solution}

		\item Find the long-term relative frequency of the events ``voltage is positive'';
		``voltage is less than $-2$.''
		\begin{solution}
			Since we can confine our sample space to $[0,1]$,
			the events ``voltage is positive'' happen
			for $t\in(0,1/2)$,
			hence the long-term relative frequencies of the event is $1/2$.
			However, ``voltage is less than $-2$'' never happens,
			hence the long-term relative frequencies of the event is $0$.
		\end{solution}

		\item Do the answers to parts (a) and (b) change
		if the sampling times are periodic and taken every $\tau$ seconds?
		\begin{solution}
			``Yes'' for (a).
			For example, if $\tau=1$ and we start sampling at $t=0$,
			the long-term sample mean is just $1$.
			``Yes'' for the first event in (b).
			Again, if $\tau=1$ and we start sampling at $t=0$,
			the long-term relative frequency of the event ``voltage is positive''
			is $1$.
			And ``No'' for the second event in (b)
			since the event never happens
			no matter how we sample the signal.
		\end{solution}
	\end{enumerate}
	\fi

	\item LG 2.2
	\ifdefined\sol
	A die is tossed twice and the number of dots facing up in each toss
	is counted and noted in the order of occurrence.
	\begin{enumerate}
		\item Find the sample space.
		\begin{solution}
		$ \sspace = \set{(x,y) \in\integers^2}{ 1 \leq x, y\leq 6, \mand x \mand y \mbox{ are integers}}.  $
		\end{solution}

		\item Find the set $A$ corresponding to the event
		``number of dots in first toss is not less
		than number of dots in second toss.''
		\begin{solution}
		$ A = \set{(x,y) \in\integers^2}{ 1 \leq y \leq  x\leq 6 }$
		\end{solution}

		\item Find the set $B$ corresponding to the event
		``number of dots in first toss is $6$.''
		\begin{solution}
			$ B = \set{(6,y)\in\integers^2}{ 1\leq y\leq 6}
			= \{ (6,1), (6,2), (6,3), (6,4), (6,5), (6,6) \}.$
		\end{solution}

		\item Does $A$ imply $B$ or does $B$ imply $A$?
		\begin{solution}
			Oviously, $B\subset A$, hence $B$ implies $A$.
		\end{solution}

		\item Find $A\cap \comp{B}$ and describe this event in words.
		\begin{solution}
			Note that
			\[
				A\cap \comp{B} = \set{(x,y)\in\integers^2}{ 1\leq y\leq x\leq 5}.
			\]
			It corresponds to the event
			that ``number of dots in first toss is not less
			than number of dots in second toss
			{and} number of dots in first toss is not $6$.''
		\end{solution}

		\item Let $C$ correspond to the event
		``number of dots in dice differs by $2$.''
		Find $A\cap C$.
		\begin{solution}
			$ A\cap C = \{ (6,4), (5,3), (4,2), (3,1) \}.  $
		\end{solution}

	\end{enumerate}
	\fi


	\item LG 2.3
	\ifdefined\sol
	Two dice are tossed and the magnitude of the difference
	in the number of dots facing up in the two dice is noted.
	\begin{enumerate}
		\item Find the sample space.
		\begin{solution}
			\sspace = \{0,1,2,3,4,5\}.
		\end{solution}

		\item Find the set $A$ corresponding to the event
		``magnitude of difference is $3$.''
		\begin{solution}
			$ A = \{3\}$.
		\end{solution}

		\item Express each of the elementary events in this experiment
		as the union of elementary events from Problem 2.2
		\begin{solution}
		\begin{eqnarray*}
			\{0\} &\mbox{ corresponds to }& \{(1,1),(2,2),(3,3),(4,4),(5,5),(6,6)\}.
			\\ \{1\} &\mbox{ corresponds to }& \{(1,2),(2,3),(3,4),(4,5),(5,6),
			(2,1),(3,2),(4,3),(5,4),(6,5)\}.
			\\ \{2\} &\mbox{ corresponds to }& \{(1,3),(2,4),(3,5),(4,6),
			(3,1),(4,2),(5,3),(6,4)\}.
			\\ \{3\} &\mbox{ corresponds to }& \{(1,4),(2,5),(3,6), (4,1),(5,2),(6,3)\}.
			\\ \{4\} &\mbox{ corresponds to }& \{(1,5),(2,6), (5,1),(6,2) \}.
			\\ \{5\} &\mbox{ corresponds to }& \{(1,6), (6,1) \}.
		\end{eqnarray*}
		\end{solution}

	\end{enumerate}
	\fi

	\item LG 2.9
	\ifdefined\sol
		The sample space of an experiment is the real line.
		Let the events $A$ and $B$ correspond to the following subsets of the real line:
		$A = (\infty, r]$ and $B = (-\infty, s]$, where $r \leq s$.
		Find an expression for the event C = $(r, s]$
		in terms of $A$ and \sB. Show that $B = A \cup C$
		and $A \cap C = \emptyset$.
		\begin{solution}
		It is obvious that
		\[
			A \cup C =( -\infty,s] \cup (s,r] = (-\infty,r] = B
		\]
		and
		\[
			A \cap C =( -\infty,s] \cap (s,r] = \emptyset.
		\]
		Note that $\{A,C\}$ is a partition of $B$,
		thus
		\[
			C = B-A = B \backslash A = B\cap \comp{A}.
		\]
		\end{solution}
	\fi

	\item LG 2.12
	\ifdefined\sol
	Show that if $A\cup B = A$ and $A \cap B=A$ then $A=B$.
	\begin{solution}
	Suppose that $A\cup B = A$.
	Then for any $x\in B$, $x\in A\cup B = A$, hence $x\in A$.
	Thus $B\subset A$.
	Now suppose that $A\cap B = A$.
	Then for any $x\in A$, $x\in A\cap B$, hence $x\in B$.
	Thus $A\subset B$.

	Therefore $A=B$
	(since $A=B$ if and only if $A\subset B$ and $B\subset A$).
	\end{solution}
	\fi

	\item LG 2.15
	\ifdefined\sol
	Figure P2.1 shows three systems of three components,
	$A_1$, $A_2$, and $A_3$.
	Figure P2.1(a) is a ``series'' system in which
	the system is functioning
	only if all three components are functioning.
	Figure 2.1(b) is a ``parallel'' system
	in which the system is functioning as long as
	at least one of the three components is functioning.
	Figure 2.1(c) is a ``two-out-of-three'' system in which the system is functioning
	as long as at least two components are functioning.
	Let \Ak\ be the event ``component $k$ is functioning.''
	For each of the three system configurations,
	express the event ``system is functioning''
	in terms of the events \Ak.
	\begin{solution}
		Let $F_1$ be the event that the ``series'' system is functioning,
		$F_2$ be the event that the ``parallel'' system is functioning,
		and $F_3$ be the event that the ``two-out-of-three'' system is functioning.
		\begin{itemize}
			\item Since the ``series'' system is functioning
			if and only if all the components are functioning,
			\[
				F_1 = A_1 \cap A_2 \cap A_3.
			\]

			\item
			Also since the ``parallel'' system is functioning
			if and only if any of the three components is functioning,
			\[
				F_2 = A_1 \cup A_2 \cup A_3.
			\]

			\item
			Lastly, we can see that the ``two-out-of-three'' system is functioning
			if and only if
			both the first and the second components are functioning
			or both the second and the third components are functioning
			or both the third and the first components are functioning.
			Thus
			\[
				F_3 =
				(A_1 \cap A_2)
				\cup
				(A_2 \cap A_3)
				\cup
				(A_1 \cap A_3).
			\]
		\end{itemize}
	\end{solution}
	\fi


	\item LG 2.23
	\ifdefined\sol
	A random experiment has sample space $S = \{a, b, c, d\}$.
	Suppose that $\pr{\{c, d\}} = 3/8$,
	$\pr{\{b, c\}} = 6/8$,
	and $\pr{\{d\}} = 1/8$.
	Use the axioms of probability
	to find the probabilities of the elementary events.
	\begin{solution}
	First, the Axiom III implies that $\prb{c} + \prb{d} = \pr{\{c,d\}}$,
	hence \[
		\prb{c} = \pr{\{c,d\}} - \prb{d} = 3/8 - 1/8 = 1/4.
	\]
	Similarly \[\prb{b} = \pr{\{b,c\}} - \prb{c} = 6/8 - 1/4 = 1/2.\]
	Lastly, the Axiom II and Corollary 4 imply
	and \[\prb{a} = 1 - \prb{a} - \prb{b} - \prb{c} = 1/8.\]
	To summarize,
	\[
		\prb{a} = 1/8,\
		\prb{b} = 1/2,\
		\prb{c} = 1/4,\
		\prb{d} = 1/8.
	\]
	\end{solution}
	\fi

	\item LG 2.36
	\ifdefined\sol
	The lifetime of a device behaves according to the probability law
	$\pr{(t, \infty)} = 1/t$ for $t > 1$.
	Let $A$ be the event ``lifetime is greater than $4$,''
	and $B$ the event ``lifetime is greater than $8$.''
	\begin{enumerate}
		\item Find the probability of \acapb, and \acupb.
		\begin{solution}
		Note that
		\[
			\acapb = (4,\infty) \cap (8,\infty) = (8,\infty) = B
			\mand
			\acupb = (4,\infty) \cup (8,\infty) = (4,\infty) = A,
		\]
		hence
		\[
			\pr{A\cap B} = \pr{B} = 1/8
			\mand
			\pr{A\cup B} = \pr{A} = 1/4.
		\]
		\end{solution}

		\item Find the probability of
		the event ``lifetime is greater than $6$ but less than or equal to $12$.''
		\begin{solution}
			Let $C$ be the event ``lifetime is greater than $6$ but less than or equal to $12$.''
			And let $D$ be the event ``lifetime is greater than $6$,''
			and $E$ the event ``lifetime is greater than $12$.''
			Then we have
			\[
				C \cup E = (6,12] \cup (12,\infty) = (6,\infty) = D.
			\]
			Since $C$ and $E$ are disjoint,
			\[
				\pr{C} = \pr{D} - \pr{E} = 1/6 - 1/12 = 1/12.
			\]
		\end{solution}
	\end{enumerate}
	\fi


	\item LG 2.62
	\ifdefined\sol
	A die is tossed twice and the number of dots facing up is
	counted and noted in the order of occurrence.
	Let $A$ be the event
	``number of dots in first toss is not less than number of dots
	in second toss,''
	and let $B$ be the event
	``number of dots in first toss is $6$.''
	Find \cpr{A}{B} and \cpr{B}{A}.
	\begin{solution}
		Note that
		\[
			A = \set{(x,y)\in\integers^2}{1\leq y\leq x\leq 6}
			\mand
			B = \set{(6,y)\in\integers^2}{1\leq y\leq 6}
		\]
		and that
		\[
			\pr{A} = \frac{1+2+\cdots+6}{36} = \frac{21}{36} = 7/12
			\mand
			\pr{B} = 1/6.
		\]
		Since $B\subset A$, $A\cap B = B$.
		Therefore
		\[
			\cpr{A}{B} = \cpreq{A}{B} = 1
			\mand
			\cpr{B}{A} = \cpreq{B}{A} = \frac{1}{6} \cdot \frac{12}{7}
			= 2/7.
		\]

	\end{solution}
	\fi

	\item LG 2.67
	\ifdefined\sol
	In Problem 2.36 (device lifetime), find \cpr{B}{A}\ and \cpr{A}{B}.
	\begin{solution}
		Note that
		\[ A = (4,\infty) \mand B = (8,\infty), \]
		thus
		\[ \pr{A} = 1/4 \mand \pr{B} = 1/8. \]
		Since $B\subset A$, $A\cap B = B$,
		hence
		\[
			\cpr{B}{A} = \cpreq{B}{A} = \frac{\pr{B}}{\pr{A}} = 1/2
		\]
		and
		\[
			\cpr{A}{B} = \cpreq{A}{B} = 1.
		\]

	\end{solution}
	\fi


	\item LG 2.76
	\ifdefined\sol
	In each lot of $100$ items,
	two items are tested,
	and the lot is rejected if either of the tested items is found defective.
	\begin{enumerate}
		\item Find the probability that a lot with $k$ defective items
		is accepted.
		\begin{solution}
			Suppose we number each time from $1$ to $100$.
			Without loss of generality the first $k$ items are defective.
			The number of different combinations of size $2$ from $100$ is
			$100 \choose 2$.
			The number of different combinations of size $2$ from $100-k$
			non-defective items is $100-k\choose 2$.
			Therefore the probability that a lot with $k$
			defective items is accepted is
			\[
				\frac{{100-k\choose 2}}{{100\choose2 }}
				= \frac{(100-k)(100-k-1)}{100\cdot 99}.
			\]
		\end{solution}

		\item Suppose that when the production process malfunctions,
		$50$ out of $100$ items are defective.
		In order to identify when the process is malfunctioning,
		how many items should be tested so that the probability
		that one or more items are found defective is at least $99$\%?
		\begin{solution}
			Suppose that $k$ items are tested.
			Then the probability that no items are found defective is
			\[
				\frac{{50 \choose k}}{{100 \choose k}}
			\]
			since the number of different combinations of choosing $k$
			items out of total of $100$ items is $100 \choose k$
			and the number of different combinations of choosing $k$
			items out of the $50$ non-defective items is $50 \choose k$.
			In order for the probability that one or more items are found defective
			to be at least $.99$,
			the probability that no items are found defective must be
			less than $.01$.
			Hence we should have
			\[
				\frac{50\cdot49\cdots (50-k+1)}
				{100\cdot99\cdots (100-k+1)}
				< 0.01.
			\]
			Running the following Python code yields
			this condition is equivalent to $k \geq 7$.
			Figure~\ref{fig-prob-2-76}
			shows this probability as a function of $k$;
			the bottom one is drawn in log-scale.
			Therefore $7$ items must be tested
			in order for the probability that one or more items are found defective
			to be at least $.99$.

			\begin{figure}
			\begin{center}
			\includegraphics[width=.8\linewidth]{pr02_76}
			\caption{
			The probability that no items are found defective as
			a function of $k$.}
			\label{fig-prob-2-76}
			\end{center}
			\end{figure}

			\begin{verbatim}
import pylab as pl
import numpy as np

K = 10

a = np.linspace(50,50-K+1,K)
b = np.linspace(100,100-K+1,K)
c = a/b
d = np.cumprod(c)

k = np.linspace(1,10,10)

print d

fig = pl.figure()
ax1 = fig.add_subplot(211)
ax2 = fig.add_subplot(212)
ax1.plot(k,d,'o-')
ax1.set_xlabel('k')
ax1.set_ylabel('probability')

ax2.semilogy(k,d,'o-')
ax2.set_xlabel('k')
ax2.set_ylabel('probability (log-scale)')

pl.show()
			\end{verbatim}



		\end{solution}
	\end{enumerate}
	\fi

	\item LG 2.82
	\ifdefined\sol
	Let $\sspace = \{1, 2, 3, 4\}$
	and $A = \{1, 2\}$, $ B = \{1, 3\}$, $ C = \{1, 4\}$.
	Assume the outcomes are equiprobable. Are $A$, $B$, and $C$ independent events?
	\begin{solution}
		No, they are not independent events.
		We must be very careful when dealing with more than $2$ events
		and test their independence.
		Obviously, these events are pairwise independent
		since
		\[
			\pr{A\cap B} = \pr{A} \pr{B},\
			\pr{B\cap C} = \pr{B} \pr{C},\
			\pr{A\cap C} = \pr{A} \pr{C}.
		\]
		However,
		\[
			1/4 = \pr{A\cap B\cap C} \ne \pr{A}\pr{B} \pr{C} = 1/8,
		\]
		thus the three events are \emph{not} independent.
	\end{solution}
	\fi

	\item LG 2.86
	\ifdefined\sol
	Show that events $A$ and $B$ are independent if $\cpr{A}{B} = \cpr{A}{\comp{B}}$.
	\begin{solution}
		Before diving into solving the problem depending on definitions and equations,
		let us examine what this means.
		The equation means the conditional probability of $A$ given $B$
		is the same as that of $A$ given $\comp{B}$
		or in other words, the conditional probability of $A$
		assuming $B$ does not happen.
		Thus the statement of the problem
		is ``if the conditional probability remains the same
		whether or not $B$ happens,
		$A$ and $B$ are independent.''


		\newcommand{\A}{\pr{A\cap B}}
		\newcommand{\B}{\pr{B}}
		\newcommand{\C}{\pr{A\cap \comp{B}}}
		\newcommand{\D}{\pr{\comp{B}}}
		Now let us try to show whether this is true
		using equations.
		We first note that $\cpr{A}{B} = \cpr{A}{\comp{B}}$ implies
		that
		\begin{eqnarray*}
			\lefteqn{
			\cpreq{A}{B} = \cpreq{A}{\comp{B}}
			\Leftrightarrow \A \D = \B \C
			}
			\\&\Leftrightarrow& \A \D + \A \B  = \B \C + \A \B
			\\&\Leftrightarrow& \A \left( \D + \B \right)  = \B \left( \C + \A  \right)
			\\&\Leftrightarrow& \pr{A\cap B} = \pr{A} \pr{B}
		\end{eqnarray*}
		where we use the two equalities
		$\pr{B} + \pr{\comp{B}} = 1$
		and $\pr{A\cap B} + \pr{A \cap \comp{B}} = \pr{A}$.
		Therefore $A$ and $B$ are independent.
		(Indeed, $\cpr{A}{B} = \cpr{A}{\comp{B}}$ if and only if $A$ and $B$ are independent.)
	\end{solution}
	\fi

	\item LG 2.92
	\ifdefined\sol
	A random experiment is repeated a large number of times
	and the occurrence of events $A$ and $B$ is noted.
	How would you test whether events $A$ and $B$ are independent?
	\begin{solution}
		While performing the random experiment repeatedly,
		we keep calculating the relative frequencies of $A$, $B$ and $A\cap B$,
		\ie,
		$\relfreq{A}(n)$,
		$\relfreq{B}(n)$,
		and $\relfreq{A\cap B}(n)$ respectively.
		Here $\relfreq{A}(n)$ and $\relfreq{B}(n)$ represent the relative frequencies in isolation,
		and $\relfreq{A\cap B}(n)$ represents the relative frequency of the joint occurence of $A$ and $B$.
		Then test whether $\relfreq{A\cap B}(n)/\relfreq{A}(n) \relfreq{B}(n)$
		converges to $1$ as the number of the random experiments increase.
	\end{solution}
	\fi


	\item LG 2.95
	\ifdefined\sol
	In the binary communication system in Example~2.26,
	find the value of \eps\ for which the input of the channel
	is independent of the output of the channel.
	Can such a channel be used to transmit information?
	\begin{solution}
	Let $S_i$ be the event that the input is $i$
	and let $R_i$ be the event that the output is $i$.
	Let us consider $S_1$ and $R_1$.
	The theorem of total probabilities
	implies
	\[
		\pr{R_1} = \cpr{R_1}{S_0}\pr{S_0} + \cpr{R_1}{S_1} \pr{S_1}
		 = (1-p) \eps + p (1-\eps).
	\]
	Thus in order for $S_1$ and $R_1$ to be independent,
	the equation $\cpr{R_1}{S_1} = \pr{R_1}$ must hold.
	Hence,
	\[
		1-\eps = (1-p) \eps + p (1-\eps),
	\]
	which implies that $\eps = 1/2$ (unless $p=1$).
	This means the receiver does not give any information regarding
	which signal was originally sent.
	Hence such a channel cannot transmit any information.
	\end{solution}
	\fi

	\item LG 2.98
	\ifdefined\sol
	A fraction $p$ of items from a certain production line is defective.
	\begin{enumerate}
		\item What is the probability that there is more than one defective
		item in a batch of $n$ items?
		\begin{solution}
			Let $A$ be the event
			that there is zero or one defective item in a batch of $n$ item.
			Then the binomial probability law implies that
			\[
				\pr{A} = (1-p)^n + {n\choose 1}p(1-p)^{n-1}
				= (1-p)^n + n p(1-p)^{n-1}.
			\]
			Therefore the probability that there is more than one defective
			item in a batch of $n$ items is
			\[
				\pr{\comp{A}} = 1-\pr{A}
				=1 - (1-p)^n - n p(1-p)^{n-1}.
			\]
		\end{solution}


		\item During normal production $p = 10^{-3}$,
		but when production malfunctions $p = 10^{-1}$.
		Find the size of a batch that should be tested
		so that if any items are found defective,
		we are $99$\% sure that there is a production malfunction.
		\begin{solution}
		\emph{I think this problem is not properly stated.
		I apologize if you spend too much time on this.
		}
		\iffalse
			\[
				1-(1-p)^n \geq 0.99
				\Leftrightarrow
				(1-p)^n \leq 0.01
				\Leftrightarrow
				n \geq \frac{\log0.01}{\log(1-p)}
				= \frac{\log0.01}{\log(0.9)}
				= 43.71,
			\]
			hence
			we should test at least $44$ items.
		\fi

			
		\iffalse
			Let $M$ be the event that there is a production malfunction
			and let $D$ be the event that any items are found defective.
			Let $q$ be the probability that the production operates in normal mode
			and let $p_1 = 10^{-3}$ and $p_2 = 10^{-1}$.
			Then the probability that there is a production malfunctions
			given that any items are found defective is
			\begin{eqnarray*}
				\lefteqn{\cpr{M}{D} = \cpreq{M}{D}
				= \frac{\cpr{D}{M}\pr{M}}{\cpr{D}{M}\pr{M} + \cpr{D}{\comp{M}}\pr{\comp{M}}}
				}
				\\&=& \frac{(1-(1-p_2)^n)(1-q)}
				{ (1-(1-p_2)^n)(1-q) + (1-(1-p_1)^n) q }
				\geq 0.99
			\end{eqnarray*}
		\fi
		\end{solution}
	\end{enumerate}
	\fi

	\item LG 2.106
	\ifdefined\sol
	A biased coin is tossed repeatedly until heads has come up three times.
	Find the probability that $k$ tosses are required.
	\emph{Hint:}
	Show that $\{\mbox{``k tosses are required''}\} = A \cap B$, where
	$A = \{\mbox{``$k$th toss is heads''}\}$ and $B = \{ \mbox{``2 heads occurs in $k - 1$ tosses''}\}$.
	\begin{solution}
		Let $A$ be the event that ``$k$th toss is heads,''
		let $B$ be the event that ``$2$ heads occurs in $k - 1$ tosses,''
		and let $C$ be the event that ``$k$ tosses are required.''
		If it required $k$ tosses to see heads three times,
		it means that the $k$th toss was head and $2$ heads have been seen
		during the first $k-1$ tosses.
		Hence, $C = A\cap B$.

		Now let $p$ be the probability that the (biased) coin toss yields heads.
		Then the event that $2$ heads occurs in $k-1$ tosses is governed by the binomial probability law:
		\[
			\pr{B} = {k-1 \choose 2}p^2 (1-p)^{k-3}.
		\]
		Thus,
		\[
			\pr{C} = \pr{A\cap B} = \pr{A} \pr{B}
			= {k-1 \choose 2} p^3 (1-p)^{k-3}
		\]
		for $k=3,4,\ldots$ since $A$ and $B$ are independent.
	\end{solution}
	\fi

\end{enumerate}

\section*{Bonus problems}

\begin{enumerate}
	\item LG 2.26
	\ifdefined\sol
	Show that
	\[
		\pr{A\cup B\cup C}
		= \pr{A} + \pr{B} + \pr{C}
		-\pr{A\cap B} -\pr{B\cap C} -\pr{A\cap C}
		+\pr{A\cap B\cap C}.
	\]
	\begin{solution}
		Corollary 5 implies that
		\beql{eq-1}
			\pr{A\cup B\cup C}
			= \pr{A\cup (B\cup C)}
			= \pr{A} + \pr{B\cup C} - \pr{A\cap (B\cup C)}
		\eeql
		First, note that the same Corollary implies
		\beql{eq-2}
			\pr{B\cup C} = \pr{B} + \pr{C} - \pr{B\cap C}.
		\eeql
		Now the distributivity implies
		$A\cap (B\cup C) = (A\cap B) \cup (A\cap C)$
		and the same Corollary again implies
		\begin{eqnarray}
			\nonumber
			\pr{A\cap (B\cup C)}
			&=& \pr{A\cap B} + \pr {A\cap C}
			- \pr{(A\cap B)\cap( A\cap C)}
			\\ &=& \pr{A\cap B} + \pr {A\cap C} - \pr {A\cap B\cap C}.
			\label{eq-3}
		\end{eqnarray}
		Then (\ref{eq-1}), (\ref{eq-2}), and (\ref{eq-3}) imply
		\[
			\pr{A\cup B\cup B}
			= \pr{A} + \pr{B} + \pr{C} - \pr{B\cap C}
			- \pr{A\cap B} - \pr{A\cap C} + \pr{A\cap B\cap C},
		\]
		hence the proof.
	\end{solution}

	\fi

	\item LG 2.30
	\ifdefined\sol
	Use Corollary 5 to prove the following:
	\begin{enumerate}
		\item $\pr{A\cup B\cup C} \leq \pr{A} + \pr{B} + \pr{C}$.
		\begin{solution}
			Corollary 5 implies that
			\[
				\pr{A\cup B} \leq \pr{A} + \pr{B}.
			\]
			Thus
			\[
				\pr{A\cup B\cup C}
				= \pr{A\cup (B\cup C)}
				\leq \pr{A} + \pr{B\cup C}
				\leq \pr{A} + \pr{B} + \pr{C},
			\]
			hence the proof.
		\end{solution}

		\item
		\beql{eq-subadd}
			\bpr{\bigcupkton A_k } \leq \sumkton \pr{A_k}.
		\eeql
		\begin{solution}
			It is obvious that (\ref{eq-subadd}) holds for $n=1$.
			Corollary 5 implies (\ref{eq-subadd}) also holds for $n=2$.
			Now we suppose that (\ref{eq-subadd}) holds for $n=m$,
			\ie,
			\[
				\bpr{\bigcupto{k}{1}{m} A_k } \leq \sumto{k}{1}{m} \pr{A_k}.
			\]
			Then the inductive assumption and Corollary 5 imply
			\begin{eqnarray*}
				\bpr{\bigcupto{k}{1}{m+1} A_k }
				&=& \bpr{\left(\bigcupto{k}{1}{m} A_k \right) \cup A_{m+1} }
				\leq \bpr{\bigcupto{k}{1}{m} A_k} + \pr{A_{m+1}}
				\\&\leq& \sumto{k}{1}{m} \pr{A_k}  + \pr{A_{m+1}}
				\leq \sumto{k}{1}{m+1} \pr{A_k},
			\end{eqnarray*}
			hence (\ref{eq-subadd}) holds also for $n= m+1$.
			Therefore the mathematical induction implies
			(\ref{eq-subadd}) is true for all $n\geq1$.
		\end{solution}
		\item
		\beql{eq-another}
			\bpr{\bigcapkton A_k } \geq 1- \sumkton \pr{\comp{A_k}}.
		\eeql
		\begin{solution}
			Note that the DeMorgan's relues and (\ref{eq-subadd}) imply
			\[
				\bpr{\comp{\left( \bigcapkton A_k \right)}}
				= \bpr{\bigcupkton \comp{A_k} }
				\leq \sumkton \pr{\comp{A_k}}.
			\]
			Then Corollary 1 and the above inequality imply
			\[
				\bpr{\bigcapkton A_k }
				= 1 - \bpr{\comp{\left( \bigcapkton A_k \right)}}
				\geq 1 - \sumkton \pr{\comp{A_k}}.
			\]

		\end{solution}
	\end{enumerate}
	\fi

	\item LG 2.61
	\ifdefined\sol
	In this problem we derive the multinomial coefficient.
	Suppose we partition a set of $n$ distinct objects
	into $J$ subsets \xcomma{B}{J}\ of size \xcomma{k}{J},
	respectively,
	where $k_i\geq 0$, and $k_1+k_2+\cdots+k_J = n$.
	\begin{enumerate}
		\item Let $N_i$ denote the number of possible outcomes when
		the $i$th subset is selected
		in an increasing order,
		\ie,
		$k_1$ objects for $B_1$ are selected,
		then $k_2$ objects for $B_2$ are selected,
		and so on. Show that
		\[
			N_1 = {n \choose k_1},\
			N_2 = {n-k_1 \choose k_2},\
			\ldots,
			N_{J-1} = {n-k_1-\cdots-k_{J-2} \choose k_{J-1}}
		\]
		\begin{solution}
		The number of different combinations of size $k_1$
		from a set of size $n$ is $n \choose k_1$, hence
		\[
			N_1 = { n \choose k_1}.
		\]
		Now we have $n-k_1$ objects left.
		Hence $N_2$
		is the number of different combinations of size $k_2$
		from a set of size $n-k_1$, thus
		\[
			N_2 = { n-k_1 \choose k_2}.
		\]
		If we repeat this procedure until $N_J$,
		we have
		\beql{eq-4}
			N_j = { n - \sumto{i}{1}{j-1} k_i \choose k_j }
		\eeql

		for $j = 1,2, \ldots, J$.
		\end{solution}

		\item Show that the number of partitions is then:
		\[
			N_1N_2\cdots N_{J-1} = \frac{n!}{k_1!k_2!\cdots k_J!}
		\]
		\begin{solution}
		The number of possible partitions is the product of $N_j$ in (\ref{eq-4})
		from $j=1$ to $J$,
		hence
		\begin{eqnarray*}
			\lefteqn{
			N_1 N_2 \cdots N_J
			= \prod_{j=1}^J { n - \sumto{i}{1}{j-1} k_i \choose k_j }
			= \prod_{j=1}^J
			\frac{ \left( n - \sumto{i}{1}{j-1} k_i\right)!  }
			{ k_j!  \left( n - \sumto{i}{1}{j} k_i\right)!  }
			}
			\\&=&
			\frac{n!} { k_1!  \left( n - \sumto{i}{1}{1} k_i\right)!  }
			\cdot
			\frac{ \left( n - \sumto{i}{1}{1} k_i\right)!  }
			{ k_2!  \left( n - \sumto{i}{1}{2} k_i\right)!  }
			\cdots
			\frac{ \left( n - \sumto{i}{1}{J-1} k_i\right)!  }
			{ k_J!  \left( n - \sumto{i}{1}{J} k_i\right)!  }
			\\&=& \frac{n!}{k_1!k_2!\cdots k_J!},
		\end{eqnarray*}
		hence the proof.
		\end{solution}
	\end{enumerate}
	\fi

	\newcommand{\szero}{\ensuremath{S_0}}
	\newcommand{\sone}{\ensuremath{S_1}}
	\newcommand{\stwo}{\ensuremath{S_2}}
	\newcommand{\skei}{\ensuremath{S_k}}

	\newcommand{\rzero}{\ensuremath{R_0}}
	\newcommand{\rone}{\ensuremath{R_1}}
	\newcommand{\rtwo}{\ensuremath{R_2}}
	\newcommand{\reye}{\ensuremath{R_i}}

	\item LG 2.81
	\ifdefined\sol
	A ternary communication system is shown in Fig.~P2.4.
	Suppose that input symbols $0$, $1$, and $2$
	occur with probability $1/3$ respectively.
	\begin{enumerate}
		\item Find the probabilities of the output symbols.
		\begin{solution}
			Let \szero, \sone, and \stwo\
			be the events that the input is $0$, $1$, and $2$
			respectively.
			Let \rzero, \rone, and \rtwo\
			be the events that the output is $0$, $1$, and $2$
			respectively.
			Then the theorm on total probability implies
			\[
				\pr{\reye} = \sum_{k=0}^2 \cpr{\reye}{\skei} \pr{\skei}
			\]
			for $i=0,1,2$.
			Thus
			\begin{eqnarray*}
				\pr{\rzero} &=&
				\onethird \cdot (1-\eps) + \onethird \cdot 0 + \onethird \cdot \eps =1/3,
				\\\pr{\rone} &=&
				\onethird \cdot \eps + \onethird \cdot (1-\eps) + \onethird \cdot 0 =1/3,
				\\\pr{\rtwo} &=&
				\onethird \cdot 0 + \onethird \cdot \eps + \onethird \cdot (1-\eps) =1/3.
			\end{eqnarray*}
		\end{solution}

		\item Suppose that a $1$ is observed at the output.
		What is the probability that the input was $0$? $1$? $2$?
		\begin{solution}
			The Bayes' rule implies
			\begin{eqnarray*}
				\cpr{\szero}{\rone}
				&=& \frac{\cpr{\rone}{\szero}\pr{\szero}}{ \sum_{k=0}^2 \cpr{\rone}{\skei} \pr{\skei} }
				= \eps,
				\\\cpr{\sone}{\rone}
				&=& \frac{\cpr{\rone}{\sone}\pr{\sone}}{ \sum_{k=0}^2 \cpr{\rone}{\skei} \pr{\skei} }
				= 1-\eps,
				\\\cpr{\stwo}{\rone}
				&=& \frac{\cpr{\rone}{\stwo}\pr{\stwo}}{ \sum_{k=0}^2 \cpr{\rone}{\skei} \pr{\skei} }
				= 0.
			\end{eqnarray*}
		\end{solution}

	\end{enumerate}

	\fi

	\item LG 2.108
	\ifdefined\sol
	In Example~2.45, let $p_0(n)$ and $p_1(n)$ be the probabilities
	that urn $0$ or urn $1$ is used in the $n$th subexperiment.
	\begin{enumerate}
		\item Find $p_0(1)$ and $p_1(1)$.
		\begin{solution}
			Since the first draw is decided by a fair coin flip,
			\[
				p_0(1) = p_1(1) = 1/2.
			\]
		\end{solution}

		\item Express $p_0(n + 1)$ and $p_1(n + 1)$ in terms of $p_0(n)$ and $p_1(n)$.
		\begin{solution}
			Note that these sequential experiments
			are Markov chains,
			\ie, a subexperiment only depends on the previous subexperiment only.
			Hence, we can derive the following recursive formula:
			\beql{eq-markov-rec}
				p_0(n+1) = \frac{2}{3} p_0(n) +  \frac{1}{6} p_1(n)
				\mand
				p_1(n+1) = \frac{1}{3} p_0(n) +  \frac{5}{6} p_1(n).
			\eeql

			If we let 
			\[
				p(n) = \colvectwo{p_0(n)}{p_1(n)} \in \reals^2,
			\]
			the recursive formula (\ref{eq-markov-rec}) implies
			\beql{eq-markov-rec-matrix}
				p(n+1) = \underbrace{\mattwotwo{2/3}{1/6}{1/3}{5/6}}_{A} p(n),
			\eeql
			hence
			\beql{eq-sol}
				p(n) = A^{n-1} p(1).
			\eeql
			The matrix $A$ here is called the \emph{transition matrix}.

		\end{solution}

		\item Evaluate $p_0(n)$ and $p_1(n)$  for $n = 2, 3, 4$.
		\begin{solution}
			Since $p(1) = [1/2\ 1/2]^T$,
			\[
				p(2) = Ap(1) = [5/12\ 6/12]^T,\
				p(3) = Ap(2) = [3/8\ 5/8]^T,\
				p(4) = Ap(3) = [17/48\ 31/48]^T.
			\]
		\end{solution}

		\item Find the solution to the recursion in part b
		with the initial conditions given in part a.
		\begin{solution}
			It is already given by (\ref{eq-sol}).
		\end{solution}

		\item What are the urn probabilities as $n$ approaches infinity?
		\begin{solution}
			Let
			\[
				A V = V \diag(\ld_1,\ld_2) = V \Lambda
			\]
			be the eigenvalue decomposition of $A$. 
			Then
			\[
				A = V \Lambda V^{-1}.
			\]
			If we let $v_1$ and $v_2$ be the column vectors of $V$ and
			$w_1$ and $w_2$ be the row vectors of $V^{-1}$,
			\[
				A^n = \sum_{i=1}^ 2\ld_i^n v_iw_i^T.
			\]
			Since we can find $V$ and $\Lambda$ such that $\ld_1 = 1/2$ and $\ld_2 = 1$,
			\[
				\lim_{n\to\infty} A^n = v_2 w_2^T.
			\]
			Therefore
			\[
				\lim_{n\to\infty} p(n) = v_2w_2^T p(1) = \colvectwo{1/3}{2/3}
			\]
			where $\Lambda$ and $V$ above can be found by the following Python code.
			(The code was also used for solving (c).)
			
\begin{verbatim}
	from numpy import matrix
	import numpy.linalg as la

	# the transition matrix for the Markov chains
	A = matrix([[2./3,1./6],[1./3,5./6]])

	# initial probability distribution
	p1 = matrix([[.5],[.5]])

	p2 = A * p1
	p3 = A * p2
	p4 = A * p3

	print p2,p3,p4

	# eigenvalue decomposition of A
	L, V = la.eig( A ) # L: eigenvalues, V: eigenvectors
	Vi = la.inv( V ) # the inverse of V

	# examine the eigenvalues of A
	print L

	# Since the first eigenvalue is .5, and the second is 1,
	# at the equilibrium, the probabilities converge to
	# those corresponding to 1.
	# Then the equilibrium can be calculated as
	print V[:,1] * Vi[1,:] * p1
\end{verbatim}

		\end{solution}
	\end{enumerate}
	\fi


	\item A set $A$ is said to be a countably infinite set
		if and only if there exists one-to-one mapping between $A$ and $\naturals$
		where $\naturals = \{1,2,\ldots\}$.
	\begin{enumerate}
		\item Prove that the set of all the integers is countable.
		\ifdefined\sol
		\begin{solution}
			Let $\integers$ be the set of all the integers.
			We define a map (or function) $f:\naturals \to \integers$
			such that
			\[
				f(n) = \left\{\begin{array}{ll}
					(n-1)/2 &\mif n \mbox{ is an odd integer,}
					\\-n/2 &\mif n \mbox{ is an even integer,}
				\end{array}\right.
			\]
			then
			\begin{eqnarray*}
				f(1) &=& 0
				\\f(2) &=& -1
				\\f(3) &=& 1
				\\f(4) &=& -2
				\\f(5) &=& 2
				\\ &\vdots&
			\end{eqnarray*}
			Therefore $f$ is a one-to-one mapping from $\naturals$ to $\integers$,
			hence $integers$ is a countable set.
		\end{solution}
		\fi

		\newlength{\aaa}
		\newlength{\bbb}
		\settowidth{\aaa}{$2/2$}
		\settoheight{\bbb}{$2/2$}
		\newcommand{\erase}{\rule[.5\bbb]{\aaa}{.1pt} \hspace{-\aaa}}

		\item Prove that the set of all the rational numbers is countable.
		\ifdefined\sol
		\begin{solution}
			Let $\rationals$ be the set of all the rational numbers.
			Then group all the positive rational numbers
			so that
			the sum of the numerator and the denominator of each number
			are the same in the same group
			with the constraint that the numerator is relatively prime to the denominator.
			If we call such groups as $G_2$, $G_3$, \etc,
			we have
			\begin{eqnarray*}
				G_2 &=& \{ 1/1 \},
				\\G_3 &=& \{ 1/2, 2/1 \},
				\\G_4 &=& \{ 1/3, \erase 2/2, 3/1 \},
				\\G_5 &=& \{ 1/4, 2/3, 3/2, 4/1 \},
				\\G_6 &=& \{ 1/5, \erase 2/4, \erase 3/3, \erase 4/2, 5/1 \},
				\\G_7 &=& \{ 1/6, 2/5, 3/4, 4/3, 5/2, 6/1 \},
				\\    &\vdots&
			\end{eqnarray*}
			Note that $2/2$, $2/4$, $3/3$ and $4/2$ are not acceptable since
			the numerators are not relatively prime to the denominators.
			Now we define a map $f:\integers \to \rationals$ as follows.
			Let $f(1) = 0$. Then we define $f(2)$ and $f(3)$ as the values listed in $G_2$
			and define $f(4)$ and $f(5)$ as their negative values.
			Therefore
			\[
				f(2) = 1/2,\
				f(3) = 2/1,\
				f(4) = -1/2,\
				f(5) = -2/1.
			\]
			Then we repeat this with $G_3$, $G_4$, and so on.
			Then this function is one-to-one,
			hence $\rationals$ is a countable set.

		\end{solution}
		\fi


		\item Prove that $[0,1] = \set{x\in\reals}{0\leq x\leq1}$ is not countable.
		\ifdefined\sol
		\begin{solution}
			Suppose that the set is countable.
			Then there must exist a function $f:\integers\to[0,1]$ which is one-to-one.
			Now let us write the numbers $f(n)$ for $n=1,2,\ldots$
			using binary notation.  (We always use inifinite number of digits.)
			For example,
			\[
				0.5 = 0.01111111111\cdots_{(2)},\
				0.7 = 0.10110011001\cdots_{(2)},\
				1 = 0.11111111111\cdots_{(2)}.
			\]
			Now suppose that
			\begin{eqnarray*}
				f(1) &=& 0.b_{11} b_{12} b_{13} b_{14} b_{15} b_{16} \cdots_{(2)}
				\\f(2) &=& 0.b_{21} b_{22} b_{23} b_{24} b_{25} b_{26} \cdots_{(2)}
				\\f(3) &=& 0.b_{31} b_{32} b_{33} b_{34} b_{35} b_{36} \cdots_{(2)}
				\\f(4) &=& 0.b_{41} b_{42} b_{43} b_{44} b_{45} b_{46} \cdots_{(2)}
				\\ &\vdots&
			\end{eqnarray*}
			\newcommand{\bb}[1]{(1-b_{#1 #1})}
			Now we consider followng number:
			\[
				x = a_1 a_2 a_3 \cdots_{(2)}
			\]
			where $a_i = 1-b_i$. Since $a_i\in\{0,1\}$ for all $i\geq 1$,
			$x$ is properly represented with base $2$ and it is in $[0.1]$.
			However, there is not $n$ such that $f(n)=x$
			since for any $n$, $n$th digit of $f(n)$ differs from that of $x$.
			This is a contradiction to the assumption that the function is one-to-one,
			\ie, there exists $x$ such that no $n$ satisfies $f(n)=x$.
			Therefore $[0,1]$ is not a coutable set.
			\qed
		\end{solution}
		\fi

		\item Argue that $\reals$ is not countable from the fact that $[0,1]$ is not countable.
		\ifdefined\sol
		\begin{solution}
			Suppose that $\reals$ is a countable set.
			Then there exists $f:\integers \to \reals$ which is one-to-one.
			Thus, for any $x\in[0,1]\subset \reals$, there exists $n$ such that $f(n)=x$.
			However, if we list such $n$'s in an increasing order
			as $n_1$, $n_2$, \ldots,
			and define $g:\integers\to[0,1]$
			such that $g(i) = f(n_i)$,
			$g$ must be an one-to-one function from $\integers$ to $[0,1]$.
			But this contradicts to the above proof.
			Therefore $\reals$ is not a countable set.
		\end{solution}
		\fi
	\end{enumerate}

	\item Prove \corollaryname~$6$ of \S2.2, \ie,
	\begin{eqnarray}
	\nonumber
		\lefteqn{
		\bpr{ \bigcup_{k=1}^{n} A_k }
		}
		\\&=& \sum_{k=1}^n \pr{A_k}
		- \sum_{1\leq j<k\leq n} \pr{A_j\cap A_k}
		+ \sum_{1\leq i<j<k\leq n} \pr{A_i\cap A_j\cap A_k}
		\cdots
		+ (-1)^{n+1} \pr{A_1\cap \cdots \cap A_n}
		\nonumber
		\\&=& \sum_{i=1}^n (-1)^{i+1} \sum_{1\leq k_1<k_2<\cdots<k_i \leq n} \bpr{\bigcap_{j=1}^i A_{k_j}}
		\label{eq-cor-6}
	\end{eqnarray}
	\ifdefined\sol
	\begin{solution}
	The Corollary~5 implies that
	\[
		\pr{A\cup B} = \pr{A} + \pr{B} - \pr{A\cap B}.
	\]
	Thus, (\ref{eq-cor-6}) holds for $n=2$.
	Now suppose that (\ref{eq-cor-6}) holds for $n=m$.
	Then Corollary~5 and the inductive assumption imply
	\begin{eqnarray*}
		\lefteqn{
		\bpr{ \bigcup_{k=1}^{m+1} A_k }
		= \bpr{ \left( \bigcup_{k=1}^{m} A_k \right) \cup A_{m+1} }
		= \bpr{ \bigcup_{k=1}^{m} A_k } + \pr{A_{m+1}}
		 - \bpr{ \left( \bigcup_{k=1}^{m} A_k \right) \cap A_{m+1} }
		}
		\\&=& \bpr{ \bigcup_{k=1}^{m} A_k } + \pr{A_{m+1}}
		 - \bpr{ \bigcup_{k=1}^{m} (A_k \cap A_{m+1}) }
		\\&=& \sum_{i=1}^m (-1)^{i+1} \sum_{1\leq k_1<\cdots<k_i \leq m} \bpr{\bigcap_{j=1}^i A_{k_j}}
		\\&&
		- \sum_{i=1}^m (-1)^{i+1} \sum_{1\leq k_1<\cdots<k_i \leq m} \bpr{\left(\bigcap_{j=1}^i A_{k_j}\right)\cap A_{m+1}}
		 + \pr{A_{m+1}}
		\\&=& \sum_{i=2}^m (-1)^{i+1} \sum_{1\leq k_1<\cdots<k_i \leq m} \bpr{\bigcap_{j=1}^i A_{k_j}}
		\\&&- \sum_{i=1}^{m-1} (-1)^{i+1} \sum_{1\leq k_1<\cdots<k_i \leq m} \bpr{\left(\bigcap_{j=1}^i A_{k_j}\right)\cap A_{m+1}}
		\\&& + \pr{A_1} + \cdots + \pr{A_m} + \pr{A_{m+1}} - (-1)^{m+1}\bpr{A_1\cap \cdots \cap A_{m+1}}
		\\&=& \sum_{i=2}^m (-1)^{i+1} \sum_{1\leq k_1<\cdots<k_i \leq m} \bpr{\bigcap_{j=1}^i A_{k_j}}
		\\&&+ \sum_{i=2}^{m} (-1)^{i+1} \sum_{1\leq k_1<\cdots<k_{i-1} \leq m} \bpr{\left(\bigcap_{j=1}^{i-1} A_{k_j}\right)\cap A_{m+1}}
		\\&& + \pr{A_1} + \cdots + \pr{A_m} + \pr{A_{m+1}} - (-1)^{m+1}\bpr{A_1\cap \cdots \cap A_{m+1}}
		\\&=& \sum_{i=2}^m (-1)^{i+1} \sum_{1\leq k_1<\cdots<k_i \leq m} \bpr{\bigcap_{j=1}^i A_{k_j}}
		\\&&+ \sum_{i=2}^{m} (-1)^{i+1} \sum_{1\leq k_1<\cdots<k_{i-1} \leq m,\ k_i = m+1} \bpr{\bigcap_{j=1}^{i} A_{k_j}}
		\\&& + \pr{A_1} + \cdots + \pr{A_m} + \pr{A_{m+1}} - (-1)^{m+1}\bpr{A_1\cap \cdots \cap A_{m+1}}
		\\&=& \sum_{i=2}^m (-1)^{i+1} \sum_{1\leq k_1<\cdots<k_i \leq m+1} \bpr{\bigcap_{j=1}^i A_{k_j}}
		\\&& + \pr{A_1} + \cdots + \pr{A_m} + \pr{A_{m+1}} + (-1)^{m+2}\bpr{A_1\cap \cdots \cap A_{m+1}}
		\\&=& \sum_{i=1}^{m+1} (-1)^{i+1} \sum_{1\leq k_1<\cdots<k_i \leq m+1} \bpr{\bigcap_{j=1}^i A_{k_j}},
	\end{eqnarray*}
	hence, (\ref{eq-cor-6}) also holds for $n=m+1$.
	Therefore the mathematical induction implies
	(\ref{eq-cor-6}) holds for all $n\geq2$.
	\qed



	\end{solution}

\fi

\end{enumerate}

\end{document}
