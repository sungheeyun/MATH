
\chapter{Mathematics}
\label{app-math}

\section{The binomial theorem}

\begin{itemize}
	\item The binomial theorem:
	for any $a$, $b$, and $n\in\naturals$,
	\beql{eq-binom-the}
		(a+b)^n = \sumto{k}{0}{n} \chs{n}{k} a^k b^{n-k}
	\eeql
	where
	\beql{eq-comb}
		\chs{n}{k} = \frac{n!}{k!(n-k)!}
		= \frac{n(n-1)\cdots (n-k+1)}{k!}
	\eeql
	is the number of combinations
	of choosing $k$ items out of $n$ items,
	which is read as ``$n$ choose $k$'' or sometimes ``$n$ combination $k$''.
	It is sometimes denoted by $\,_nC_k$.
	(Note the convension defined $0!=1$.)
	\begin{proof}
		It is obvious that (\ref{eq-binom-the}) holds for $n=1$.
		Suppose that (\ref{eq-binom-the}) holds for $n=m$.
		Then the inductive assumption implies
		\begin{eqnarray*}
			\lefteqn{
			(a+b)^{m+1} = (a+b) (a+b)^m
			= (a+b) \sumto{k}{0}{m} \chs{m}{k} a^k b^{m-k}
			}
			\\ &=&
			\sumto{k}{0}{m} \chs{m}{k} a^k b^{m-k+1}
			+ \sumto{k}{0}{m} \chs{m}{k} a^{k+1} b^{m-k}
			\\ &=&
			\sumto{k}{0}{m} \chs{m}{k} a^k b^{m-k+1}
			+ \sumto{k'}{1}{m+1} \chs{m}{k'-1} a^{k} b^{m-k'+1}
			\\ &=&
			\sumto{k}{0}{m} \chs{m}{k} a^k b^{m-k+1}
			+ \sumto{k'}{1}{m+1} \chs{m}{k'-1} a^{k'} b^{m-k'+1}
			\\ &=&
			a^0b^{m+1}
			+ \sumto{k}{1}{m} \left( \chs{m}{k} + \chs{m}{k-1} \right) a^{k} b^{m+1-k}
			+ a^{m+1} b^0
			\\ &=&
			a^0b^{m+1}
			+ \sumto{k}{1}{m} \chs{m+1}{k} a^{k} b^{m+1-k}
			+ a^{m+1} b^0
			\\ &=&
			\sumto{k}{0}{m+1} \chs{m+1}{k} a^{k} b^{m+1-k}
		\end{eqnarray*}
		where the following formula is used for the last equality:
		\begin{eqnarray*}
			%\lefteqn{
			\chs{m}{k} + \chs{m}{k-1}
			&=& \frac{m!}{k!(m-k)!} + \frac{m!}{(k-1)!(m-k+1)!}
			%}
			\\&=&
			 \frac{m!(m-k+1)+ m!k}{k!(m-k+1)!}
			= \frac{(m+1)!}{k!(m-k+1)!}
			= \chs{m+1}{k}
		\end{eqnarray*}
		Thus, (\ref{eq-binom-the}) holds for $n=m+1$, too.
		Therefore the mathematical induction implies
		(\ref{eq-binom-the}) holds for all $n\in\naturals$.
		\qed
	\end{proof}

	\item If we differentiate both sides of (\ref{eq-binom-the}) \wrt\ $a$
	and multiplying both sides by $a$,
	we have
	\beql{eq-k-binom}
		\sumkton \chs{n}{k} ka^{k} b^{n-k}
		= na(a+b)^{n-1}.
	\eeql
	If we again differentiate both sides of (\ref{eq-k-binom}) \wrt\ $a$
	and multiplying both sides by $a$,
	we have
	\beql{eq-ks-binom}
	%	= n(n-1)a(a+b)^{n-2} + n(a+b)^{n-1}
		\sumkzton \chs{n}{k} k^2a^{k-1} b^{n-k}
		=
		na(a+b)^{n-2} ( a(n-1) + a + b ).
	\eeql
	\iffalse
	\[
		n(a+b)^{n-1} = \sumkton k \chs{n}{k} a^{k-1} b^{n-k}.
	\]
	\[
		n(n-1)(a+b)^{n-2} = \sumto{k}{2}{n} k(k-1)\chs{n}{k} a^{k-2} b^{n-k}.
	\]
	\[
		\sumto{k}{1}{n} k^2\chs{n}{k} a^{k} b^{n-k}
		 = \sumto{k}{2}{n} k(k-1)\chs{n}{k} a^{k} b^{n-k}
		 + \sumto{k}{1}{n} k\chs{n}{k} a^{k} b^{n-k}
		 \]
		 \[
		 = a^2 n(n-1) (a+b)^{n-2} + na(a+b)^{n-1}
		 = na(a+b)^{n-2} ( a(n-1) + (a+b))
	\]
	\fi


\end{itemize}



\section{Inifinite sequences and series}

All the following equations hold only for $x\in(-1,1)$.
\begin{eqnarray}
\label{eq-is-xk}
	\sumkztoi x^k
	&=& \frac{1}{1-x},
\\ \label{eq-is-kxk-2}
	\sumktoi k x^{k-1}
	&=& \frac{1}{(1-x)^2},
\\ \label{eq-is-ksxk-2}
	\sumto{k}{2}{\infty} k (k-1)x^{k-2}
	&=& \frac{2}{(1-x)^3},
\\ \label{eq-is-kcxk-2}
	\sumto{k}{3}{\infty} k (k-1)(k-2)  x^{k-3}
	&=& \frac{6}{(1-x)^4},
\\ \label{eq-is-kmxk-2}
	\sumto{k}{m}{\infty} k (k-1) \cdots (k-m+1)  x^{k-m}
	&=& \frac{m!}{(1-x)^{m+1}}.
\end{eqnarray}

Multiplying $x$ on both sides of (\ref{eq-is-kxk-2}) yields
\beql{eq-is-kxk}
	\sumktoi k x^k = \frac{x}{(1-x)^2}.
\eeql

Multiplying $x^2$ on both sides of (\ref{eq-is-ksxk-2}) and adding it to (\ref{eq-is-kxk})
yields
\beql{eq-is-ksxk}
	\sumktoi k^2 x^k = \frac{2x^2}{(1-x)^3} + \frac{x}{(1-x)^2}.
	= \frac{2x^2+x(1-x)}{(1-x)^3}
	= \frac{x^2+x}{(1-x)^3}
	= \frac{x(1+x)}{(1-x)^3}
\eeql

\section{Taylor series}

\begin{itemize}
	\item The Taylor series of a real or complex function $f(x)$ that is
	infinitely differentiable in a neighborhood of a real or complex number $a$
	is the power series
	\beql{eq-taylor-series}
		\sumnztoi \frac{f^{(n)}(a)}{n!} (x-a)^n
		=
		f(a)
		+ \frac{f'(a)}{1!} (x-a)
		+ \frac{f''(a)}{2!} (x-a)^2
		+ \frac{f'''(a)}{3!} (x-a)^3
		+ \cdots
	\eeql
	where $n!$ denotes the factorial of $n$ and
	$f^{(n)}(a)$ denotes the $n$th derivative of $f$ evaluated at the point $a$.
	The zeroth derivative of $f$ is defined to be $f$ itself and
	$(x - a)^0$ and $0!$ are both defined to be $1$.
	In the case that a = 0, the series is also called a \emph{Maclaurin series}.

	\item A function is \emph{analytic} in an open disc centered at $a$
	if and only if its Taylor series converges to the value of the function
	at each point of the disc.

	\item If $f(x)$ is equal to its Taylor series everywhere
	it is called \emph{entire}.
	\begin{itemize}
		\item The polynomials and the exponential function $e^x$ and
		the trigonometric functions (sine and cosine)
		are examples of entire functions.
		\item Examples of functions that are not entire include the logarithm,
		the trigonometric function tangent,
		and its inverse arctan.
		For these functions the Taylor series do not converge if x is far from a.
		\item Taylor series can be used to calculate the value of an entire function
		in every point, if the value of the function,
		and of all of its derivatives, are known at a single point.
	\end{itemize}

	\item The exponential function is entire,
	thus
	\beql{eq-taylor-exp}
		\exp(x) = \sumnztoi \frac{1}{n!} x^n
	\eeql
	for all $x\in\complexes$.
\end{itemize}


\chapter{Proofs}
\label{app-proofs}

\begin{itemize}
	\item Alternative proofs for the expected value and the \var\ of \binomrv\ in \S\ref{loc-binom-alter-proof}
	\begin{enumerate}
		\item By regarding the \binomrv\ as a sum of $n$ independent and identically distributed
		\bernrv s:
		\begin{proof}
		Let \X\ be the \binomrv\ with $n$ and $p$.
		If we let $X_i$ for $i=1,2,\ldots,n$
		be the $n$ independent and identically distributed \bernrv s with $p$,
		\[
			X = \sumkton X_i.
		\]
		Then the linearity of the expected value operator
		implies
		\[
			\Exp{X} = \sumkton \Exp{X_i} = np
		\]
		and the independence of $X_i$ implies
		\[
			\Exp{X^2} = \sumkton \Exp{X_i^2} + 2\sum_{i<j} \Exp{X_i}\Exp{X_j}
			= n(p(1-p)+p^2) + n(n-1)p^2
			= np(1-p) + (np)^2,
		\]
		thus
		\[
			\VAR{X} = \Exp{X^2} - \Exp{X}^2
			= np(1-p),
		\]
		hence the proof.
		\end{proof}
		
		\item Using (\ref{eq-k-binom}) and (\ref{eq-ks-binom}):
		\begin{proof}
		Let \X\ be the \binomrv\ with $n$ and $p$.
		Then (\ref{eq-k-binom}) implies
		\[
			\Exp{X} = \sumkzton k \dbinompmf
			= np (p+(1-p))^{n-1} = np
		\]
		and (\ref{eq-ks-binom}) implies
		\[
			\Exp{X^2} = \sumkzton k^2 \dbinompmf
			 = np(p(n-1)+1)
			 = (np)^2 + np(1-p),
		\]
		thus
		\[
			\VAR{X} = \Exp{X^2} - \Exp{X}^2
			= np(1-p),
		\]
		hence the proof.
		\end{proof}
	\end{enumerate}
\end{itemize}
