
\chapter{Basic Concepts of Probability Theory}

\section{Specifying Random Experiments}

\bit
	\item A \emph{random experiment} is an experiment
	in which the outcome \emph{varies in an unpredictable fashion}
	when the experiment is repeated under the same conditions.
	A random experiment is specified by stating an experimental procedure and a set of one or more measurements or observations.

	\item \lgexam{2.1}{}
	\bit
		\item Experiment \ep{1}: Select a ball from an urn containing balls numbered 1 to 50. Note the number of the ball.
		\item Experiment \ep{2}: Select a ball from an urn containing balls numbered 1 to 4. Suppose that balls 1 and 2 are black and that balls 3 and 4 are white. Note the number and color of the ball you select.
		\item Experiment \ep{3}: Toss a coin three times and note the sequence of heads and tails.
		\item Experiment \ep{4}: Toss a coin three times and note the number of heads.
		\item Experiment \ep{5}: Count the number of voice packets containing only silence produced from a group of $N$ speakers in a $10$-ms period.
		\item Experiment \ep{6}: A block of information is transmitted repeatedly over a noisy channel until an error-free block arrives at the receiver. Count the number of transmissions required.
		\item Experiment \ep{7}: Pick a number at random between zero and one.
		\item Experiment \ep{8}: Measure the time between page requests in a Web server.
		\item Experiment \ep{9}: Measure the lifetime of a given computer memory chip in a specified environment.
		\item Experiment \ep{10}: Determine the value of an audio signal at time $t_1$.
		\item Experiment \ep{11}: Determine the values of an audio signal at times $t_1$ and $t_2$.
		\item Experiment \ep{12}: Pick two numbers at random between zero and one.
		\item Experiment \ep{13}: Pick a number $X$ at random between zero and one, then pick a number $Y$ at random between zero and $X$.
		\item Experiment \ep{14}: A system component is installed at time $t = 0$. For $t \geq 0$ let $X(t) = 1$ as long as the component is functioning, and let $X(t) = 0$ after the component fails.
	\eit


	\item The specification of a random experiment must include an \emph{unambiguous} statement of \emph{exactly what is measured or observed}. For example, random experiments may consist of the same procedure but differ in the observations made, as illustrated by \ep{3} and \ep{4}.

	\item The \emph{sample space} $S$ of a random experiment is defined as the set of all possible outcomes.  We can
	\begin{enumerate}
		\item list all the elements, separated by commas, inside a pair of braces:
		\[ A = \{0,1,2,3\},\]
		\item give a property that specifies the elements of the set:
		\[ A = \set{x}{x \mbox{ is an integer such that } 0\leq x\leq 3}.\]
	\end{enumerate}

	\item The sample spaces corresponding to the experiments in \lgexamref{2.1} are given below using set notation:
	\begin{eqnarray*}
		S_1 &=& \{ 1,2,\ldots, 50\}
		\\S_2 &=& \{(1,b),(2,b),(3,w),(4,w)\}
		\\S_3 &=& \{HHH, HHT, HTH, THH, TTH, THT, HTT, TTT\}
		\\S_4 &=& \{0,1,2,3\}
		\\S_5 &=& \{0,1,2,\ldots,N\}
		\\S_6 &=& \{1,2,3,\ldots\}
		\\S_7 &=& \set{x}{0\leq x\leq 1} = [0,1]
		\\S_8 &=& \set{t}{t\geq 0} = [0,\infty)
		\\S_9 &=& \set{t}{t\geq 0} = [0,\infty)
		\\S_{10} &=& \reals = \set{v}{-\infty < v < \infty } = (-\infty,\infty)
		\\S_{11} &=& \reals^2 = \set{(v_1,v_2)}{ -\infty < v_1 < \infty \mbox{ and } -\infty < v_2 < \infty }
		\\S_{12} &=& \set{(x,y)}{0\leq x\leq 1 \mbox{ and } 0\leq y\leq 1}
		\\S_{13} &=& \set{(x,y)}{0\leq y\leq x\leq 1}
		\\S_{14} &=& \mbox{set of functions $X(t)$ for which }
			X(t) = 1 \mbox{ for } 0\leq t < t_0
			\mbox{ and } X(t) = 0
		\\&& \mbox{ for } t\geq t_0
		\mbox{ where } t_0 > 0 \mbox{ is the time when the component fails.}
	\end{eqnarray*}

	\item The three types of sample space:
	\begin{enumerate}
		\item finite sample space, \eg,
			\ep{1}, \ep{2}, \ep{3}, \ep{4}, and \ep{5}
		\item discrete sample space or countably infinite sample space, \eg, \ep{6}
		\item continous sample space, \eg, \ep{7} through \ep{13}
	\end{enumerate}

	\item \emph{Events} correspond to subsets of $S$.
	\bit
		\item For example,
			\begin{enumerate}
				\item $\{HHH,HHT\}$ is an event of \ep{3}
				\item $[2,\infty)$ is an event of \ep{8}
			\end{enumerate}
		\item Two special events
			\begin{enumerate}
				\item $S$: the sample space itself
				\item $\emptyset$: impossible or null event
			\end{enumerate}
	\eit

	\item Event examples: \ev{k} refers to an event corresponding to Experiment \ep{k}
	\bit
		\item \ep{1}: ``An even-numbered ball is selected,'' $\ev{1} = \{2, 4, \ldots, 48, 50\}$.
		\item \ep{2}: ``The ball is white and even-numbered,'' $\ev{2} = \{(4, w)\}$.
		\item \ep{3}: ``The three tosses give the same outcome,'' $\ev{3} = \{HHH, TTT\}$.
		\item \ep{4}: ``The number of heads equals the number of tails,'' $\ev{4} = \emptyset$.
		\item \ep{5}: ``No active packets are produced,'' $\ev{5} = \{0\}$.
		\item \ep{6}: ``Fewer than $10$ transmissions are required,'' $\ev{6} = \{1, \ldots, 9\}$.
		\item \ep{7}: ``The number selected is nonnegative,'' $\ev{7} = S_7$.
	\eit

	\item Elementary event: an event from a discrete sample space that consists of a single outcome

\eit

\paragraph{Review of Set Theory}

\bit
	\item A \emph{set} is a collection of objects and will be denoted by capital
	letters $S$, $A$, $B$, \ldots\ We define $U$ as the \emph{universal set}
	that consists of all possible objects of interest in a given setting or application.
	In the context of random experiments we refer to the universal set as
	the \emph{sample space}.


	\item A \emph{set} $A$ is a collection of objects from $U$,
	and these objects are called the \emph{elements} or \emph{points} of the set $A$
	and will be denoted by lowercase letters, $\zeta$, $a$, $b$, $x$, $y$, \ldots\
	We use the notation:
	\[
		x \in A \mbox{ and } x \notin A
	\]

	\item A \emph{Venn diagram} is an illustration of sets and their interrelationships.


	\item We say \sA\ is a \emph{subset} of \sB\ if every element of \sA\ also belongs to \sB,
	that is, if $x\in\sA$ implies $x\in\sB$.
	We say that ``\sA\ is contained in \sB'' and we write:
	\beq A \subset B \eeq
	or
	\beq A \subseteq B. \eeq

	\item The \emph{empty set} $\emptyset$ is defined as the set with no elements.

	\item We say that \emph{\sA\ and \sB\ are equal} if they contain the same elements.
	\beq
		\sA = \sB \mbox{ if and only if } \sA \subset \sB \mbox{ and } \sB \subset \sA.
	\eeq

	\item Three basic operations on sets: \emph{union}, \emph{intersection}, and \emph{complement}
	\bit
		\item union
			\beq A \cup B = \set{x}{ x\in A \mbox{ or } x \in B}\eeq

		\item intersection
			\beq A \cap B = \set{x}{ x\in A \mbox{ and } x \in B}\eeq

		\item complement
			\beq \comp{A} = \set{x}{x \notin A} \eeq
	\eit

	\item \emph{Relative complement} or \emph{difference}
	\beq
		A - B = \diff{A}{B} = \set{x}{x\in A \mbox{ and } x \notin B}
	\eeq

	\item Properties of set operations
	\bit
		\item commutativity
		\beq
%			A \cup B = B\cup A \mand  A \cap B = B \cap A.
			A \cup B = B\cup A \mbox{ and }  A \cap B = B \cap A.
		\eeq

		\item associativity
		\beq
			A \cup ( B \cup C ) = (A \cup B ) \cup C
%			\mand
			\mbox{ and }
			A \cap ( B \cap C ) = (A \cap B ) \cap C
		\eeq

		\item distributivity
		\beq
			A \cup ( B \cap C) = ( A \cup B ) \cap ( A \cup C )
%			\mand
			\mbox{ and }
			A \cap ( B \cup C) = ( A \cap B ) \cup ( A \cap C )
		\eeq
	\eit

	\item DeMorgan's rules
		\beq
			\comp{(A\cup B)} = \comp{A} \cap \comp{B}
%			\mand
			\mbox{ and }
			\comp{(A\cap B)} = \comp{A} \cup \comp{B}
		\eeq

	\item \lgexam{2.7}{}


	\item The union and intersection operations can be repeated
		for an arbitrary number of sets.
		\bit
		\item intersection
			\bit
			\item for finite number of sets
				\beq
					\bigcapkton A_k = A_1 \cap \cdots \cap A_n
				\eeq
			\item for countably infinite sequence of sets
				\beq
					\bigcap_{k=1}^{\infty} A_k
				\eeq
			\eit
		\item union
			\bit
			\item for finite number of sets
				\beq
					\bigcupkton A_k = A_1 \cup \cdots \cup A_n
				\eeq
			\item for countably infinite sequence of sets
				\beq
					\bigcup_{k=1}^{\infty} A_k
				\eeq
			\eit
		\eit

	\item countable unions and intersections of sets are essential in dealing with sample spaces that are not finite.

	\item \emph{Event class} \evcl\ is the set of events of interest.
	\bit
		\item \evcl\ is a \emph{set of subsets} of the sample space $S$
		\item The three set operations, \ie, complements, countable unions, and countable intersections,
		result in events in \evcl


	\eit

	\item \emph{Borel field} is the set of subsets in $\reals$
	which can be generated by the three set operations from intervals of real line. (later\ldots)
	\item \emph{power set of $S$}: when $S=\{1,2,\ldots,k\}$ is a finite sample space,
	the of all the subsets of $S$ is called the power set of $S$ and
	it is denoted by $2^S$.

	\item \lgexam{2.8}{}
\eit

\section{The Axioms of Probability}
\bit
	\item A \emph{probability law} for a random experiment
	is a rule that assigns probabilities to the events
	of the experiment that belong to the event class \evcl.
	\beq
		P: \evcl \to \preals.
	\eeq
	For an event $A \in \evcl$, a number $\pr{A}$ is called the \emph{probability of $A$}.

	\item The probability law must satisfy the following axioms:
	\bit
		\item Axiom I: \beql{eq-ax-1} \pr{A} \geq 0. \eeql
		\item Axiom II: \beql{eq-ax-2} \pr{S} = 1.\eeql
		\item Axiom III: If $A \cap B = \emptyset$, then
			\beql{eq-ax-3} \pr{A\cup B} = \pr{A} + \pr{B}. \eeql
		\item Axiom III': If $A_1,A_2, \ldots$ is a sequence of mutually exclusive events,
			\ie, $A_i \cap A_j = \emptyset$ for all $i\neq j$, then
			\beql{eq-ax-3-1}
				\bpr{ \bigcup_{k=1}^\infty A_k } = \sum_{k=1}^\infty \pr{A_k}.
			\eeql

	\eit
\eit

\begin{coro}
	\beq \pr{\comp{A}} = 1 - \pr{A} \eeq
\end{coro}

\begin{coro}
	\beq \pr{A} \leq 1 \eeq
\end{coro}

\begin{coro}
	\beq \pr{\emptyset} = 0  \eeq
\end{coro}

\begin{coro}
\label{coro-finite-me}
	If $A_1,A_2,\ldots,A_n$ are mutually exclusive, then
	\beq \bpr{ \bigcup_{k=1}^{n} A_k } = \sum_{k=1}^n \pr{A_k} \mfor n\geq 2. \eeq
\end{coro}

\begin{coro}
	\beq
		\pr{A\cup B} = \pr{A} + \pr{B} - \pr{A\cap B}.
	\eeq
\end{coro}

\begin{coro}
	\beq
		\bpr{ \bigcup_{k=1}^{n} A_k }
		= \sum_{k=1}^n \pr{A_k}
		- \sum_{j<k}^n \pr{A_j\cap A_k}
		+ \sum_{i<j<k}^n \pr{A_i\cap A_j\cap A_k}
		\cdots
		+ (-1)^{n+1} \pr{A_1\cap \cdots \cap A_n},
	\eeq
\end{coro}
from which it is follows that
\beq
	\pr{A\cup B} \leq \pr{A} + \pr{B}.
\eeq
This results is frequently used to obtain
upper bounds for probabilities of interest.

\begin{coro}
If $A\subset B$, then
\beq
	\pr{A} \leq \pr{B}.
\eeq
\end{coro}

However,
we still
\emph{need an initial probability assignment for some basic set of events}
from which the probability of all other events can be computed.

\paragraph{Discrete Sample Spaces}

\bit
	\item The \emph{probability law} for an experiment
	with a countable sample space
	can be specified by giving
	the \emph{probabilities of the elementary events}.

	\item \emph{Finite case:} Suppose that $S= \br{a_1, a_2,\ldots,a_n}$
	and let \evcl\ consist of all subsets of $S$.
	Then for any event $B=\br{a_1', a_2', \ldots, a_m'}\in\evcl$,
	\beq
		\pr{B}=\pr{\br{a_1', a_2', \ldots, a_m'}}
		=\pr{\br{a_1'}} +\pr{\br{a_1'}} + \cdots +\pr{\br{a_m'}}
	\eeq

	\begin{itemize}
		\item Of particular interest is the case of \emph{equally likely outcomes}.
		\beq
			\pr{\br{a_1}} = \pr{\br{a_2}} = \cdots = \pr{\br{a_n}} = \frac{1}{n}.
		\eeq
		If $B=\br{a_1', a_2', \ldots, a_k'}\in\evcl$,
		\beq
			\pr{B}=\pr{\br{a_1', a_2', \ldots, a_k'}}
			=\pr{\br{a_1'}} +\pr{\br{a_1'}} + \cdots +\pr{\br{a_k'}}
			= \frac{k}{n},
		\eeq
		\ie,
		if outcomes are equally likely,
		then the probability of an event is equal to the number of outcomes
		in the event divided by the total number of outcomes in the sample space.
	\end{itemize}

	\item \emph{Countably infinite case:}
		Suppose $S=\br{a_1,a_2,\ldots}$.
		Axiom III' implies
		that if $D = \br{a_1',a_2',\ldots}$,
		\beq
		\pr{D}=\pr{\br{a_1', a_2', a_3', \ldots}}
		=\pr{\br{a_1'}} +\pr{\br{a_2'}} + \pr{\br{a_3'}} + \cdots,
		\eeq
		\ie,
		the probability of an event with a countably infinite sample space
		is determined from the infinite (or finite) sum of
		the probabilities of the elementary events.

	\item \lgexam{2.10}{} a coin is tossed three times.

	\item \lgexam{2.11}{} a coin is tossed repeatedly until the first heads shows up.

\eit

\paragraph{Continuous Sample Spaces}

\bit

	\item All the subset of $\reals$ is too large!
	\item The \emph{Borel field}, \borel, contains all open and closed intervals of the real line
	as well as all events that can be obtained as countable unions, intersections,
	and complements.
	\emph{Axiom III' is once again the key to calculating probabilities of events},
	\ie,
	\[
		\bpr{\bigcup_{k=1}^\infty A_k} = \sum_{k=1}^\infty \pr{A_k}
	\]
	for a sequence of mutually exclusive events
	that are represented by intervals of the real line.
	For this reason, probability laws in experiments with continuous sample spaces
	specify a rule for assigning numbers to \emph{intervals of the real line}.

	\item \emph{The probability that the outcome takes on a specific value is zero.}
	Why?

	\item \lgexam{2.13}{}
	The proportion of chips whose lifetime exceeds $t$
	decreases exponentially at a rate $\alpha$.
	Then
	\beq
		\pr{(t,\infty)} = e^{-\alpha t}
	\eeq
	for $t>0$.

\eit

\paragraph{How do we proceed from a problem statement to its probability model?}

\begin{enumerate}
	\item The problem statement implicitly or explicitly \emph{defines a random experiment},
	which specifies an experimental procedure and a set of measurements and observations.
	These \emph{measurements and observations determine the set of all possible outcomes}
	and hence the sample space \sspace.

	\item An \emph{initial probability} assignment that specifies the probability of certain events
	must be determined next. This probability assignment must satisfy the axioms of probability.
	\begin{itemize}
		\item If \sspace\ is \emph{discrete},
		then it suffices to specify the \emph{probabilities of elementary events}.
		\item If \sspace\ is \emph{continuous},
		it suffices to specify the \emph{probabilities of intervals of the real line
		or regions of the plane}.
	\end{itemize}

	\item The probability of other events of interest can then be determined
	from the \emph{initial probability assignment}
	and \emph{the axioms of probability} and their corollaries.
\end{enumerate}

\section{Computing Probabilities Using Counting Methods \optional}

This section is optional.
However, reading it carefully
will make it easier for you to understand subsequent sections;
I encourage you to read it if you have time.

This section deals with
\bit
	\item permutations
	\item $n$ factorial:  $n! = 1\cdot2\cdots (n-1)n$
	\item Stirling's formula
	\beql{eq-stirling}
		n! \sim \sqrt{2\pi} n^{n+1/2} e^{-n}
	\eeql

	\item binomial coefficient:
	\beql{eq-binom-coef}
		{n \choose{k}} = C_k^n = \frac{n!}{k!(n-k)!}
	\eeql
	for $0\leq k\leq n$ with the convention $0!=1$.

	\item multinomial coefficient:
	\beql{eq-multinom-coef}
		\frac{n!}{k_1!\cdots k_m!}
	\eeql
	where $k1+\cdots+k_m = n$.

\eit


\section{Conditional Probability}

\bit
	\item \emph{Conditional probability},
	$\cpr{A}{B}$,
	is the probability of event \sA\ given that even \sB\ has occurred.
	\beql{eq-cond-prob-1}
		\cpr{A}{B} = \frac{\pr{A\cap B}}{\pr{B}}
		\mfor
		\pr{B} > 0.
	\eeql
	This implies
	\beql{eq-cond-prob-2}
		\pr{A\cap B} = \cpr{A}{B} \pr{B}
	\eeql
	and by symmetry
	\beql{eq-cond-prob-3}
		\pr{A\cap B} = \cpr{B}{A} \pr{A}.
	\eeql


	\item Please read and understand (thoroughly) \lgexamref{2.24} and \lgexamref{2.25}.

	\item \lgexam{2.26}{Binary Communication System}


	\item Let $B_1$, \ldots, $B_n$ be mutually exclusive events whose
	union equals the sample space \sspace\ as shown in \lgfig{2.12},
	\ie,
	\beql{eq-partition}
		\bigcup_{k=1}^n B_k = \sspace
%		\mand
		\mbox{ and }
		B_i \cap B_j = \emptyset
		\mforall
		i\neq j.
	\eeql
	We refer to these sets as a \emph{partition} of \sspace.

	\item Any event \sA\ can be represented as the union
	of mutually exclusive events in the following way:
	\beq
		A = A \cap \sspace
		= A \cap \bigcupkton B_k
		= \bigcupkton ( A \cap B_k)
	\eeq
	by the distributivity.
	\corollaryname~\ref{coro-finite-me}
	implies that
	\beql{eq-partition-prob}
		\pr{A} = \sumkton \pr{A\cap B_k}
		= \pr{A\cap B_1} + \cdots + \pr{A \cap B_n}.
	\eeql

	\item \emph{Theorem on total probability}:
	(\ref{eq-partition-prob}) and (\ref{eq-cond-prob-2})
	imply
	\beql{eq-total-prob}
		\pr{A} = \sumkton \appr{A}{B_k}
		= \appr{A}{B_1} + \cdots + \appr{A }{B_n}.
	\eeql

	\item \lgexam{2.28}{}
\eit

\paragraph{Bayes' Rule}

\bit
	\item
	Let \Bi{1}, \ldots, \Bi{n}\ be a partition of a sample space \sspace.
	Suppose that event \sA\ occurs.
	Then the probability of event \Bi{j} is
	\beql{eq-bayes-rule}
		\cpr{B_j}{A}
		= \frac{\pr{B_j \cap A}}{\pr{A}}
		= \frac{\appr{B_j}{A}}
			{ \sumkton \appr{A}{B_k}}.
	\eeql
	where the theorem on total probability is used
	and (\ref{eq-bayes-rule}) is called \emph{Bayes' rule}.

	\item
	Bayes' rule is often applied in the following situation.
	We have some random experiment in which the events of interest
	form a partition.
	The \emph{``a priori probabilities''} of these events,
	\pr{B_j}, are the probabilities of the events
	before the experiment is performed.
	Now suppose that the experiment is performed,
	and we are informed that event \sA\ occurred;
	the \emph{``a posteriori probabilities''}
	are the probabilities of the events in the partition,
	\cpr{B_j}{A}, given this additional information.


	\item \lgexam{2.30}{Quality Control}

\eit

\section{Independence of Events}

\bit
	\item If knowledge of the occurrence of an event \sB\
	does not alter the probability of some other event \sA,
	then it would be natural to say that event \sA\ is independent of \sB.
	In terms of probabilities this situation occurs when
	\beq
		\pr{A} = \cpr{A}{B} = \frac{\pr{A \cap B}}{\pr{B}}
	\eeq
	We will define two events \sA\ and \sB\ to be \emph{independent}
	if
	\beql{eq-indep-1}
		\pr{A\cap B} = \pr{A} \pr{B}.
	\eeql
	Note that if $\pr{B}\neq 0$, (\ref{eq-indep-1}) with (\ref{eq-cond-prob-2}) implies
	\beql{eq-indep-2}
		\cpr{A}{B} = \pr{A}
	\eeql
	and if $\pr{A}\neq 0$, (\ref{eq-indep-1}) with (\ref{eq-cond-prob-3}) implies
	\beql{eq-indep-3}
		\cpr{B}{A} = \pr{B}.
	\eeql

	\item In general
	if two events have nonzero probability and are mutually exclusive,
	then they cannot be independent.
	For suppose they were independent and mutually exclusive;
	then
	\beq
		0 = \pr{A \cap B} = \pr{A} \pr{B},
	\eeq
	which implies that at least one of the events must have zero probability.

	\item \lgexam{2.32}{}


	\item \emph{Independence of three events}
	$A$, $B$, and $C$:
	\begin{itemize}
		\item pairwise independence:
		\beq
			\pr{A\cap B} = \pr{A} \pr{B},
			\pr{B\cap C} = \pr{B} \pr{C},
%			\mand
			\mbox{ and }
			\pr{C\cap A} = \pr{C} \pr{A}.
		\eeq

		\item The knowledge of the joint occurrence of any two,
		say \sA\ and \sB,
		should not affect the probability of the third,
		\ie,
		\beq
			\pr{C} = \cpr{C}{A\cap B}
			\iff
			\pr{C} = \frac{\pr{A\cap B\cap C}}{\pr{A\cap B}}
				= \frac{\pr{A\cap B\cap C}}{\pr{A}\pr{B}},
		\eeq
		hence
		\beq
			\pr{A\cap B\cap C} = \pr{A} \pr{B} \pr{C}.
		\eeq

		\item \emph{Independence of $n$ events}:
		The events \Ai{1}, \ldots, \Ai{n}\ are said to be independent
		if for any $k=2,\ldots,n$ and any $1\leq i_1 < i_2 < \cdots < i_k \leq n$,
		\beql{eq-n-indep}
			\pr{ \Ai{i_1} \cap \cdots \cap \Ai{i_k} }
			 = \pr{ \Ai{i_1}} \cdots \pr{\Ai{i_k} }.
		\eeql
		We must check $2^n-n-1$ combinations!

		\item
		Most common application
		of the independence concept is
		in making the assumption that the events of separate experiments are independent.
		We refer to such experiments as \emph{independent experiments}.

	\end{itemize}

\eit

\section{Sequential Experiments}
\bit
	\item
	Many random experiments can be viewed as sequential experiments
	that consist of a sequence of simpler subexperiments.
\eit

\paragraph{Sequences of Independent Experiments}

\bit
	\item Suppose that a \emph{random experiment consists of performing experiments}
	\Ei{1}, \ldots, \Ei{n}.
	The outcome of this experiment will then be an \emph{$n$-tuple}
	\beq
		s = (s_1, ..., s_n),
	\eeq
	where $s_k$ is the outcome of the $k$th subexperiment.
	The sample space of the sequential experiment is defined
	as the set that contains the above $n$-tuples
	and is denoted by the \emph{Cartesian product}
	of the individual sample spaces
	\beql{eq-cart-prod}
		\sspace_1 \times \cdots \times \sspace_n.
	\eeql

	\item
	We can usually determine, \emph{because of physical considerations,}
	when the subexperiments are independent,
	in the sense that the outcome of any given subexperiment
	cannot affect the outcomes of the other subexperiments.
	Let \Ai{1}, \ldots, \Ai{n}\ be events such that \Ai{k}\ concerns
	only the outcome of the $k$th subexperiment.
	\emph{If the subexperiments are independent,
	then it is reasonable to assume that
	the above events \Ai{1}, \ldots, \Ai{n}\ are independent.}
	Thus
	\beq
		\pr{A_1\cap \cdots\cap A_n} = \pr{A_1} \cdots \pr{A_n}.
	\eeq
	\emph{This expression allows us to compute all probabilities
	of events of the sequential experiment.}

\eit

\paragraph{The Binomial Probability Law}
\bit
	\item A \emph{Bernoulli} trial involves performing an experiment once and noting
	whether a particular event $A$ occurs.
	The outcome of the Bernoulli trial is said to be
	a ``success'' if $A$ occurs and a ``failure'' otherwise.


	\item
	Let $k$ be the number of successes in $n$ independent Bernoulli trials,
	then the probabilities of $k$ are given by the \emph{binomial probability law:}
	\beql{eq-binom-prob-law}
		p_n(k) = {n \choose{k}} p^k (1-p)^{n-k}
		\mfor
		k=0,\ldots,n,
	\eeql
	where $p_n(k)$ is the probability of $k$ successes in $n$ trials,
	and
	\beql{eq-comb-1}
		{n \choose{k}} = \frac{n!}{k!(n-k)!}.
	\eeql
	is the \emph{binomial coefficient}.

	\item The \emph{binomial theorem} implies
	\beql{eq-binom-theorem}
		(a+b)^n = \sumkton {n\choose{k}} a^kb^{n-k},
	\eeql
	hence
	\beq
		\sumkton p_n(k)
		= \sumkton {n \choose{k}} p^k (1-p)^{n-k}
		= (p + (1-p))^n = 1
	\eeq
	confirming that \emph{the binomial probabilities sum to $1$}.

	\item recursive formula for the binomial probability:
	\beql{eq-binom-prob-recurs}
		p_n(k+1) = \frac{(n-k)p}{(k+1)(1-p)} p_n(k).
	\eeql

	\item \lgexam{2.40}{Error Correction Coding}
\eit

\paragraph{The Multinomial Probability Law}

\bit
	\item Let \Bi{1}, \Bi{2}, \ldots, \Bi{M}\
	be a partition of the sample space \sspace\ of some random experiment
	and let $\pr{B_j} = p_j$.
	The events are \emph{mutually exclusive}, so
	\beq
		p_1 + p_2 + ... + p_M = 1.
	\eeq
	Suppose that $n$ independent repetitions of the experiment are performed.
	Let $k_j$ be the number of times event \Bi{j}\ occurs,
	then the vector $(\xcomma{k}{M})$
	specifies the number of times each of the events \Bi{j}\ occurs.
	The probability of the vector $(\xcomma{k}{M})$
	satisfies the \emph{multinomial probability law:}
	\beql{eq-multinom-prob-law}
		\pr{(\xcomma{k}{M})} = \frac{n!}{k_1!k_2!\cdots k_M!}
		p_1^{k_1} p_2^{k_2} \cdots p_M^{k_M}
	\eeql
	where $\xplus{k}{M} = n$.
	It reduces to the binomial probability law when $M=2$.
\eit

\paragraph{The Geometric Probability Law}

\bit
	\item Consider a sequential experiment
	in which we \emph{repeat independent Bernoulli trials until the occurrence of the first success}.
	Let the outcome of this experiment be $m$,
	the number of trials carried out until the occurrence of the first success.
	Then
	\beql{eq-geo-prob-law}
		p(m) = \pr{ \comp{A_1} \comp{A_2} \cdots \comp{A_{m-1}} A_m}
		= (1-p)^{m-1} p
		\mfor
		m \in \{1,2,\ldots\}.
	\eeql
	This is called the \emph{geometric probability law}.

	\item The formula for the sum of infinite geometric sequence
	gives
	\beq
		\sumto{m}{1}{\infty} p(m) = \sumto{m}{1}{\infty} p q^{m-1} = \frac{p}{1-q} = 1
	\eeq
	where $q=1-p$.
	Also
	\beq
		\pr{m>K}
		= \sumto{m}{K+1}{\infty} p(m) = \sumto{m}{K+1}{\infty} p q^{m-1}
		= \frac{pq^k}{1-q} = q^k.
	\eeq

	\item \lgexam{2.43}{Error Control By Retransmission}

\eit

\paragraph{Sequences of Dependent Experiments}

\bit
	\item
	Suppose the sequence of outcomes $s_1,s_2,\ldots$ satisfy
	\beql{eq-only-depend-prev}
		\cpr{s_n}{\xcap{s}{n-1}} = \cpr{s_n}{s_{n-1}}
	\eeql
	for any $n\leq1$,
	\ie,
	the outcome $s_n$ only depends on the previous outcome $s_{n+1}$.
	The sequential experiments that satisfy (\ref{eq-only-depend-prev})
	are called \emph{Markov chains}
	For these experiments, the probability of a sequence
	$\xcomma{s}{n}$\ is given by
	\beql{eq-markov-chains}
		\pr{\xcomma{s}{n}} = \cpr{s_{n}}{s_{n-1}} \cpr{s_{n-1}}{s_{n-2}}
			\cdots \cpr{s_{1}}{s_{0}} \pr{s_0}.
	\eeql
\eit


\section{Synthesizing Randomness: Random Number Generators \optional}
MAYBE LATER

\section{Fine Points: Event Classes \optional}
\label{sec:fine-points-1}
NO CONTENTS

\section{Fine Points: Probabilities of Sequences of Events Summary Problems \optional}
\label{sec:fine-points-2}
(ALMOST) NO CONTENTS

\subsection{The Borel Field of Events}
\subsection{Continuity of Probability}
\begin{coro}[Continuity of Probability Function]
\label{coro-cont-prob-fcn}
Let $A_1$, $A_2$, \ldots\
be an increasing or decreasing sequences of events
in \evcl, then:
\beql{eq-cont-prob-fcn}
	\lim_{n\to\infty} \pr{A_n} = \pr{\lim_{n\to\infty}A_n}.
\eeql

\end{coro}



