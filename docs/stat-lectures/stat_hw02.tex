

\input{D:/Multimedia/mydefs}
%\input{D:/mydefs}

\usepackage{fullpage}
\usepackage{fancyhdr}

\usepackage{graphicx}

\pagestyle{fancy}
\fancyhead[R]{Probability and Statistics for Electrical Engineering}

\addtolength{\headsep}{.5cm}



\newcommand{\mfor}{\mbox{ for }}
\newcommand{\mand}{\mbox{ and }}
\newcommand{\mforall}{\mbox{ for all }}
\newcommand{\mif}{\mbox{if }}
\newcommand{\ld}{\ensuremath{\lambda}}


\newcommand{\lgprob}[1]{LGProb~{#1}}
\newcommand{\lgfig}[1]{LGFig~{#1}}
\newcommand{\lgexam}[2]{LGExample~{#1} {\bf #2}}
\newcommand{\lgexamref}[1]{LGExample~{#1}}
\newcommand{\lgeq}[1]{LGEquation~(#1)}

\newcommand{\br}[1]{\{#1\}}

\newcommand{\randvar}{random variable}
\newcommand{\var}{variance}
\newcommand{\bernrv}{{Bernoulli \randvar}}
\newcommand{\binomrv}{{binomial \randvar}}
\newcommand{\geomrv}{{geometric \randvar}}
\newcommand{\possrv}{{Poisson \randvar}}

\newcommand{\unifrv}{{uniform \randvar}}
\newcommand{\exprv}{{exponential \randvar}}
\newcommand{\gaussrv}{{uniform \randvar}}

\newcommand{\possdist}{{Poisson distribution}}
\newcommand{\cond}{conditional}

\newcommand{\unitfcn}{unit step function}
\newcommand{\delfcn}{delta function}



\newcommand{\al}{\ensuremath{\alpha}}


\newcommand{\sA}{\ensuremath{A}}
\newcommand{\sB}{\ensuremath{B}}
\newcommand{\Ai}[1]{\ensuremath{A_{#1}}}
\newcommand{\Bi}[1]{\ensuremath{B_{#1}}}
\newcommand{\Ei}[1]{\ensuremath{E_{#1}}}
\newcommand{\Ak}{\Ai{k}}
\newcommand{\X}{\ensuremath{X}}
\newcommand{\xk}{\ensuremath{x_k}}
\newcommand{\mzeta}{\ensuremath{\zeta}}

\newcommand{\comp}[1]{\ensuremath{{#1}^c}}
\newcommand{\diff}[2]{{#1}\backslash{#2}}
\newcommand{\sspace}{\ensuremath{S}}
\newcommand{\ssx}{\ensuremath{S_X}}
\newcommand{\evcl}{\ensuremath{\mathcal{F}}}

\newcommand{\pr}[1]{\ensuremath{P[#1]}}
%\renewcommand{\pr}[1]{\ensuremath{{\bf Pr}\{#1\}}}
\newcommand{\prb}[1]{\ensuremath{P[\{#1\}]}}
\newcommand{\bpr}[1]{\ensuremath{P\left[#1\right]}}
\newcommand{\cpr}[2]{\pr{#1|#2}}
\newcommand{\cpreq}[2]{\frac{\pr{#1\cap #2}}{\pr{#2}}}
\newcommand{\cprbeq}[2]{\cpreq{\br{#1}}{\br{#2}}}
\newcommand{\appr}[2]{\pr{#1|#2}\pr{#2}}
\newcommand{\relfreq}[1]{f_{#1}}

\newcommand{\expect}[1]{\ensuremath{m_{#1}}}
\newcommand{\cexp}[2]{\expect{#1|#2}}
\renewcommand{\Expect}[1]{\ensuremath{{\bf E}[#1]}}
\newcommand{\bExp}[1]{\ensuremath{{\bf E}\left[#1\right]}}
\newcommand{\Exp}[1]{\Expect{#1}}
\newcommand{\cExp}[2]{\Exp{#1|#2}}
\newcommand{\samplemean}[2]{\langle{#1}\rangle_{#2}}
\newcommand{\std}[1]{\ensuremath{\sigma_{#1}}}
\newcommand{\VAR}[1]{\ensuremath{{\bf VAR}[#1]}}
\newcommand{\cVAR}[2]{\VAR{#1|#2}}
\newcommand{\STD}[1]{\ensuremath{{\bf STD}[#1]}}

\newcommand{\ep}[1]{\ensuremath{E_{#1}}}
\renewcommand{\ev}[1]{\ensuremath{A_{#1}}}

\newcommand{\borel}{\ensuremath{\mathcal{B}}}

\newcommand{\chs}[2]{\ensuremath{{{#1} \choose {#2}}}}

\newcommand{\bigcupto}[3]{\bigcup_{#1=#2}^{#3}}
\newcommand{\bigcupkton}{\bigcupto{k}{1}{n}}
\newcommand{\bigcupktoi}{\bigcupto{k}{1}{\infty}}

\newcommand{\bigcapto}[3]{\bigcap_{#1=#2}^{#3}}
\newcommand{\bigcapkton}{\bigcapto{k}{1}{n}}

\newcommand{\sumto}[3]{\sum_{#1=#2}^{#3}}
\newcommand{\sumkton}{\sumto{k}{1}{n}}
\newcommand{\sumiton}{\sumto{i}{1}{n}}
\newcommand{\sumktoi}{\sumto{k}{1}{\infty}}
\newcommand{\sumntoi}{\sumto{n}{1}{\infty}}
\newcommand{\sumkzton}{\sumto{k}{0}{n}}
\newcommand{\sumkztoi}{\sumto{k}{0}{\infty}}
\newcommand{\sumnztoi}{\sumto{n}{0}{\infty}}

\newcommand{\xcomma}[2]{\ensuremath{{#1}_1, {#1}_2, \ldots, {#1}_{#2} }}
\newcommand{\xcommai}[1]{\ensuremath{{#1}_1, {#1}_2, \ldots }}
\newcommand{\xplus}[2]{\ensuremath{{#1}_1 + {#1}_2 + \cdots + {#1}_{#2} }}
\newcommand{\xcap}[2]{\ensuremath{{#1}_1 \cap {#1}_2 \cap \cdots \cap {#1}_{#2} }}

\newcommand{\ceil}[1]{\ensuremath{\lceil #1 \rceil}}
\newcommand{\floor}[1]{\ensuremath{\lfloor #1 \rfloor}}
\newcommand{\kmax}{\ensuremath{k_\mathrm{max}}}

\newcommand{\posspmf}[2]{\frac{{#1}^{#2}}{{#2}!}e^{-{#1}}}
\newcommand{\binompmf}[3]{\chs{{#1}}{{#3}}{#2}^{#3}(1-{#2})^{{#1}-{#3}}}
\newcommand{\dbinompmf}{\binompmf{n}{p}{k}}


%\newcommand{\numbereveryeqns}{}

\ifdefined\numbereveryeqns
	\newcommand{\beq}{\begin{equation}}
	\newcommand{\eeq}{\end{equation}}
\else
	\newcommand{\beq}{\[}
	\newcommand{\eeq}{\]}
\fi

\newcommand{\beql}[1]{\begin{equation}\label{#1}}
\newcommand{\eeql}{\end{equation}}

\newcommand{\optional}{$\dagger$}

\newcommand{\pmf}[1]{p_{#1}}
\newcommand{\pmfk}[2]{\ensuremath{\pmf{#1}(#2)}}
\newcommand{\pmfxk}[1]{\pmfk{X}{#1}}
\newcommand{\cpmfk}[3]{\ensuremath{\pmf{#1}(#2|#3)}}
\newcommand{\cpmfxk}[2]{\cpmfk{X}{#1}{#2}}

\newcommand{\cdf}[1]{F_{#1}}
\newcommand{\cdfk}[2]{\ensuremath{\cdf{#1}(#2)}}
\newcommand{\cdfxk}[1]{\cdfk{X}{#1}}
\newcommand{\ccdfk}[3]{\ensuremath{\cdf{#1}(#2|#3)}}
\newcommand{\ccdfxk}[2]{\ccdfk{X}{#1}{#2}}

\newcommand{\pdf}[1]{f_{#1}}
\newcommand{\pdfk}[2]{\ensuremath{\pdf{#1}(#2)}}
\newcommand{\pdfxk}[1]{\pdfk{X}{#1}}
\newcommand{\cpdfk}[3]{\ensuremath{\pdf{#1}(#2|#3)}}
\newcommand{\cpdfxk}[2]{\cpdfk{X}{#1}{#2}}

\newcommand{\indi}[1]{I_{#1}}

\newcommand{\acapb}{\ensuremath{A\cap B}}
\newcommand{\acupb}{\ensuremath{A\cup B}}


\newcommand{\limninfty}{\lim_{n\to\infty}}
\newcommand{\intinftoinf}{\int_{-\infty}^\infty}
\newcommand{\intinfto}[1]{\int_{-\infty}^{#1}}

\newcommand{\ddx}{\frac{d}{dx}}

\newcommand{\gausspdf}[3]{\frac{1}{\sqrt{2\pi} {#3}} e^{-({#1}{#2})^2/2{#3}^2} }



\renewcommand{\emph}[1]{{\it #1}}

\begin{document}
\setlength{\headheight}{15pt}
\maketitle

\begin{enumerate}
	\item LG 2.127. In order for a circuit board to work,
	seven identical chips must be in working order.
	To improve reliability, an additional chip is included in the board,
	and the design allows it to replace any of the seven other chips
	when they fail.
	\begin{enumerate}
		\item Find the probability $p_b$ that the board is working
		in terms of the probability $p$ that an individual chip is working.
		\ifdefined\sol
		\begin{solution}
			\begin{enumerate}
				\item Method 1:
				The board can afford to have one chip failure at most
				due to the additional chip.
				The number of working chips is governed
				by the binomial probability law,
				hence
				\begin{eqnarray*}
					\lefteqn{
					p_b =
					\pr{\{\mbox{all chips are working}\}}
					}
					\\&&
					+ \pr{\{\mbox{only one chip is not working \emph{and} the additional chip is working}\}}
					\\&=& p^7 + {7 \choose 1} (1-p) p^6 \cdot p
					= p^7 ( 8 - 7p).
				\end{eqnarray*}
				since the event that ``only one chip is not working''
				and the event ``the additional chip is working''
				are independent.

				\item Method 2:
				Indeed, we can consider the $8$ chips ($7$ main chips plus $1$ additional chip)
				all together. The board will work if at least $7$ chips are working.
				Hence, the answer can be obtained by
				\[
					p_b = p^8 + {8 \choose 1} p^7 (1-p)
					= p^7(8-7p).
				\]
			\end{enumerate}
			(It is interesting to verify that $p_b$ is a nondecreasing increasing
			function in $p$ (since it should be!).
			To verify this, let $f(p) = 8p^7 - 7p^8$
			and differentiate it by $p$:
			\[
				f'(p) = 56p^6 - 56p^7 = 56p^6(1-p).
			\]
			Since $f'(p)>0$ for $0<p<1$, $p_b$ indeed is a monotonically increasing function in $p$,
			\ie, if the probability that an individual chip is working increases,
			the probability that the board is working also increase.)
		\end{solution}
		\fi

		\item Suppose that $n$ circuit boards are operated in parallel,
		and that we require a $99.9$\% probability that at least one
		board is working. How many boards are needed?

		\ifdefined\sol
		\begin{solution}
			In order for the probability that at least one board is working
			to be at least $99.9$\%,
			the probability that all board fail to work should be less than $0.1$\%.
			Hence,
			\[
				(1-p_b)^n < 10^{-3}
				\iff
				n \log_{10}(1-p_b) < -3
				\Rightarrow
				n \geq \lceil -3 / \log_{10}(1-p_b) \rceil
			\]
			where $\lceil x \rceil$ is defined by
			the smallest integer that is greater than or equal to $x$.
			For example,
			if $p=.9$, $n\geq 5$,
			if $p=.7$, $n\geq 24$,
			and if $p=.5$, $n\geq 194$.
			
		\end{solution}
		\fi

	\end{enumerate}

	\item LG 3.1. Let \X\ be the maximum of the number of heads
	obtained when Carlos and Michael each flip a fair coin twice.
	\begin{enumerate}
		\item Describe the underlying space \sspace\ of this random experiment
		and specify the probabilities of its elementary events.
		\ifdefined\sol
		\begin{solution}
		\begin{eqnarray*}
			\sspace &=& \{
			(HH,HH), (HH,HT), (HH,TH), (HH,TT),
			\\&&(HT,HH), (HT,HT), (HT,TH), (HT,TT),
			\\&&(TH,HH), (TH,HT), (TH,TH), (TH,TT),
			\\&&(TT,HH), (TT,HT), (TT,TH), (TT,TT)
			\}
		\end{eqnarray*}
		and the probability of every elementary event
		is $1/16$.
		\end{solution}
		\fi

		\item Show the mapping from \sspace\ to \ssx, the range of \X.
		\ifdefined\sol
		\begin{solution}
		Let $A_0, A_1, A_2 \subset \sspace$ such that
		\[
			A_0 = \{ (TT,TT) \},
		\]
		\[
			A_1 = \{
			(HT,HT), (HT,TH), (HT,TT),
			(TH,HT), (TH,TH), (TH,TT),
			(TT,HT), (TT,TH) \},
		\]
		\[
			A_2 = \{
			(HH,HH), (HH,HT), (HH,TH), (HH,TT),
			(HT,HH),
			(TH,HH),
			(TT,HH) \}.
		\]
		Obviously, $\ssx = \{0,1,2\}$
		and $X$ maps all the elements in $A_k$ to $k$
		for $k=0,1,2$.
		\end{solution}
		\fi

		\item Find the probabilities for the various values of \X.
		asdf
		\ifdefined\sol
		\begin{solution}
		\begin{enumerate}
			\item Method I: Since the event $A_k$ corresponds to $X=k$ for $k=0,1,2$,
			we can easily calculate each probability as follows:
			\begin{eqnarray*}
				\pr{X=0} &=& \pr{A_0} = 1/16,
				\\\pr{X=1} &=& \pr{A_1} = 8/16,
				\\\pr{X=2} &=& \pr{A_2} = 7/16.
			\end{eqnarray*}

			\item Method II:
			This method can seem a bit complicated, but it will pay off later.
			Let us introduce two new random varaibles, $Y$ and $Z$,
			each of each represents the number of heads of Carlos and Michael
			respectively.
			Then obviously,
			\[ \pr{Y=0} = 1/4,\ \pr{Y=1} = 1/2,\ \pr{Y=2} = 1/4 \]
			and
			\[ \pr{Z=0} = 1/4,\ \pr{Z=1} = 1/2,\ \pr{Z=2} = 1/4. \]
			Since $X = \max{Y,Z}$, and $Y$ and $Z$ are independent,
			we have
			\begin{eqnarray*}
				\pr{X=0} &=& \pr{Y=0 \mand Z=0}
				= \pr{Y=0}\pr{Z=0} = 1/16,
				\\ \pr{X=1}&=&
				\pr{Y\leq 1 \mand Z\leq 1} - \pr{Y=0 \mand Z=0}
				= (3/4)^2 - 1/16 =  8/16,
				\\ \pr{X=2}&=& 1 - \pr{X=0} - \pr{X=1}
				 = 7/16.
			\end{eqnarray*}

		\end{enumerate}
		\end{solution}
		\fi


	\end{enumerate}

	\item LG 3.6.
	An information source produces binary triplets
	$\{000, 111, 010, 101, 001, 110, 100, 011\}$
	with corresponding probabilities
	$\{1/4, 1/4, 1/8, 1/8, 1/16, 1/16, 1/16, 1/16\}$.
	A binary code assigns a codeword of length $-\log_2 p_k$
	to triplet $k$.
	Let \X\ be the length of the string assigned to the output of the information source.
	\begin{enumerate}
		\item Show the mapping from \sspace\ to \ssx,
		the range of \X.
		\ifdefined\sol
		\begin{solution}
			Since $-\log_2 (1/4) = 2$, $-\log_2 (1/8) = 3$, and $-\log_2 (1/16) = 4$,
			\[
				\ssx = \{2,3,4\}.
			\]
			Let
			\[ A_2 = \{ 000, 111 \}, \]
			\[ A_3 = \{ 010, 101 \}, \]
			\[ A_4 = \{ 001, 110, 100, 011 \}. \]
			Then $X$ maps every element in $A_k$ to $k$
			for $k=2,3,4$.
		\end{solution}
		\fi

		\item Find the probabilities for the various values of \X.
		\ifdefined\sol
		\begin{solution}
		\begin{eqnarray*}
			\pr{X=2} &=& \pr{A_2} = 1/4 + 1/4 = 1/2,
			\\\pr{X=3} &=& \pr{A_3} = 1/8 + 1/8 = 1/4,
			\\\pr{X=4} &=& \pr{A_4} = 1/16 + 1/16 + 1/16 + 1/16 = 1/4.
		\end{eqnarray*}
		\end{solution}
		\fi

	\end{enumerate}

	\item LG 3.10.
	An $m$-bit password
	is required to access a system.
	A hacker systematically works through all possible $m$-bit patterns.
	Let \X\ be the number of patterns
	tested until the correct password is found.
	(Assume that all the possible combination for the password
	is equiprobable.)
	\begin{enumerate}
		\item Describe the sample space of \sspace.
		\ifdefined\sol
		\begin{solution}
		Let \sspace\ be the set of all possible
		passwords. Then
		\[
			\sspace  =  \{0,1\}^m
			= \set{(\xcomma{x}{n})}{x_i\in\{0,1\}\mfor 1\leq i\leq m}.
		\]
		The size of the sample space is $2^m$.
		\end{solution}
		\fi

		\item Show the mapping from \sspace\ to \ssx, the range of \X.
		\ifdefined\sol
		\begin{solution}
			Without loss of generality,
			we assume that the hacker tries
			passwords in an increasing order
			when the $n$-tuple is interpreted as binary number,
			\ie, in the following order:
			\[
				(\ldots,0,0,0),
				(\ldots,0,0,1),
				(\ldots,0,1,0),
				(\ldots,0,1,1),
				(\ldots,1,0,0),
				(\ldots,1,0,1),
			\]
			\[
				(\ldots,1,1,0),
				(\ldots,1,1,1),
				\ldots,
				(1,\ldots,1,1,1),
				\]
			Then $\ssx = \{1,2,\ldots, 2^m\}$
			and \X\ maps each \[(\xcomma{x}{n})\]
			to \[2^{n-1}x_1 + 2^{n-2} x_2 + \cdots + 2 x_{n-1} + x_{n} + 1.\]

		\end{solution}
		\fi


		\item Find the probabilities for the various values of \X.
		\ifdefined\sol
		\begin{solution}
			Since all the possible passwords are equiprobable,
			\[
				\pr{X=k} = 1/2^m
				\mfor k =1,2,\ldots,2^m.
			\]

		\end{solution}
		\fi

	\end{enumerate}

	\item LG 3.11.
	Let \X\ be the maximum of the coin tosses in Problem 3.1.
	\begin{enumerate}
		\item Compare the PMF of \X\ with the PMF of $Y$,
		the number of heads in two tosses of a fair coin.
		Explain the difference.
		\ifdefined\sol
		\begin{solution}
			In Problem 3.1, 
			the PMF of \X\ was obtained as
			\[
				\pmf{X}(0) = 1/16,\
				\pmf{X}(1) = 8/16,\
				\pmf{X}(2) = 7/16,
			\]
			and that of $Y$ is
			\[
				\pmf{Y}(0) = 1/4,\
				\pmf{Y}(1) = 1/2,\
				\pmf{Y}(2) = 1/4.
			\]
			The ranges of the two random variables
			are the same; $\{0,1,2\}$.
			However, the probability assignment is different
			because \emph{the underlying sample spaces are different}.
	
		\end{solution}
		\fi

		\item Suppose that Carlos uses a coin with probability of heads
		$p = 3/4$. Find the PMF of \X.
		\ifdefined\sol
		\begin{solution}
		\begin{enumerate}
			\item Method I:
			Let \sspace\ be the sample space defined in 3.1
			and $A_0$, $A_1$, and $A_2$ be the events defined in the same problem.
			Then
			\begin{eqnarray*}
				\pmf{X}(0) &=& \pr{A_0} = \frac{1}{16} \cdot \frac{1}{4}
					= \frac{1}{64},
				\\\pmf{X}(1) &=& \pr{A_1}
				= 6 \cdot \frac{3}{16} \cdot \frac{1}{4}
				+ 2 \cdot \frac{1}{16} \cdot \frac{1}{4}
				= \frac{20}{64},
				\\\pmf{X}(2) &=& \pr{A_2}
				= 4 \cdot \frac{9}{16} \cdot \frac{1}{4}
				+ 2 \cdot \frac{3}{16} \cdot \frac{1}{4}
				+ \frac{1}{16} \cdot \frac{1}{4}
				= \frac{43}{64}.
			\end{eqnarray*}

			\item Method II:
			Define $Y$ and $Z$ as in Method II of Problem 3.1.
			The PMF of $Z$ remaines the same,
			but that of $Y$ is changed as follows:
			\[
				\pr{Y=0} = 1/16,\ \pr{Y=1} = 6/16,\ \pr{Y=2} = 9/16.
			\]
			Thus
			\begin{eqnarray*}
				\pr{X=0} &=& \pr{Y=0 \mand Z=0}
				= \pr{Y=0}\pr{Z=0} = \frac{1}{16} \cdot \frac{1}{4} = \frac{1}{64},
				\\ \pr{X=1}&=&
				\pr{Y\leq 1} \pr{Z\leq 1} - \pr{Y=0} \pr{ Z=0}
				= \frac{7}{16}\cdot \frac{3}{4} - \frac{1}{64} =  \frac{20}{64},
				\\ \pr{X=2}&=& 1 - \pr{X=0} - \pr{X=1}
				 = \frac{43}{64}.
			\end{eqnarray*}


		\end{enumerate}
		\end{solution}
		\fi

	\end{enumerate}

	\item LG 3.12. Consider an information source that produces binary pairs
	that we designate as $\ssx = \{1, 2, 3, 4\}$.
	Find and plot the PMF in the following cases.
	You're free to use the following two facts without proof:
	\begin{itemize}
		\item An infinite series $\sumktoi 1/k^p$
		converges if and only if $p > 1$.

		\item If two infinite sequences
		$\{a_k\}_{k=1}^\infty$
		and $\{b_k\}_{k=1}^\infty$
		satisfy $0\leq a_k\leq b_k$ for all $k\in\{1,2,\ldots\}$
		and $\sumktoi b_k$ converges,
		then $\sumktoi a_k$ converges.
	\end{itemize}
	\begin{enumerate}
		\item $p_k = p_1/k$ for all $k$ in \ssx.
		\ifdefined\sol
		\begin{solution}
		Since the sum of all the PMF values must be $1$,
		we have
		\[
			\sumto{k}{1}{4} \pmf{X}(k)
			= \sumto{k}{1}{4} p_k
			= (1 + 1/2 + 1/3 + 1/4) p_1
			= \frac{25}{12} p_1 = 1.
		\]
		Thus, $p_1 = 12/25$ and
		\[
			\pmf{X}(k) = \frac{12}{25k} \mfor k=1,2,3,4.
		\]
		\end{solution}
		\fi

		\item $p_{k+1} = p_k/2$ for $k = 1, 2, 3$.
		\ifdefined\sol
		\begin{solution}
		The recursive formula implies
		\[
			p_2 = p_1 /2,\
			p_3 = p_1 /4,\
			p_4 = p_1 /8,
		\]
		thus,
		\[
			\sumto{k}{1}{4} \pmfxk{k}
			= \sumto{k}{1}{4} p_1 / 2^{k-1}
			= (1+1/2+1/4+1/8) p_1
			= \frac{15}{8} p_1 = 1
		\]
		and we have $p_1 = 8/15$.
		Therefore
		\[
			\pmfxk{k} = \frac{16}{15\cdot 2^k}
			= \frac{1}{15} 2^{-k+4}
			\mfor k=1,2,3,4.
		\]
		\end{solution}
		\fi

		\item $p_{k+1} = p_k/2^k$ for $k = 1, 2, 3$.
		\ifdefined\sol
		\begin{solution}
		The recursive equation implies
		\[
			p_2 = p_1 /2,\
			p_3 = p_2 /4 = p_1/8,\
			p_4 = p_3 /8 = p_1/64,
		\]
		thus,
		\[
			\sumto{k}{1}{4} \pmfxk{k}
			= (1+1/2+1/8+1/64) p_1
			= \frac{105}{64} p_1 = 1
		\]
		and we have $p_1 = 64/105$.
		Therefore
		\[
			\pmfxk{k} = \frac{64}{105} 2^{-k(k-1)/2}
			\mfor k=1,2,3,4.
		\]
		\end{solution}
		\fi


		\item Can the random variables in parts (a), (b), and (c)
		be extended to take on values in the set $\{1, 2, \ldots\}$?
		If yes, specify the PMF of the resulting random variables.
		If no, explain why not.
		\ifdefined\sol
		\begin{solution}
			\begin{enumerate}
			\item
				For (a), we \emph{cannot}
				extend the random variable so as to take
				on values $\{1,2,\ldots\}$
				for if $p_1>0$,
				\[
					\sumktoi \pmfxk{k} = p_1 \sumktoi 1/k = \infty
				\]
				and if $p_1=0$, $\pmfxk{k} = 0 $ for all $k$.

			\item
				For (b), we \emph{can}
				extend the random variable so as to take
				on values $\{1,2,\ldots\}$
				for the recursion gives $p_k  = p_1/2^{k-1}$
				and
				\[
					\sumktoi \pmfxk{k}
					= \sumktoi p_1/ 2^{k-1}
					= \frac{p_1}{1-1/2} = 2p_1 = 1.
				\]
				Therefore the PMF
				\[
					\pmfxk{k} = 1/ 2^k \mfor k =1,2,\ldots
				\]
				defines the random variable.

			\item
				For (c), we \emph{can}
				extend the random variable so as to take
				on values $\{1,2,\ldots\}$.
				If we let $q_k = \log_2 p_k$,
				the recursion becomes
				\[
					q_{k+1} = q_k - k,
				\]
				hence
				\begin{eqnarray*}
					\lefteqn{
					q_k = q_{k-1} - (k-1)
					= q_{k-2} - (k-2) - (k-1) 
					= \cdots
					}
					\\&=& q_{1} - (1+2+\cdots+(k-1))
					= q_{1} - k(k-1)/2.
				\end{eqnarray*}
				Thus,
				\[
					p_k = 2^{q_k} = p_1 2^{-k(k-1)/2}.
				\]
				Now we need to check whether there exists $p_1>0$
				such that the following equation holds:
				\begin{equation}
				\label{eq-1}
					\sumktoi p_k = p_1 \sumktoi 2^{-k(k-1)/2} = 1.
				\end{equation}
				Since $ 2^{-k(k-1)/2} \leq 2^{-(k-1)}$
				for all $k\in\{1,2,\ldots\}$
				and $\sumktoi 2^{-(k-1)}$ converges,
				the infinite series (\ref{eq-1}) converges
				(though we (or at least I) cannot evaulate the exact value).
				If we let $p = 1/\sumktoi 2^{-k(k-1)/2}$,
				the PMF
				\[
					\pmfxk{k} = p \cdot 2^{-k(k-1)/2}
				\]
				defined the random variable.
			\end{enumerate}
		\end{solution}
		\fi

	\end{enumerate}

	\item LG 3.17.
	A modem transmits a $+2$
	voltage signal into a channel.
	The channel adds to this signal a noise term that
	is drawn from the set $\{0, -1, -2, -3\}$
	with respective probabilities $\{4/10, 3/10, 2/10, 1/10\}$.
	\begin{enumerate}
		\item Find the PMF of the output $Y$ of the channel.
		\ifdefined\sol
		\begin{solution}
		\[
			\pmf{Y}(2) = 4/10,\
			\pmf{Y}(1) = 3/10,\
			\pmf{Y}(0) = 2/10,\
			\pmf{Y}(-1) = 1/10.
			\]
		\end{solution}
		\fi

		\item What is the probability that the output of the channel
		is equal to the input of the channel?
		\ifdefined\sol
		\begin{solution}
			In order for the output of the channel to be equal to the input of the channel,
			the noise must be zero,
			hence the probability is $4/10$.
		\end{solution}
		\fi

		\item What is the probability that the output of the channel is positive?
		\ifdefined\sol
		\begin{solution}
			The probability is $0.7$ for
			\[
				\sum_{k>0} \pmf{Y}(k)
				= \pmf{Y}(1) + \pmf{Y}(2)
				= 3/10 + 4/10 = 7/10.
			\]
		\end{solution}
		\fi

	\end{enumerate}


\end{enumerate}

\end{document}



