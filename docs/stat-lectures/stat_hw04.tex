

\input{D:/Multimedia/mydefs}
%\input{D:/mydefs}

\usepackage{fullpage}
\usepackage{fancyhdr}

\usepackage{graphicx}

\pagestyle{fancy}
\fancyhead[R]{Probability and Statistics for Electrical Engineering}

\addtolength{\headsep}{.5cm}



\newcommand{\mfor}{\mbox{ for }}
\newcommand{\mand}{\mbox{ and }}
\newcommand{\mforall}{\mbox{ for all }}
\newcommand{\mif}{\mbox{if }}
\newcommand{\ld}{\ensuremath{\lambda}}


\newcommand{\lgprob}[1]{LGProb~{#1}}
\newcommand{\lgfig}[1]{LGFig~{#1}}
\newcommand{\lgexam}[2]{LGExample~{#1} {\bf #2}}
\newcommand{\lgexamref}[1]{LGExample~{#1}}
\newcommand{\lgeq}[1]{LGEquation~(#1)}

\newcommand{\br}[1]{\{#1\}}

\newcommand{\randvar}{random variable}
\newcommand{\var}{variance}
\newcommand{\bernrv}{{Bernoulli \randvar}}
\newcommand{\binomrv}{{binomial \randvar}}
\newcommand{\geomrv}{{geometric \randvar}}
\newcommand{\possrv}{{Poisson \randvar}}

\newcommand{\unifrv}{{uniform \randvar}}
\newcommand{\exprv}{{exponential \randvar}}
\newcommand{\gaussrv}{{uniform \randvar}}

\newcommand{\possdist}{{Poisson distribution}}
\newcommand{\cond}{conditional}

\newcommand{\unitfcn}{unit step function}
\newcommand{\delfcn}{delta function}



\newcommand{\al}{\ensuremath{\alpha}}


\newcommand{\sA}{\ensuremath{A}}
\newcommand{\sB}{\ensuremath{B}}
\newcommand{\Ai}[1]{\ensuremath{A_{#1}}}
\newcommand{\Bi}[1]{\ensuremath{B_{#1}}}
\newcommand{\Ei}[1]{\ensuremath{E_{#1}}}
\newcommand{\Ak}{\Ai{k}}
\newcommand{\X}{\ensuremath{X}}
\newcommand{\xk}{\ensuremath{x_k}}
\newcommand{\mzeta}{\ensuremath{\zeta}}

\newcommand{\comp}[1]{\ensuremath{{#1}^c}}
\newcommand{\diff}[2]{{#1}\backslash{#2}}
\newcommand{\sspace}{\ensuremath{S}}
\newcommand{\ssx}{\ensuremath{S_X}}
\newcommand{\evcl}{\ensuremath{\mathcal{F}}}

\newcommand{\pr}[1]{\ensuremath{P[#1]}}
%\renewcommand{\pr}[1]{\ensuremath{{\bf Pr}\{#1\}}}
\newcommand{\prb}[1]{\ensuremath{P[\{#1\}]}}
\newcommand{\bpr}[1]{\ensuremath{P\left[#1\right]}}
\newcommand{\cpr}[2]{\pr{#1|#2}}
\newcommand{\cpreq}[2]{\frac{\pr{#1\cap #2}}{\pr{#2}}}
\newcommand{\cprbeq}[2]{\cpreq{\br{#1}}{\br{#2}}}
\newcommand{\appr}[2]{\pr{#1|#2}\pr{#2}}
\newcommand{\relfreq}[1]{f_{#1}}

\newcommand{\expect}[1]{\ensuremath{m_{#1}}}
\newcommand{\cexp}[2]{\expect{#1|#2}}
\renewcommand{\Expect}[1]{\ensuremath{{\bf E}[#1]}}
\newcommand{\bExp}[1]{\ensuremath{{\bf E}\left[#1\right]}}
\newcommand{\Exp}[1]{\Expect{#1}}
\newcommand{\cExp}[2]{\Exp{#1|#2}}
\newcommand{\samplemean}[2]{\langle{#1}\rangle_{#2}}
\newcommand{\std}[1]{\ensuremath{\sigma_{#1}}}
\newcommand{\VAR}[1]{\ensuremath{{\bf VAR}[#1]}}
\newcommand{\cVAR}[2]{\VAR{#1|#2}}
\newcommand{\STD}[1]{\ensuremath{{\bf STD}[#1]}}

\newcommand{\ep}[1]{\ensuremath{E_{#1}}}
\renewcommand{\ev}[1]{\ensuremath{A_{#1}}}

\newcommand{\borel}{\ensuremath{\mathcal{B}}}

\newcommand{\chs}[2]{\ensuremath{{{#1} \choose {#2}}}}

\newcommand{\bigcupto}[3]{\bigcup_{#1=#2}^{#3}}
\newcommand{\bigcupkton}{\bigcupto{k}{1}{n}}
\newcommand{\bigcupktoi}{\bigcupto{k}{1}{\infty}}

\newcommand{\bigcapto}[3]{\bigcap_{#1=#2}^{#3}}
\newcommand{\bigcapkton}{\bigcapto{k}{1}{n}}

\newcommand{\sumto}[3]{\sum_{#1=#2}^{#3}}
\newcommand{\sumkton}{\sumto{k}{1}{n}}
\newcommand{\sumiton}{\sumto{i}{1}{n}}
\newcommand{\sumktoi}{\sumto{k}{1}{\infty}}
\newcommand{\sumntoi}{\sumto{n}{1}{\infty}}
\newcommand{\sumkzton}{\sumto{k}{0}{n}}
\newcommand{\sumkztoi}{\sumto{k}{0}{\infty}}
\newcommand{\sumnztoi}{\sumto{n}{0}{\infty}}

\newcommand{\xcomma}[2]{\ensuremath{{#1}_1, {#1}_2, \ldots, {#1}_{#2} }}
\newcommand{\xcommai}[1]{\ensuremath{{#1}_1, {#1}_2, \ldots }}
\newcommand{\xplus}[2]{\ensuremath{{#1}_1 + {#1}_2 + \cdots + {#1}_{#2} }}
\newcommand{\xcap}[2]{\ensuremath{{#1}_1 \cap {#1}_2 \cap \cdots \cap {#1}_{#2} }}

\newcommand{\ceil}[1]{\ensuremath{\lceil #1 \rceil}}
\newcommand{\floor}[1]{\ensuremath{\lfloor #1 \rfloor}}
\newcommand{\kmax}{\ensuremath{k_\mathrm{max}}}

\newcommand{\posspmf}[2]{\frac{{#1}^{#2}}{{#2}!}e^{-{#1}}}
\newcommand{\binompmf}[3]{\chs{{#1}}{{#3}}{#2}^{#3}(1-{#2})^{{#1}-{#3}}}
\newcommand{\dbinompmf}{\binompmf{n}{p}{k}}


%\newcommand{\numbereveryeqns}{}

\ifdefined\numbereveryeqns
	\newcommand{\beq}{\begin{equation}}
	\newcommand{\eeq}{\end{equation}}
\else
	\newcommand{\beq}{\[}
	\newcommand{\eeq}{\]}
\fi

\newcommand{\beql}[1]{\begin{equation}\label{#1}}
\newcommand{\eeql}{\end{equation}}

\newcommand{\optional}{$\dagger$}

\newcommand{\pmf}[1]{p_{#1}}
\newcommand{\pmfk}[2]{\ensuremath{\pmf{#1}(#2)}}
\newcommand{\pmfxk}[1]{\pmfk{X}{#1}}
\newcommand{\cpmfk}[3]{\ensuremath{\pmf{#1}(#2|#3)}}
\newcommand{\cpmfxk}[2]{\cpmfk{X}{#1}{#2}}

\newcommand{\cdf}[1]{F_{#1}}
\newcommand{\cdfk}[2]{\ensuremath{\cdf{#1}(#2)}}
\newcommand{\cdfxk}[1]{\cdfk{X}{#1}}
\newcommand{\ccdfk}[3]{\ensuremath{\cdf{#1}(#2|#3)}}
\newcommand{\ccdfxk}[2]{\ccdfk{X}{#1}{#2}}

\newcommand{\pdf}[1]{f_{#1}}
\newcommand{\pdfk}[2]{\ensuremath{\pdf{#1}(#2)}}
\newcommand{\pdfxk}[1]{\pdfk{X}{#1}}
\newcommand{\cpdfk}[3]{\ensuremath{\pdf{#1}(#2|#3)}}
\newcommand{\cpdfxk}[2]{\cpdfk{X}{#1}{#2}}

\newcommand{\indi}[1]{I_{#1}}

\newcommand{\acapb}{\ensuremath{A\cap B}}
\newcommand{\acupb}{\ensuremath{A\cup B}}


\newcommand{\limninfty}{\lim_{n\to\infty}}
\newcommand{\intinftoinf}{\int_{-\infty}^\infty}
\newcommand{\intinfto}[1]{\int_{-\infty}^{#1}}

\newcommand{\ddx}{\frac{d}{dx}}

\newcommand{\gausspdf}[3]{\frac{1}{\sqrt{2\pi} {#3}} e^{-({#1}{#2})^2/2{#3}^2} }



\renewcommand{\emph}[1]{{\it #1}}

\begin{document}
\setlength{\headheight}{15pt}
\maketitle

\begin{enumerate}

	\item \lgprob{3.35}.
	\begin{enumerate}
		\item In \lgprob{3.11} (a),
		find the conditional PMF of \X,
		the maximum of coin tosses, given that $\X > 0$.
		\ifdefined\sol
		\begin{solution}
		Recall from \lgprob{3.11} (a) that the PMF of \X\ is defined by
		\[
			\pmfxk{0} = 1/16,\
			\pmfxk{1} = 8/16,\
			\pmfxk{2} = 7/16.
		\]
		Thus
		\begin{eqnarray*}
			\cpmfxk{1}{X>0} &=& \cpr{X=1}{X>0}
			= \frac{\pr{\{X=1\}\cup\{X>0\}}}{\pr{X>0}}
			\\&=& \frac{\pr{X=1}}{\pr{X>0}}
			= \frac{8/16}{15/16} = 8/15.
		\end{eqnarray*}
		Likewise,
		\begin{eqnarray*}
			\cpmfxk{2}{X>0} &=& \cpr{X=2}{X>0}
			= \frac{\pr{\{X=2\}\cup\{X>0\}}}{\pr{X>0}}
			\\&=& \frac{\pr{X=2}}{\pr{X>0}}
			= \frac{7/16}{15/16} = 7/15.
		\end{eqnarray*}
		\end{solution}
		\fi

		\item Find the conditional PMF of \X\ given that Michael got one head in two tosses.
		\ifdefined\sol
		\begin{solution}
			Let $A$ be the event that Michael got one head in two tosses
			and let $Y$ be the number of heads of Carlos.
			Suppose that the Michael got one head in two coin tosses.
			Since the maximum of the numbers of heads of Carlos and Michael
			cannot be less than 1,
			\[
				\cpr{X=0}{A}= 0.
			\]
			Also since the maximum is $1$ the number of heads of Carlos
			is either $0$ or $1$,
			we have
			\[
				\cpr{X=1}{A} = \pr{Y=0} + \pr{Y=1} = 3/4.
			\]
			The maximum is $2$ if and only if the number of heads of Carlos is $2$,
			\[
				\cpr{X=2}{A} = \pr{Y=2} = 1/4.
			\]
			Therefore
			\[
				\cpmfxk{k}{A} = \left\{\begin{array}{ll}
					3/4	&\mbox{if }k=1,
					\\1/4	&\mbox{if }k=2.
				\end{array} \right.
			\]

		\end{solution}
		\fi

		\item Find the conditional PMF of \X\ given
		that Michael got one head in the first toss.
		\ifdefined\sol
		\begin{solution}
			Let $B$ be the event that Michael got one head in the first toss.
			Let $Y$ be the number of heads Carlos got
			and $Z$ be the number of heads for the second toss of Michael,
			\ie, $Z=0$ if the second toss of Michael is tail
			and $Z=1$ otherwise.
			Again, since Michael already got one head in the first toss,
			the maximum number cannot be less than $1$,
			hence
			\[
				\cpr{X=0}{B} = 0.
			\]
			Now suppose that $B$ happens.
			In order for $X$ to be $1$,
			$Z$ must be $0$
			and $Y$ must be either $0$ or $1$.
			Therefore
			\[
				\cpr{X=1}{B} = \pr{\{Y \leq 1\}\cap\{Z=0\}}
				= \pr{Y \leq 1} \pr{Z=0 }
				= 3/4 \times  1/2 = 3/8
			\]
			and
			\[
				\cpr{X=2}{B} = 1 - \cpr{X=0}{B} - \cpr{X=1}{B} = 5 /8.
			\]
			Therefore
			\[
				\cpmfxk{k}{B} = \left\{\begin{array}{ll}
					3/8	&\mbox{if }k=1,
					\\5/8	&\mbox{if }k=2.
				\end{array} \right.
			\]

		\end{solution}
		\fi

		\item In \lgprob{3.11} (b),
		find the probability that Carlos got the maximum given that $X = 2$.
		\ifdefined\sol
		\begin{solution}
			Recall from \lgprob{3.11} (a) that the PMF of \X\ is defined by
			\[
				\pmfxk{0} = 1/64,\
				\pmfxk{1} = 20/64,\
				\pmfxk{2} = 43/64.
			\]
			Now let $Y$ be the number of heads of Carlos.
			Then
			\[
				\pr{\{Y=2\}\cap\{X=2\}}
				= \pr{Y=2} = 9/16.
			\]
			Thus
			the (conditional) probability
			that Carlos got the maximum given that $X = 2$
			is
			\[
				\cpr{Y=2}{X=2} = \frac{\pr{\{Y=2\}\cap\{X=2\}}}{\pr{X=2}}
				= \frac{9}{16} \cdot \frac{64}{43}
				= \frac{36}{43}.
			\]
		\end{solution}
		\fi
	\end{enumerate}


	\item \lgprob{3.36}.
	Find the conditional PMF for the quaternary information source in \lgprob{3.12},
	parts (a), (b), and (c) given that $X < 4$.
	\ifdefined\sol
	\begin{solution}
	\begin{enumerate}
		\item Recall from \lgprob{3.12} (a) that
		\[
			\pmfxk{k} = \frac{12}{25k}
		\]
		for $k=1,2,3,4$.
		Thus
		\[
			\pr{X<4} = 1- \pr{X=4} = 1-\frac{3}{25} = \frac{22}{25}.
		\]
		Therefore
		\[
			\cpmfxk{k}{X<4} = \frac{\pr{\{X=k\}\cap \{X<4\}}}{\pr{X<4}}
			= \frac{\pr{X=k}}{\pr{X<4}}
			= \frac{12}{25k} \cdot \frac{25}{22} = \frac{6}{11k}.
		\]
		for $k=1,2,3$.

		\item Recall from \lgprob{3.12} (b) that
		\[
			\pmfxk{k} = \frac{16}{15\cdot 2^k}
		\]
		for $k=1,2,3,4$.
		Thus
		\[
			\pr{X<4} = 1- \pr{X=4} = 1-\frac{1}{15} = \frac{14}{15}.
		\]
		Therefore
		\[
			\cpmfxk{k}{X<4} = \frac{\pr{\{X=k\}\cap \{X<4\}}}{\pr{X<4}}
			= \frac{\pr{X=k}}{\pr{X<4}}
			= \frac{16}{15\cdot 2^k} \cdot \frac{15}{14} = \frac{8}{7\cdot 2^k}.
		\]
		for $k=1,2,3$.

		\item Recall from \lgprob{3.12} (c) that
		\[
			\pmfxk{1} = \frac{64}{105},\
			\pmfxk{2} = \frac{32}{105},\
			\pmfxk{3} = \frac{8}{105},\
			\pmfxk{4} = \frac{1}{105}.
		\]
		Thus
		\[
			\pr{X<4} = 1- \pr{X=4} = \frac{104}{105}.
		\]
		Therefore
		\begin{eqnarray*}
			\cpmfxk{1}{X<4} &=& \frac{64}{105} \cdot \frac{105}{104} = \frac{8}{13},
			\\\cpmfxk{2}{X<4} &=& \frac{32}{105} \cdot \frac{105}{104} = \frac{4}{13},
			\\\cpmfxk{3}{X<4} &=& \frac{8}{105} \cdot \frac{105}{104} = \frac{1}{13}.
		\end{eqnarray*}

	\end{enumerate}
	\end{solution}
	\fi


	\item \lgprob{3.40}.
	Explain why \lgeq{3.31b} can be used to find \Exp{X^2},
	but it cannot be used to directly find \VAR{X}.
	\ifdefined\sol
	\begin{solution}
		Note that \lgeq{3.31b} is
		\begin{equation}
		\label{eq-3.31b}
			\Exp{g(X)} = \sumiton \cExp{g(X)}{B_i} \pr{B_i}
		\end{equation}
		for some partition $B_1, B_2, \ldots, B_n$.
		If we let $g(x) = x^2$, then (\ref{eq-3.31b})
		becomes
		\[
			\Exp{X^2} = \sumiton \cExp{X^2}{B_i} \pr{B_i},
		\]
		hence it can be used to find \Exp{X^2}.
		However,
		in general
		\begin{eqnarray*}
			\lefteqn{
			\VAR{X} = \Exp{X^2} - \Exp{X}^2
			= \sumiton \cExp{X^2}{B_i} \pr{B_i}
			- \left(\sumiton \cExp{X}{B_i} \pr{B_i}  \right)^2
			}
			\\&=&
			\sumiton (\cExp{X^2}{B_i} \pr{B_i} - \cExp{X}{B_i}^2 \pr{B_i}^2)
			- 2 \sum_{1\leq i< j\leq n} \cExp{X}{B_i} \cExp{X}{B_j} \pr{B_i} \pr{B_j}
			\\&\neq&
			\sumiton \VAR{X} \pr{B_i}.
		\end{eqnarray*}
	\end{solution}
	\fi

	\item \lgprob{3.50}.
	Let \X\ be the binomial random variable.
	\begin{enumerate}
		\item Show that
		\[
			\frac{\pmfxk{k+1}}{\pmfxk{k}}
			= \frac{n-k}{k+1} \cdot \frac{p}{1-p}
		\]
		where $\pmfxk{0} = (1-p)^n$.
		\ifdefined\sol
		\begin{solution}
			Note that
			\[
				\pmfxk{k} = \chs{n}{k} p^k q^{n-k},
			\]
			thus
			\begin{eqnarray*}
				\frac{\pmfxk{k+1}}{\pmfxk{k}}
				&=& \frac{n!}{(k+1)!(n-k-1)!} p^{k+1} q^{n-k-1}
				\frac{k!(n-k)!}{n!} p^{-k} q^{-n+k}
				\\&=&
				\frac{n-k}{k+1} p q^{-1}
				= \frac{n-k}{k+1} \cdot \frac{p}{1-p}.
			\end{eqnarray*}
			Note that if we use this recursion,
			we can calculate the PMF of the \binomrv\ in $\mathcal{O}(n)$.
		\end{solution}
		\fi

		\item Show that part (a) implies that:
		\begin{enumerate}
			\item \pr{X = k}\ is maximum at $\kmax = \floor{(n + 1)p}$,
			where \floor{x}\ denotes the largest integer that is smaller than or equal to $x$;
			and
			\ifdefined\sol
			\begin{solution}
			Let $f:\{0,\ldots,n-1\} \to \reals$ be a function defined by
			$f(k) = {\pmfxk{k+1}}/{\pmfxk{k}}$.
			Then
			\begin{eqnarray*}
				\lefteqn{
				f(k) - 1
				= \frac{n-k}{k+1} \frac{p}{1-p} - 1
				= \frac{(n-k) p - (k+1) (1-p)}{(k+1)(1-p)}
				}
				\\&=&
				 \frac{(-p-1+p)k + np-1+p}{(k+1)(1-p)}
				= \frac{-k + (n+1)p - 1}{(k+1)(1-p)}.
			\end{eqnarray*}
			Therefore
			\begin{equation}
			\label{eq-var}
				f(k) =
				\frac{\pmfxk{k+1}}{\pmfxk{k}}
				\left\{ \begin{array}{ll}
				> 1	&\mbox{if } k < (n+1)p - 1,
				\\ = 1	&\mbox{if } k = (n+1)p - 1,
				\\ < 1	&\mbox{if } k > (n+1)p - 1.
				\end{array} \right.
			\end{equation}
			The definition of $\floor{x}$ implies that
			\[(n+1)p - 1 < \kmax \leq (n+1)p.\]
			If $k \leq \kmax - 2$,
			then $k < \kmax - 1 \leq (n+1)p - 1$.
			Thus $f(k) > 1$
			and
			\begin{equation}
			\label{eq-inc}
				\pmfxk{0} < \pmfxk{1} < \cdots < \pmfxk{\kmax-2} < \pmfxk{\kmax-1}.
			\end{equation}
			Also since $\kmax - 1 \leq (n+1)p-1$, $f(\kmax-1) \geq 1$,
			thus
			\begin{equation}
			\label{eq-eq}
				\pmfxk{\kmax-1} \leq \pmfxk{\kmax}.
			\end{equation}
			Lastly, if $k \geq \kmax$,
			then $k \geq \kmax > (n+1)p-1$, hence $ f(k) < 1$.
			Therefore we have
			\begin{equation}
			\label{eq-dec}
				\pmfxk{\kmax} > \pmfxk{\kmax+1} > \cdots > \pmfxk{n}.
			\end{equation}
			In summary, (\ref{eq-inc}), (\ref{eq-eq}), and (\ref{eq-dec}) imply
			\begin{equation}
			\label{eq-order}
				\pmfxk{0} < \cdots
				< \pmfxk{\kmax-1} \leq \pmfxk{\kmax}
				> \pmfxk{\kmax+1} > \cdots > \pmfxk{n}.
			\end{equation}
			Therefore \pmfxk{k}\ achieves its maximum at $k=\kmax$.
			\end{solution}
			\fi

			\item when $(n + 1)p$ is an integer,
			then the maximum is achieved at $\kmax$ and $\kmax-1$.
			\ifdefined\sol
			\begin{solution}
				If $(n+1)p$ is an integer,
				then
				$\kmax-1 = \floor{(n+1)p} -1 = (n+1)p -1 $
				and (\ref{eq-var}) implies $f(\kmax-1)=1$,
				\ie,
				\[
					\pmfxk{\kmax-1} = \pmfxk{\kmax}.
				\]
				Then (\ref{eq-order}) becomes
			\begin{equation}
			\label{eq-order-1}
				\pmfxk{0} < \cdots
				< \pmfxk{\kmax-1} = \pmfxk{\kmax}
				> \pmfxk{\kmax+1} > \cdots > \pmfxk{n},
			\end{equation}
			thus the maximum is achieved at both $\kmax-1$ and $\kmax$.
			\end{solution}
			\fi
		\end{enumerate}
	\end{enumerate}

	\item \lgprob{3.54}.
	Let $M$ be a geometric random variable.
	Show that $M$ satisfies the memoryless property:
	$\cpr{M \geq k + j}{M \geq j + 1} = \pr{M \geq k}$ for all $j$, $k > 1$.
	\ifdefined\sol
	\begin{solution}
		First note that
		\[
			\pr{M\geq n} = \sumto{k}{n}{\infty} q^{k-1}p
			= q^{n-1}p \frac{1}{1-q} = q^{n-1}.
		\]
		Thus
		\begin{eqnarray*}
			\lefteqn{
			\cpr{M \geq k + j}{M \geq j + 1}
			= \cprbeq{M \geq k + j}{M \geq j + 1}
			}
			\\&=&
			\frac{\pr{M \geq k+j}}{\pr{M\geq j+1}}
			= \frac{q^{k+j-1}}{q^{j}}
			= q^{k-1}
			= \pr{M\geq k}.
		\end{eqnarray*}
	\end{solution}
	\fi


	\item \lgprob{3.57}.
	A Christmas fruitcake has Poisson-distributed independent
	numbers of sultana raisins,
	iridescent red cherry bits,
	and radioactive green cherry bits with respective averages
	$48$, $24$, and $12$ bits per cake
	respectively.
	Suppose you politely accept a slice of $1/12$ of the cake.
	\begin{enumerate}
		\item What is the probability that you get lucky and get no green bits in your slice?
		\ifdefined\sol
		\begin{solution}
			Let $X$ be the number of green cherry bits in $1/12$ of the cake.
			Then since the average number of green cherry bits in $1/12$ of the cake is $12/12=1$,
			$X$ follows the \possdist\ with $\al=1$.
			Therefore the probability that you get no green bits in your slice is
			\[
				\pr{X=0}
				= \pmfxk{0} = \frac{(1)^0}{0!}e^{-1} = 0.3679.
			\]
		\end{solution}
		\fi

		\item What is the probability that you get really lucky and get no green bits
		and two or fewer red bits in your slice?
		\ifdefined\sol
		\begin{solution}
			Let $Y$ be the number of red cherry bits in $1/12$ of the cake.
			Then since the average number of red cherry bits in $1/12$ of the cake is $24/12=2$,
			$Y$ follows the \possdist\ with $\al=2$.
			Thus the probability that you get two or fewer red bits in your slice is
			\[
				\pr{Y\leq2}
				= \sumto{k}{0}{2} \pmfk{Y}{k}
				= \sumto{k}{0}{2} \frac{2^k}{k!} e^{-2}
				= 0.6767.
			\]
			Since $X$ and $Y$ are independent,
			the probability that you get no green bits
			and two or fewer red bits in your slice
			is
			\[
				\pr{X=0\mand Y\leq2}
				= \pr{X=0} \pr{Y\leq2}
				= 0.3679 \times 0.6767
				= 0.2489.
			\]
		\end{solution}
		\fi

		\item What is the probability that you get extremely lucky and get no green
		or red bits and more than five raisins in your slice?
		\ifdefined\sol
		\begin{solution}
			Let $Z$ be the number of sultana raisins in $1/12$ of the cake.
			Since the average number of sultana raisins in $1/12$
			of the cake is $48/12=4$,
			$Z$ follows the \possdist\ with $\al=4$.
			Hence,
			the probability that you get more than five raisins in your slice
			is
			\[
				\pr{Z>5} = 1 - \pr{Z\leq 5}
				= 1- \sumto{k}{0}{5} \frac{4^k}{k!} e^{-4}
				= 0.2149.
			\]
			Since $X$, $Y$, and $Z$ are mutually independent,
			the probability that you get extremely lucky
			is
			\[
				\pr{X=0} \pr{Y=0} \pr{Z>5}
				= 0.3679 \times 0.1353 \times 0.2149
				= 0.0107.
			\]
		\end{solution}
		\fi

	\end{enumerate}
\ifdefined\sol
\begin{pcode}
\begin{verbatim}
import scipy.stats as ss
x = ss.poisson(1)
y = ss.poisson(2)
z = ss.poisson(4)

print 'Pr{X=0} = %g' % x.pmf(0)
print 'Pr{Y<=2} = %g' % y.cdf(2)
print 'Pr{X=0}*Pr{Y<=2} = %g' % (x.pmf(0)*y.cdf(2))
print 'Pr{X=0}*Pr{Y=0}*Pr{Z>5} = %g' % (x.pmf(0)*y.cdf(0)*z.sf(5)
\end{verbatim}
\end{pcode}
\fi

	\item \lgprob{3.65}.
	An LCD display has $1000 \times 750$ pixels.
	A display is accepted if it has $15$ or fewer faulty pixels.
	The probability that a pixel is faulty coming out of the production line is $10^{-5}$.
	Find the proportion of displays that are accepted.
	\ifdefined\sol
	\begin{solution}
	Let $X$ be the number of faulty pixels of an LCD display.
	Then it is a \binomrv\ with $n=1000\times 750$ and $p=10^{-5}$.
	Since $n$ is huge, it (almost) perfectly approximate
	the \possdist\ with $\al = np = 7.5$.
	Hence
	the acceptance probability is
	\[
		\pr{X\leq 15}
		= \sumto{k}{0}{15} \frac{7.5^k}{k!} e^{-7.5}
		= 0.9954.
	\]
	\end{solution}
	\fi


	\item \lgprob{3.66}.
	A data center has $10,000$ disk drives.
	Suppose that a disk drive fails in a given day with probability $10^{-3}$.
	\begin{enumerate}
		\item Find the probability that there are no failures in a given day.
		\ifdefined\sol
		\begin{solution}
			Let $X$ be the number of disk drives that fail.
			Then it is a \binomrv\ with $n=10,000$ and $p=10^{-3}$.
			Again since $n$ is large, it can be well approximated
			by the \possrv\ with $\al = np = 10$.
			Thus the probability of no failure in a day is
			\[
				\pr{X=0} =
				\frac{10^0}{0!} e^{-10}
				= 4.54 \times 10^{-5}.
			\]
			\end{solution}
			\fi

			\item Find the probability that there are fewer than $10$ failures in two days.
			\ifdefined\sol
			\begin{solution}
			The number of failures for two days
			is the \possrv\ with $\al = 20$.
			Thus the probability that there are fewer than $10$ failures in two days
			is
			\[
				\pr{X<10} =
				\sumto{k}{0}{9} \frac{20^k}{k!}e^{-20}
				= 4.995 \times 10^{-3}.
			\]
		\end{solution}
		\fi

		\item Find the number of spare disk drives that should be available
		so that all failures in a day can be replaced with probability $99$\%.
		\ifdefined\sol
		\begin{solution}
			We need to find $x$ so that
			\[
				\pr{X\leq x}
				= \sumto{k}{0}{x} \frac{10^k}{k!}e^{-10}
				\geq 0.99.
			\]
			The below Python code finds that $x\geq 18$.
			Thus we need $18$ spare disk drives
			in order to replace all the failures in a day
			with probability $99$\%.
		\end{solution}
		\fi
	\end{enumerate}
\ifdefined\sol
\begin{pcode}
\begin{verbatim}
import scipy.stats as ss
n = 10000
p = 1e-3
x = ss.poisson( n * p )
print 'Pr{X=0} = %g' % x.pmf( 0 )
y = ss.poisson( 2. * n * p )
print 'Pr{Y<10} = %g' % y.cdf( 9 )

print 'Pr{X<%d} >= 0.99' % x.ppf( .99 )
\end{verbatim}
\end{pcode}
\fi


	\item \lgprob{3.67}.
	A binary communication channel has a probability of bit error of $10^{-6}$.
	Suppose that transmissions occur in blocks of $10,000$ bits.
	Let $N$ be the number of errors introduced by the channel in a transmission block.
	\begin{enumerate}
		\item Find \pr{N = 0}, \pr{N \leq 3}.
		\ifdefined\sol
		\begin{solution}
			The number of errors $N$ is the \binomrv\
			with $n=10,000$ and $p=10^{-6}$.
			Thus
			\[
				\pr{N=0} = (1-p)^n = 0.99005
			\]
			and
			\[
				\pr{N\leq 3} = \sumto{k}{0}{3} \chs{n}{k} p^k(1-p)^{n-k}
				= 1.
			\]
		\end{solution}
		\fi

		\item For what value of $p$ will the probability of $1$ or more errors in a block be
		$99$\%?
		\ifdefined\sol
		\begin{solution}
			\[
				(1-p)^n = 0.01
				\Leftrightarrow
				p = 1- (0.01)^{1/n}
				= 4.6 \times 10^{-4}.
			\]
		\end{solution}
		\fi

	\end{enumerate}

	\item \lgprob{3.68}.
	Find the mean and variance of the uniform discrete random variable that takes
	on values in the set $\{1,2,\ldots,L\}$ with equal probability.
	You will need the following formulas:
	\[ \sumkton k = \frac{n(n+1)}{2} \]
	\[ \sumkton k^2 = \frac{n(n+1)(2n+1)}{6}  \]
	\ifdefined\sol
	\begin{solution}
		Note that $\pmfxk{k} = 1/L$ for $k\in\{1,2,\ldots,L\}$.
		Then by definition
		\[
			\Exp{X} =
			\sumkzton k \pmfxk{k} =
			\frac{1}{L} \sumkzton k =
			\frac{1}{L} \frac{L(L+1)}{2} =
			\frac{L+1}{2}.
		\]
		The second moment of \X\ is
		\[
			\Exp{X^2} =
			\sumkzton k^2 \pmfxk{k} =
			\frac{1}{L} \sumkzton k^2 =
			\frac{1}{L} \frac{L(L+1)(2L+1)}{6} =
			\frac{(L+1)(2L+1)}{6},
		\]
		hence the variance of \X\ is
		\[
			\VAR{X} =
			\Exp{X^2} - \Exp{X}^2 =
			\frac{(L+1)(2L+1)}{6} - \frac{(L+1)^2}{4}
			=\frac{L^2-1}{12}.
			\]
	\end{solution}
	\fi


\end{enumerate}

\section*{Bonus problem}

\begin{enumerate}
	\item \lgprob{3.55}.
	Let \X\ be a discrete random variable that assumes only positive integer
	values and that satisfies the memoryless property.
	Show that \X\ must be a geometric random variable.
	Hint: Find an equation that must be satisfied by $g(x) = \pr{X \geq x}$.
	\ifdefined\sol
	\begin{solution}
		The memoryless property is
		\begin{equation}
		\label{eq-memoryless}
			\cpr{X\geq k+j}{X\geq j+1} = \pr{X\geq k}
		\end{equation}
		for all $j,k\geq1$.

		Now suppose a discrete \randvar\ taking only on positive integer values
		satisfies the memoryless property (\ref{eq-memoryless}).
		Let $g:\naturals \to [0,1]$
		such that $g(x) = \pr{X\geq x}$.
		Since
		\[
			\cpr{X\geq k+j}{X\geq j+1} =
			\cprbeq{X\geq k+j}{X\geq j+1} =
			\frac{\pr{X\geq k+j}}{\pr{X\geq j+1}},
		\]
		(\ref{eq-memoryless}) implies
		\[
			\frac{g(k+j)}{g(j+1)} = g(k).
		\]
		If we substitute $j$ with $1$, we have
		\[
			g(k+1) = g(k)g(2)
		\]
		and
		\begin{equation}
		\label{eq-rec}
			g(k) = g(1) c ^{k-1} = c^{k-1}
		\end{equation}
		for $k=1,2,\ldots$ where $c = g(2)$
		and the following fact is used:
		\[ g(1)=\pr{X\geq 1} = 1. \]

		Now note that $0\leq c=\pr{X\geq 2} \leq 1$.
		We consider the following three cases.
		\begin{enumerate}
			\item
			Suppose that $0<c<1$.
			Then (\ref{eq-rec}) implies
			\[
				\pmfxk{k} =
				\pr{X=k} = \pr{X\geq k } - \pr{X\geq k+1}
				= c^{k-1} - c^k
				= c^{k-1} (1-c).
			\]
			Therefore it is the \geomrv\ with $p=1-c$.

			\item Suppose that $c=0$.
			Then $\pr{X\geq2} = 1$,
			thus $\pmfxk{k} = 0$ for $k=2,3,\ldots$ and $\pmfxk{1} = 1$,
			hence it is the \geomrv\ with $p=1$.

			\item Suppose that $c=1$.
			Then (\ref{eq-rec}) implies that $\pr{X\geq k}=1$
			for any $k=1,2,\ldots$.
			However, then $\pmfxk{k} = \pr{X\geq k} - \pr{X\geq k+1}
			= 0$ for any $k$, hence the contradiction.
			Therefore $c$ cannot be $1$.
		\end{enumerate}

		In summary, any memoryless \randvar\
		taking only on positive integer values
		must be the \geomrv\ with $0< p \leq 1$
		with
		\[
			\pmfxk{k} = (1-p)^{k-1} p
		\]
		for $k=1,2,\ldots$ with convention $0^0 = 1$.
		\qed


	\end{solution}
	\fi


	\newcommand{\mux}{\expect{x}}
	\newcommand{\sigx}{\std{x}}
	\item
	The skewness of a \randvar\ \X\ is the third standardized moment,
	denoted by $\gamma_1$ and defined as
	\[
		\gamma_1 = \bExp{\left(\frac{X-\mux}{\sigx}\right)^3}.
	\]
	The \emph{skewness} is a measure of the asymmetry of the probability distribution
	of a real-valued random variable.
	The skewness value can be positive or negative, or even undefined.
	Qualitatively,
	a negative skew indicates that the tail on the left side of the probability density function
	is longer than the right side and the bulk of the values (possibly including the median)
	lie to the right of the mean.
	A positive skew indicates the opposite.
	A zero value indicates that the values are relatively evenly distributed on both sides of the mean, typically but not necessarily implying a symmetric distribution.

	\begin{enumerate}
		\item Show that
		\[
			\gamma_1 = \frac{\Exp{X^3}-3\mux \Exp{X^2} + 2 \mux^3}{\std{x}^3}.
		\]
		\ifdefined\sol
		\begin{solution}
			Note that
			\[
				(X-\mux)^3
				= X^3 - 3\mux X^2 + 3 \mux^2 X - \mux^3.
			\]
			Then the linearity of \Exp{\cdot}\ implies
			\[
				\Exp{(X-\mux)^3}
				= \Exp{X^3} - 3\mux \Exp{X^2} + 3 \mux^2 \Exp{X} - \mux^3
				= \Exp{X^3} - 3\mux \Exp{X^2} + 2 \mux^3,
			\]
			hence
			\[
				\gamma_1 = \frac{1}{\sigx^3} \Exp{(X-\mux)^3}
				= \frac{ \Exp{X^3} - 3\mux \Exp{X^2} + 2 \mux^3 }{\sigx^3}.
			\]

		\end{solution}
		\fi


		\item The skewnesses of \randvar s:
		\begin{enumerate}
			\item Show that the skewness of the \binomrv\ is
			\begin{equation}
			\label{eq-skew-binom}
				\frac{1-2p}{\sqrt{np(1-p)}}.
			\end{equation}
			\ifdefined\sol
			\begin{solution}
				Let $X$ be the \binomrv\ with parameters $n$ and $p$
				and let $Y$ be the \binomrv\ with parameters $n-1$ and $p$.
				Then the third moment of $X$ is
				\begin{eqnarray*}
					\lefteqn{
					\Exp{X^3}
					= \sumkzton k^3 \chs{n}{k}  p^k q^{n-k}
					= \sumkton k^3 \frac{n!}{k!(n-k)!} p^k q^{n-k}
					}
					\\
					&=& np \sumkton k^2 \frac{(n-1)!}{(k-1)!(n-k)!} p^{k-1} q^{n-k}
					= np \sumkton k^2 \chs{n-1}{k-1} p^{k-1} q^{n-k}
					\\
					&=& np \sumto{k}{0}{n-1} (k+1)^2 \chs{n-1}{k} p^{k} q^{(n-1)-k}
					\\
					&=& np \sumto{k}{0}{n-1} k^2 \chs{n-1}{k} p^{k} q^{(n-1)-k}
					+ 2np \sumto{k}{0}{n-1} k \chs{n-1}{k} p^{k} q^{(n-1)-k}
					\\
					&&
					+ np \sumto{k}{0}{n-1} \chs{n-1}{k} p^{k} q^{(n-1)-k}
					\\
					&=&
					np(\Exp{Y^2} + 2\Exp{Y} + 1 )
					= np(\VAR{Y^2} + \Exp{Y}^2 + 2\Exp{Y} + 1 )
					\\ &=&
					np( (n-1)pq + (n-1)^2p^2 + 2 (n-1)p + 1)
					\\ &=&
					np( n^2p^2 + n ( pq -2p^2 + 2p ) -pq + p^2 - 2p  + 1 )
					\\ &=&
					np( n^2p^2 + 3 n pq -pq + p^2 - p  + q )
					\\ &=&
					np( n^2p^2 + 3 n pq -pq - pq  + q )
					\\ &=&
					np( n^2p^2 + 3 n pq -2pq + q )
					\\ &=&
					n^3p^3 + 3 n^2 p^2q + npq(1-2p)
				\end{eqnarray*}
				and the second moment of $X$ is
				\[
					\Exp{X^2} = \VAR{X} + \Exp{X}^2
					= npq + n^2p^2
				\]
				where $q=1-p$,
				thus
				\begin{eqnarray*}
					\lefteqn{
					\Exp{X^3} - 3\mux \Exp{X^2} + 2 \mux^3
					}
					\\ &=&
					n^3p^3 + 3 n^2 p^2q + npq(1-2p)
					-3np(npq+n^2p^2) + 2n^3p^3
					\\ &=&
					n^3( p^3 -3 p^3 + 2p^3)
					+ n^2( 3p^2q - 3p^2q)
					+ npq(1-2p)
					\\ &=&
					npq(1-2p).
				\end{eqnarray*}
				Therefore the skewness of the \binomrv\ is
				\[
					\frac{npq(1-2p)}{(npq)^{3/2}}
					= \frac{1-2p}{\sqrt{np(1-p)}},
				\]
				hence the proof.
			\end{solution}
			\fi

			\item Show that the skewness of the \geomrv\ is
			\begin{equation}
			\label{eq-skew-geom}
				\frac{2-p}{\sqrt{1-p}}.
			\end{equation}
			\ifdefined\sol
			\begin{solution}
			First, we note that for $0\leq q<1$
			\[
				\sumkztoi q^k = \frac{1}{1-q}
			\]
			and differentiating both sides three times \wrt\ $q$ yields
			\[
				\sumkztoi k(k-1)(k-2)q^{k-3} = \frac{6}{(1-q)^4},
			\]
			thus
			\begin{equation}
			\label{eq-1}
				\sumkztoi k(k-1)(k-2)q^{k-1}p = \frac{6pq^2}{(1-q)^4}.
			\end{equation}
			Note that the PMF of the \geomrv\ is $\pmfxk{k} = q^{k-1}p$
			and the mean and the variance are
			\[
				\Exp{X} = 1/p,\
				\VAR{X} = (1-p)/p^2.
			\]
			Thus if we expand the left-hand-side (LHS) of (\ref{eq-1}),
			\begin{eqnarray*}
				\lefteqn{
				\sumkztoi (k^3 - 3k^2 + 2k ) q^{k-1} p
				= \sumkztoi k^3 \pmfxk{k}
				-3 \sumkztoi k^2 \pmfxk{k}
				+ 2 \sumkztoi k \pmfxk{k}
				}
				\\&=&
				\Exp{X^3}
				-3\Exp{X^2}
				+2\Exp{X}
				= 6q^2/p^3
			\end{eqnarray*}
			where we let $q=1-p$.
			Thus the third moment of the \geomrv\ is
			\begin{eqnarray*}
				\lefteqn{
				\Exp{X^3}  = 3 \Exp{X^2} - 2\Exp{X} + 6q^2/p^3
				= 3 (\VAR{X}+\Exp{X}^2) - 2\Exp{X} + 6q^2/p^3
				}
				\\&=&
				3q/p^2 + 3/p^2 - 2/p + 6q^2/p^3
				= (3pq + 3p - 2p^2 + 6q^2)/p^3,
			\end{eqnarray*}
			thus
			\begin{eqnarray*}
				\lefteqn{
				\Exp{X^3} - 3\mux \Exp{X^2} + 2 \mux^3
				= \Exp{X^3} - 3 (\VAR{X} + \Exp{X}^2)/p + 2 /p^3
				}
				\\&=&
				\Exp{X^3} - 3 (q/p^2 + 1/p^2)/p + 2 /p^3
				= ( 3pq + 3p - 2p^2 + 6q^2 - 3q - 3 + 2 )/p^3
				\\&=&
				 ( (-3 -2 +6 )p^2 + (3+3-12+3)p+(6-3-3+2))/p^3
				\\&=&
				(p^2-3p+2)/p^3
				= (1-p)(2-p)/p^3
				= q(2-p)/p^3.
			\end{eqnarray*}
			Therefore the skewness of the \geomrv\ is
			\[
				\frac{q(2-p)}{p^3} \cdot \frac{p^3}{q^{3/2}}
				= \frac{2-p}{q^{1/2}}
				= \frac{2-p}{\sqrt{1-p}},
			\]
			hence the proof.
			\end{solution}
			\fi


			\item Show that the skewness of the \possrv\ is
			\begin{equation}
			\label{eq-skew-poss}
				\alpha^{-1/2}.
			\end{equation}
			\ifdefined\sol
			\begin{solution}
			Since the PMF of the \possrv\ is
			\[
				\pmfxk{k} =  \frac{\alpha^k}{k!}e^{-\alpha},
			\]
			the third moment is
			\begin{eqnarray*}
				\lefteqn{
				\Exp{X^3}
				= \sumkztoi k^3 \frac{\alpha^k}{k!}e^{-\alpha}
				= \sumkztoi k^2 \frac{\alpha^k}{(k-1)!}e^{-\alpha}
				}
				\\&=&
				\sumkztoi (k+1)^2 \frac{\alpha^{k+1}}{k!}e^{-\alpha}
				= \alpha \sumkztoi (k^2+2k+1) \frac{\alpha^{k}}{k!}e^{-\alpha}
				\\&=&
				\alpha( \Exp{X^2} + 2\Exp{X} + 1 )
				= \alpha( \al^2 + \al + 2\al + 1 )
				= \al^3 + 3\al^2 + \al,
			\end{eqnarray*}
			thus
			\[
				\Exp{X^3} - 3\mux \Exp{X^2} + 2 \mux^3
				= \al^3 + 3\al^2 + \al - 3\al(\al^2+\al) + 2 \al^3
				= \al,
			\]
			hence
			the skewness of the \possrv\ is
			\[
				\frac{\al}{\al^{3/2}} = \al^{-1/2},
			\]
			hence the proof.
			\end{solution}
			\fi


		\end{enumerate}

		\item For the above three cases, explain how they make sense,
		\eg, explain why the skewness of the \binomrv\ goes down in the negative direction
		as $p$ grows.
		\ifdefined\sol
		\begin{solution}
		\begin{enumerate}
			\item The skewness of the \binomrv:
			\figurename~\ref{fig-binoms}
			show the three \binomrv s
			with $p=0.3$, $p=0.5$, and $p=0.8$
			and with same $n=10$.
			In the figure,
			we notice that
			if $p=0.5$,
			the PMF is symmetric around $k=5$.
			However,
			if $p<0.5$,
			it is skewed to the left
			and if $p>0.5$,
			it is skewed to the right.
			The skewness for the \binomrv\ in (\ref{eq-skew-binom})
			is drawn in \figurename~\ref{fig-skew-binom}
			as a function of $p$ when $n=10$.
			The skewness is $0$ when $p=0.5$,
			which is consistent
			with the fact that the PMF is symmetric about $k=5$
			when $p=0.5$.
			The skewness grows
			as $p$ departs from $0.5$ in the negative direction,
			which is consistent
			with the fact that the PMF has longer tail on the right
			for $p=0.3$.
			On the contrary,
			the skewness decreases below $0$
			as $p$ departs from $0.5$ in the positive direction,
			which is consistent
			with the fact that the PMF has longer tail on the left
			for $p=0.8$.

			\begin{figure}\begin{center}
			\includegraphics[width=.6\linewidth]{figures/binoms}
			\caption{The PMFs of the \binomrv s with
			$p=0.3$, $p=0.5$, and $p=0.8$ (from top to bottomo)
			with $n=10$ for all three.}
			\label{fig-binoms}
			\end{center}\end{figure}

			\begin{figure}\begin{center}
			\includegraphics[width=.6\linewidth]{figures/skew_binom}
			\caption{The skewness of the \binomrv\ as a function of $p$ when $n=10$.}
			\label{fig-skew-binom}
			\end{center}\end{figure}


			\item The skewness of the \geomrv:
			\figurename~\ref{fig-geoms}
			show the three \geomrv s
			with $p=0.2$, $p=0.6$, and $p=0.95$
			(with green lines for mean values).
			In the figure,
			we notice that
			as $p$ grows,
			the PMF is skewed more to the right.
			The skewness can be also thought
			of how mush portion of the total probability
			is to the left of the mean;
			in the bottom most PMF,
			almost all probability lies to the left of the mean.
			\figurename~\ref{fig-skew-geom}
			shows the skewness of the \geomrv\
			as a function of $p$,
			which is consistent with the above observation.

			\begin{figure}\begin{center}
			\includegraphics[width=.6\linewidth]{figures/geoms}
			\caption{The PMFs of the \binomrv s with
			$p=0.2$, $p=0.6$, and $p=0.95$}
			\label{fig-geoms}
			\end{center}\end{figure}

			\begin{figure}\begin{center}
			\includegraphics[width=.6\linewidth]{figures/skew_geom}
			\caption{The skewness of the \geomrv\ as a function of $p$.}
			\label{fig-skew-geom}
			\end{center}\end{figure}


			\item
			\figurename~\ref{fig-posss}
			show the three \possrv s
			with $\al=1$, $\al=5$, and $\al=10$.
			In the figure,
			we notice that
			as $\al$ grows,
			the PMF becomes more symmetric
			whereas for small $\al$,
			the PMF is skewed to the right,
			which is consistent
			with (\ref{eq-skew-poss}).

			\begin{figure}\begin{center}
			\includegraphics[width=.6\linewidth]{figures/posss}
			\caption{The PMFs of the \possrv s with
			with $\al=1$, $\al=5$, and $\al=10$.}
			\label{fig-posss}
			\end{center}\end{figure}
		\end{enumerate}
		\end{solution}
		\fi
	\end{enumerate}


\end{enumerate}

\end{document}


