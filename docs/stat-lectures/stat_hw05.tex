

% theorems and such

\ifdefined\theoremsandsuchchapter
\newtheorem{definition}{Definition}[chapter]
\newtheorem{theorem}{Theorem}[chapter]
\newtheorem{lemma}{Lemma}[chapter]
\newtheorem{corollary}{Corollary}[chapter]
\newtheorem{conjecture}{Conjecture}[chapter]
\else
\newtheorem{definition}{Definition}
\newtheorem{theorem}{Theorem}
\newtheorem{lemma}{Lemma}
\newtheorem{corollary}{Corollary}
\newtheorem{conjecture}{Conjecture}
\fi

\newcommand{\definename}{Definition}
\newcommand{\theoremname}{Theorem}
\newcommand{\lemmaname}{Lemma}
\newcommand{\corollaryname}{Corollary}

\newenvironment{proof}{\begin{quote}\textit{Proof}:}{\end{quote}}
\newenvironment{solution}{\begin{quote}\textit{Solution}:}{\end{quote}}
\newenvironment{pcode}{\begin{quote}\textit{Python source code}:}{\end{quote}}
\newenvironment{names}{\begin{itshape}}{\end{itshape}}

\newcommand{\qed}{{\bf Q.E.D.}}
\renewcommand{\qed}{\rule[-.5ex]{.5em}{2ex}}
\newcommand{\textfn}[1]{\textsl{#1}}

% table

\newcommand{\tparbox}[2]{%
{\parbox[c]{#1}{\center\vspace{-.4\baselineskip}{#2}\vspace{.3\baselineskip}}}}

% smile
\renewcommand{\smile}{\ensuremath{^\wedge\&^\wedge}}

% softwares

\newenvironment{code}{\begin{quote}\begin{tt}}{\end{tt}\end{quote}}


% math


\newcommand{\bmyeq}{\[}
\newcommand{\emyeq}{\]}
\newcommand{\bmyeql}[1]{\begin{equation}\label{#1}}
\newcommand{\emyeql}{\end{equation}}
%\newenvironment{myeq}{\[}{\]}
%\newenvironment{myeql}[1]{\begin{equation}\label{#1}}{\end{equation}}

\newcommand{\onehalf}{\ensuremath{\frac{1}{2}}}
\newcommand{\onethird}{\ensuremath{\frac{1}{3}}}
\newcommand{\onefourth}{\frac{1}{4}}
\newcommand{\sumft}[2]{\sum_{#1}^{#2}}
\newcommand{\sumioneton}{\sumionetok{n}}
\newcommand{\sumionetok}[1]{\sum_{i=1}^#1}
\newcommand{\sumoneto}[2]{\sum_{#1=1}^{#2}}
\newcommand{\sumoneton}[1]{\sumoneto{#1}{n}}
\newcommand{\prodoneto}[2]{\prod_{#1=1}^{#2}}
\newcommand{\prodoneton}[1]{\prodoneto{#1}{n}}

\newcommand{\listoneto}[1]{\ensuremath{1,2,\ldots,#1}}
\newcommand{\diagxoneto}[2]{\ensuremath{\diag({#1}_1,{#1}_2,\ldots,{#1}_{#2})}}
\newcommand{\setxoneto}[2]{\ensuremath{\{\listxoneto{#1}{#2}\}}}
\newcommand{\listxoneto}[2]{\ensuremath{{#1}_1,{#1}_2,\ldots,{#1}_{#2}}}
\newcommand{\setoneto}[1]{\ensuremath{\{1,2,\ldots,#1\}}}

\newcommand{\diagmat}[2]{\diagoneto{#1}{#2}}

\newcommand{\setoneton}[1]{\setoneton{#1}}
\newcommand{\setxtok}[2]{\setxoneto{#1}{#2}}



\newcommand{\colvectwo}[2]{\ensuremath{\begin{my-matrix}{c}{#1}\\{#2}\end{my-matrix}}}
\newcommand{\colvecthree}[3]{\ensuremath{\begin{my-matrix}{c}{#1}\\{#2}\\{#3}\end{my-matrix}}}
\newcommand{\colvecfour}[4]{\ensuremath{\begin{my-matrix}{c}{#1}\\{#2}\\{#3}\\{#4}\end{my-matrix}}}
\newcommand{\rowvectwo}[2]{\ensuremath{\begin{my-matrix}{cc}{#1}&{#2}\end{my-matrix}}}
\newcommand{\rowvecthree}[3]{\ensuremath{\begin{my-matrix}{ccc}{#1}&{#2}&{#3}\end{my-matrix}}}
\newcommand{\rowvecfour}[4]{\ensuremath{\begin{my-matrix}{cccc}{#1}&{#2}&{#3}&{#4}\end{my-matrix}}}
\newcommand{\diagtwo}[2]{\ensuremath{\begin{my-matrix}{cc}{#1}&0\\0& {#2}\end{my-matrix}}}
\newcommand{\mattwotwo}[4]{\ensuremath{\begin{my-matrix}{cc}{#1}&{#2}\\{#3}&{#4}\end{my-matrix}}}
\newcommand{\bigmat}[9]{\ensuremath{\begin{my-matrix}{cccc} #1&#2&\cdots&#3\\ #4&#5&\cdots&#6\\ \vdots&\vdots&\ddots&\vdots\\ #7&#8&\cdots&#9 \end{my-matrix}}}
%\newcommand{\matdotff}[9]{\matff{#1}{#2}{\cdots}{#3}{#4}{#5}{\cdots}{#6}{\vdots}{\vdots}{\ddots}{\vdots}{#7}{#8}{\cdots}{#9}}

\newcommand{\matthreethree}[9]{%
	\begin{my-matrix}{ccc}%
	{#1}&{#2}&{#3}%
	\\{#4}&{#5}&{#6}%
	\\{#7}&{#8}&{#9}%
	\end{my-matrix}%
}
\newcommand{\matthreethreeT}[9]{%
	\matthreethree%
	{#1}{#4}{#7}%
	{#2}{#5}{#8}%
	{#3}{#6}{#9}%
}

\newenvironment{my-matrix}{\left[\begin{array}}{\end{array}\right]}

\newcommand{\mbyn}[2]{\ensuremath{#1\times #2}}
\newcommand{\realmat}[2]{\ensuremath{\reals^{\mbyn{#1}{#2}}}}
\newcommand{\realsqmat}[1]{\ensuremath{\reals^{\mbyn{#1}{#1}}}}
\newcommand{\defequal}{\triangleq}

\newcommand{\dspace}{\,}
\newcommand{\dx}{{\dspace dx}}
\newcommand{\dy}{{\dspace dy}}
\newcommand{\dt}{{\dspace dt}}
\newcommand{\intspace}{\!\!}
\newcommand{\sqrtspace}{\,}
\newcommand{\aftersqrtspace}{\sqrtspace}
\newcommand{\dividespace}{\!}

\newcommand{\one}{\mathbf{1}}
\newcommand{\arginf}{\mathop{\mathrm{arginf}}}
\newcommand{\argsup}{\mathop{\mathrm {argsup}}}
\newcommand{\argmin}{\mathop{\mathrm {argmin}}}
\newcommand{\argmax}{\mathop{\mathrm {argmax}}}

\newcommand{\reals}{{\mbox{\bf R}}}
\newcommand{\preals}{{\reals_+}}
\newcommand{\ppreals}{{\reals_{++}}}
\newcommand{\complexes}{{\mbox{\bf C}}}
\newcommand{\integers}{{\mbox{\bf Z}}}
\newcommand{\naturals}{{\mbox{\bf N}}}
\newcommand{\rationals}{{\mbox{\bf Q}}}

\newcommand{\realspace}[2]{\reals^{#1\times #2}}
\newcommand{\compspace}[2]{\complexes^{#1\times #2}}

\newcommand{\identity}{\mbox{\bf I}}
\newcommand{\nullspace}{{\mathcal N}}
\newcommand{\range}{{\mathcal R}}

\newcommand{\set}[2]{\{#1~|~#2\}}
\newcommand{\bigset}[2]{\left\{#1\left|#2\right.\right\}}


% operators

\newcommand{\Expect}{\mathop{\bf E{}}}
\newcommand{\Var}{\mathop{\bf  Var{}}}
\newcommand{\Cov}{\mathop{\bf Cov}}
\newcommand{\Prob}{\mathop{\bf Prob}}



% operators

\newcommand{\smallo}{{\mathop{\bf o}}}

\newcommand{\jac}{{\mathcal{J}}}
\newcommand{\diag}{\mathop{\bf diag}}
\newcommand{\Rank}{\mathop{\bf Rank}}
\newcommand{\rank}{\mathop{\bf rank}}
\newcommand{\dimn}{\mathop{\bf dim}}
\newcommand{\Tr}{\mathop{\bf Tr}} % trance
\newcommand{\dom}{\mathop{\bf dom}}
\newcommand{\Det}{\det}
\newcommand{\adj}{\mathop{\bf adj}}
%\newcommand{\Det}{{\mathop{\bf Det}}}
%\newcommand{\determinant}[1]{|#1|}
\newcommand{\sign}{{\mathop{\bf sign}}}
\newcommand{\dist}{{\mathop{\bf dist}}}




% probability space

\newcommand{\probsubset}{{\mathcal{P}}}
\newcommand{\eset}{{\mathcal{E}}}

\newcommand{\probspace}{{\Omega}}


% acronyms

\newcommand{\eg}{{\it e.g.}}
\newcommand{\ie}{{\it i.e.}}
\newcommand{\etc}{{\it etc.}}
\newcommand{\cf}{{\it cf.}}


% some commands & environments for making lecture notes

\newcounter{oursection}
\newcommand{\oursection}[1]{
 \addtocounter{oursection}{1}
 \setcounter{equation}{0}
 \clearpage \begin{center} {\Huge\bfseries #1} \end{center}
 {\vspace*{0.15cm} \hrule height.3mm} \bigskip
 \addcontentsline{toc}{section}{#1}
}
\newcommand{\oursectionf}[1]{  % for use with foiltex
 \addtocounter{oursection}{1}
 \setcounter{equation}{0}
 \foilhead[-.5cm]{#1 \vspace*{0.8cm} \hrule height.3mm }
 \LogoOn
}
\newcommand{\oursectionfl}[1]{  % for use with foiltex landscape
 \addtocounter{oursection}{1}
 \setcounter{equation}{0}
 \foilhead[-1.0cm]{#1}
 \LogoOn
}

\newcounter{lecture}
\newcommand{\lecture}[1]{
 \addtocounter{lecture}{1}
 \setcounter{equation}{0}
 \setcounter{page}{1}
 \renewcommand{\theequation}{\arabic{equation}}
 \renewcommand{\thepage}{\arabic{lecture} -- \arabic{page}}
 \raggedright \sffamily \LARGE
 \cleardoublepage\begin{center}
 {\Huge\bfseries Lecture \arabic{lecture} \bigskip \\ #1}\end{center}
 {\vspace*{0.15cm} \hrule height.3mm} \bigskip
 \addcontentsline{toc}{chapter}{\protect\numberline{\arabic{lecture}}{#1}}
 \pagestyle{myheadings}
 \markboth{Lecture \arabic{lecture}}{#1}
}
\newcommand{\lecturef}[1]{
 \addtocounter{lecture}{1}
 \setcounter{equation}{0}
 \setcounter{page}{1}
 \renewcommand{\theequation}{\arabic{equation}}
 \renewcommand{\thepage}{\arabic{lecture}--\arabic{page}}
 \parindent 0pt
 \MyLogo{#1}
 \rightfooter{\thepage}
 \leftheader{}
 \rightheader{}
 \LogoOff
 \begin{center}
 {\large\bfseries Lecture \arabic{lecture} \bigskip \\ #1}
 \end{center}
 {\vspace*{0.8cm} \hrule height.3mm}
 \bigskip
}
\newcommand{\lecturefl}[1]{   % use with foiltex landscape
 \addtocounter{lecture}{1}
 \setcounter{equation}{0}
 \setcounter{page}{1}
 \renewcommand{\theequation}{\arabic{equation}}
 \renewcommand{\thepage}{\arabic{lecture}--\arabic{page}}
 \addtolength{\topmargin}{-1.5cm}
 \raggedright
 \parindent 0pt
 \rightfooter{\thepage}
 \leftheader{}
 \rightheader{}
 \LogoOff
 \begin{center}
 {\Large \bfseries Lecture \arabic{lecture} \\*[\bigskipamount] {#1}}
 \end{center}
 \MyLogo{#1}
}


%% theorems and such

\ifdefined\theoremsandsuchchapter
\newtheorem{definition}{Definition}[chapter]
\newtheorem{theorem}{Theorem}[chapter]
\newtheorem{lemma}{Lemma}[chapter]
\newtheorem{corollary}{Corollary}[chapter]
\newtheorem{conjecture}{Conjecture}[chapter]
\else
\newtheorem{definition}{Definition}
\newtheorem{theorem}{Theorem}
\newtheorem{lemma}{Lemma}
\newtheorem{corollary}{Corollary}
\newtheorem{conjecture}{Conjecture}
\fi

\newcommand{\definename}{Definition}
\newcommand{\theoremname}{Theorem}
\newcommand{\lemmaname}{Lemma}
\newcommand{\corollaryname}{Corollary}

\newenvironment{proof}{\begin{quote}\textit{Proof}:}{\end{quote}}
\newenvironment{solution}{\begin{quote}\textit{Solution}:}{\end{quote}}
\newenvironment{pcode}{\begin{quote}\textit{Python source code}:}{\end{quote}}
\newenvironment{names}{\begin{itshape}}{\end{itshape}}

\newcommand{\qed}{{\bf Q.E.D.}}
\renewcommand{\qed}{\rule[-.5ex]{.5em}{2ex}}
\newcommand{\textfn}[1]{\textsl{#1}}

% table

\newcommand{\tparbox}[2]{%
{\parbox[c]{#1}{\center\vspace{-.4\baselineskip}{#2}\vspace{.3\baselineskip}}}}

% smile
\renewcommand{\smile}{\ensuremath{^\wedge\&^\wedge}}

% softwares

\newenvironment{code}{\begin{quote}\begin{tt}}{\end{tt}\end{quote}}


% math


\newcommand{\bmyeq}{\[}
\newcommand{\emyeq}{\]}
\newcommand{\bmyeql}[1]{\begin{equation}\label{#1}}
\newcommand{\emyeql}{\end{equation}}
%\newenvironment{myeq}{\[}{\]}
%\newenvironment{myeql}[1]{\begin{equation}\label{#1}}{\end{equation}}

\newcommand{\onehalf}{\ensuremath{\frac{1}{2}}}
\newcommand{\onethird}{\ensuremath{\frac{1}{3}}}
\newcommand{\onefourth}{\frac{1}{4}}
\newcommand{\sumft}[2]{\sum_{#1}^{#2}}
\newcommand{\sumioneton}{\sumionetok{n}}
\newcommand{\sumionetok}[1]{\sum_{i=1}^#1}
\newcommand{\sumoneto}[2]{\sum_{#1=1}^{#2}}
\newcommand{\sumoneton}[1]{\sumoneto{#1}{n}}
\newcommand{\prodoneto}[2]{\prod_{#1=1}^{#2}}
\newcommand{\prodoneton}[1]{\prodoneto{#1}{n}}

\newcommand{\listoneto}[1]{\ensuremath{1,2,\ldots,#1}}
\newcommand{\diagxoneto}[2]{\ensuremath{\diag({#1}_1,{#1}_2,\ldots,{#1}_{#2})}}
\newcommand{\setxoneto}[2]{\ensuremath{\{\listxoneto{#1}{#2}\}}}
\newcommand{\listxoneto}[2]{\ensuremath{{#1}_1,{#1}_2,\ldots,{#1}_{#2}}}
\newcommand{\setoneto}[1]{\ensuremath{\{1,2,\ldots,#1\}}}

\newcommand{\diagmat}[2]{\diagoneto{#1}{#2}}

\newcommand{\setoneton}[1]{\setoneton{#1}}
\newcommand{\setxtok}[2]{\setxoneto{#1}{#2}}



\newcommand{\colvectwo}[2]{\ensuremath{\begin{my-matrix}{c}{#1}\\{#2}\end{my-matrix}}}
\newcommand{\colvecthree}[3]{\ensuremath{\begin{my-matrix}{c}{#1}\\{#2}\\{#3}\end{my-matrix}}}
\newcommand{\colvecfour}[4]{\ensuremath{\begin{my-matrix}{c}{#1}\\{#2}\\{#3}\\{#4}\end{my-matrix}}}
\newcommand{\rowvectwo}[2]{\ensuremath{\begin{my-matrix}{cc}{#1}&{#2}\end{my-matrix}}}
\newcommand{\rowvecthree}[3]{\ensuremath{\begin{my-matrix}{ccc}{#1}&{#2}&{#3}\end{my-matrix}}}
\newcommand{\rowvecfour}[4]{\ensuremath{\begin{my-matrix}{cccc}{#1}&{#2}&{#3}&{#4}\end{my-matrix}}}
\newcommand{\diagtwo}[2]{\ensuremath{\begin{my-matrix}{cc}{#1}&0\\0& {#2}\end{my-matrix}}}
\newcommand{\mattwotwo}[4]{\ensuremath{\begin{my-matrix}{cc}{#1}&{#2}\\{#3}&{#4}\end{my-matrix}}}
\newcommand{\bigmat}[9]{\ensuremath{\begin{my-matrix}{cccc} #1&#2&\cdots&#3\\ #4&#5&\cdots&#6\\ \vdots&\vdots&\ddots&\vdots\\ #7&#8&\cdots&#9 \end{my-matrix}}}
%\newcommand{\matdotff}[9]{\matff{#1}{#2}{\cdots}{#3}{#4}{#5}{\cdots}{#6}{\vdots}{\vdots}{\ddots}{\vdots}{#7}{#8}{\cdots}{#9}}

\newcommand{\matthreethree}[9]{%
	\begin{my-matrix}{ccc}%
	{#1}&{#2}&{#3}%
	\\{#4}&{#5}&{#6}%
	\\{#7}&{#8}&{#9}%
	\end{my-matrix}%
}
\newcommand{\matthreethreeT}[9]{%
	\matthreethree%
	{#1}{#4}{#7}%
	{#2}{#5}{#8}%
	{#3}{#6}{#9}%
}

\newenvironment{my-matrix}{\left[\begin{array}}{\end{array}\right]}

\newcommand{\mbyn}[2]{\ensuremath{#1\times #2}}
\newcommand{\realmat}[2]{\ensuremath{\reals^{\mbyn{#1}{#2}}}}
\newcommand{\realsqmat}[1]{\ensuremath{\reals^{\mbyn{#1}{#1}}}}
\newcommand{\defequal}{\triangleq}

\newcommand{\dspace}{\,}
\newcommand{\dx}{{\dspace dx}}
\newcommand{\dy}{{\dspace dy}}
\newcommand{\dt}{{\dspace dt}}
\newcommand{\intspace}{\!\!}
\newcommand{\sqrtspace}{\,}
\newcommand{\aftersqrtspace}{\sqrtspace}
\newcommand{\dividespace}{\!}

\newcommand{\one}{\mathbf{1}}
\newcommand{\arginf}{\mathop{\mathrm{arginf}}}
\newcommand{\argsup}{\mathop{\mathrm {argsup}}}
\newcommand{\argmin}{\mathop{\mathrm {argmin}}}
\newcommand{\argmax}{\mathop{\mathrm {argmax}}}

\newcommand{\reals}{{\mbox{\bf R}}}
\newcommand{\preals}{{\reals_+}}
\newcommand{\ppreals}{{\reals_{++}}}
\newcommand{\complexes}{{\mbox{\bf C}}}
\newcommand{\integers}{{\mbox{\bf Z}}}
\newcommand{\naturals}{{\mbox{\bf N}}}
\newcommand{\rationals}{{\mbox{\bf Q}}}

\newcommand{\realspace}[2]{\reals^{#1\times #2}}
\newcommand{\compspace}[2]{\complexes^{#1\times #2}}

\newcommand{\identity}{\mbox{\bf I}}
\newcommand{\nullspace}{{\mathcal N}}
\newcommand{\range}{{\mathcal R}}

\newcommand{\set}[2]{\{#1~|~#2\}}
\newcommand{\bigset}[2]{\left\{#1\left|#2\right.\right\}}


% operators

\newcommand{\Expect}{\mathop{\bf E{}}}
\newcommand{\Var}{\mathop{\bf  Var{}}}
\newcommand{\Cov}{\mathop{\bf Cov}}
\newcommand{\Prob}{\mathop{\bf Prob}}



% operators

\newcommand{\smallo}{{\mathop{\bf o}}}

\newcommand{\jac}{{\mathcal{J}}}
\newcommand{\diag}{\mathop{\bf diag}}
\newcommand{\Rank}{\mathop{\bf Rank}}
\newcommand{\rank}{\mathop{\bf rank}}
\newcommand{\dimn}{\mathop{\bf dim}}
\newcommand{\Tr}{\mathop{\bf Tr}} % trance
\newcommand{\dom}{\mathop{\bf dom}}
\newcommand{\Det}{\det}
\newcommand{\adj}{\mathop{\bf adj}}
%\newcommand{\Det}{{\mathop{\bf Det}}}
%\newcommand{\determinant}[1]{|#1|}
\newcommand{\sign}{{\mathop{\bf sign}}}
\newcommand{\dist}{{\mathop{\bf dist}}}




% probability space

\newcommand{\probsubset}{{\mathcal{P}}}
\newcommand{\eset}{{\mathcal{E}}}

\newcommand{\probspace}{{\Omega}}


% acronyms

\newcommand{\eg}{{\it e.g.}}
\newcommand{\ie}{{\it i.e.}}
\newcommand{\etc}{{\it etc.}}
\newcommand{\cf}{{\it cf.}}


% some commands & environments for making lecture notes

\newcounter{oursection}
\newcommand{\oursection}[1]{
 \addtocounter{oursection}{1}
 \setcounter{equation}{0}
 \clearpage \begin{center} {\Huge\bfseries #1} \end{center}
 {\vspace*{0.15cm} \hrule height.3mm} \bigskip
 \addcontentsline{toc}{section}{#1}
}
\newcommand{\oursectionf}[1]{  % for use with foiltex
 \addtocounter{oursection}{1}
 \setcounter{equation}{0}
 \foilhead[-.5cm]{#1 \vspace*{0.8cm} \hrule height.3mm }
 \LogoOn
}
\newcommand{\oursectionfl}[1]{  % for use with foiltex landscape
 \addtocounter{oursection}{1}
 \setcounter{equation}{0}
 \foilhead[-1.0cm]{#1}
 \LogoOn
}

\newcounter{lecture}
\newcommand{\lecture}[1]{
 \addtocounter{lecture}{1}
 \setcounter{equation}{0}
 \setcounter{page}{1}
 \renewcommand{\theequation}{\arabic{equation}}
 \renewcommand{\thepage}{\arabic{lecture} -- \arabic{page}}
 \raggedright \sffamily \LARGE
 \cleardoublepage\begin{center}
 {\Huge\bfseries Lecture \arabic{lecture} \bigskip \\ #1}\end{center}
 {\vspace*{0.15cm} \hrule height.3mm} \bigskip
 \addcontentsline{toc}{chapter}{\protect\numberline{\arabic{lecture}}{#1}}
 \pagestyle{myheadings}
 \markboth{Lecture \arabic{lecture}}{#1}
}
\newcommand{\lecturef}[1]{
 \addtocounter{lecture}{1}
 \setcounter{equation}{0}
 \setcounter{page}{1}
 \renewcommand{\theequation}{\arabic{equation}}
 \renewcommand{\thepage}{\arabic{lecture}--\arabic{page}}
 \parindent 0pt
 \MyLogo{#1}
 \rightfooter{\thepage}
 \leftheader{}
 \rightheader{}
 \LogoOff
 \begin{center}
 {\large\bfseries Lecture \arabic{lecture} \bigskip \\ #1}
 \end{center}
 {\vspace*{0.8cm} \hrule height.3mm}
 \bigskip
}
\newcommand{\lecturefl}[1]{   % use with foiltex landscape
 \addtocounter{lecture}{1}
 \setcounter{equation}{0}
 \setcounter{page}{1}
 \renewcommand{\theequation}{\arabic{equation}}
 \renewcommand{\thepage}{\arabic{lecture}--\arabic{page}}
 \addtolength{\topmargin}{-1.5cm}
 \raggedright
 \parindent 0pt
 \rightfooter{\thepage}
 \leftheader{}
 \rightheader{}
 \LogoOff
 \begin{center}
 {\Large \bfseries Lecture \arabic{lecture} \\*[\bigskipamount] {#1}}
 \end{center}
 \MyLogo{#1}
}



\usepackage{fullpage}
\usepackage{fancyhdr}

\usepackage{graphicx}

\pagestyle{fancy}
\fancyhead[R]{Probability and Statistics for Electrical Engineering}

\addtolength{\headsep}{.5cm}



\newcommand{\mfor}{\mbox{ for }}
\newcommand{\mand}{\mbox{ and }}
\newcommand{\mforall}{\mbox{ for all }}
\newcommand{\mif}{\mbox{if }}
\newcommand{\ld}{\ensuremath{\lambda}}


\newcommand{\lgprob}[1]{LGProb~{#1}}
\newcommand{\lgfig}[1]{LGFig~{#1}}
\newcommand{\lgexam}[2]{LGExample~{#1} {\bf #2}}
\newcommand{\lgexamref}[1]{LGExample~{#1}}
\newcommand{\lgeq}[1]{LGEquation~(#1)}

\newcommand{\br}[1]{\{#1\}}

\newcommand{\randvar}{random variable}
\newcommand{\var}{variance}
\newcommand{\bernrv}{{Bernoulli \randvar}}
\newcommand{\binomrv}{{binomial \randvar}}
\newcommand{\geomrv}{{geometric \randvar}}
\newcommand{\possrv}{{Poisson \randvar}}

\newcommand{\unifrv}{{uniform \randvar}}
\newcommand{\exprv}{{exponential \randvar}}
\newcommand{\gaussrv}{{uniform \randvar}}

\newcommand{\possdist}{{Poisson distribution}}
\newcommand{\cond}{conditional}

\newcommand{\unitfcn}{unit step function}
\newcommand{\delfcn}{delta function}



\newcommand{\al}{\ensuremath{\alpha}}


\newcommand{\sA}{\ensuremath{A}}
\newcommand{\sB}{\ensuremath{B}}
\newcommand{\Ai}[1]{\ensuremath{A_{#1}}}
\newcommand{\Bi}[1]{\ensuremath{B_{#1}}}
\newcommand{\Ei}[1]{\ensuremath{E_{#1}}}
\newcommand{\Ak}{\Ai{k}}
\newcommand{\X}{\ensuremath{X}}
\newcommand{\xk}{\ensuremath{x_k}}
\newcommand{\mzeta}{\ensuremath{\zeta}}

\newcommand{\comp}[1]{\ensuremath{{#1}^c}}
\newcommand{\diff}[2]{{#1}\backslash{#2}}
\newcommand{\sspace}{\ensuremath{S}}
\newcommand{\ssx}{\ensuremath{S_X}}
\newcommand{\evcl}{\ensuremath{\mathcal{F}}}

\newcommand{\pr}[1]{\ensuremath{P[#1]}}
%\renewcommand{\pr}[1]{\ensuremath{{\bf Pr}\{#1\}}}
\newcommand{\prb}[1]{\ensuremath{P[\{#1\}]}}
\newcommand{\bpr}[1]{\ensuremath{P\left[#1\right]}}
\newcommand{\cpr}[2]{\pr{#1|#2}}
\newcommand{\cpreq}[2]{\frac{\pr{#1\cap #2}}{\pr{#2}}}
\newcommand{\cprbeq}[2]{\cpreq{\br{#1}}{\br{#2}}}
\newcommand{\appr}[2]{\pr{#1|#2}\pr{#2}}
\newcommand{\relfreq}[1]{f_{#1}}

\newcommand{\expect}[1]{\ensuremath{m_{#1}}}
\newcommand{\cexp}[2]{\expect{#1|#2}}
\renewcommand{\Expect}[1]{\ensuremath{{\bf E}[#1]}}
\newcommand{\bExp}[1]{\ensuremath{{\bf E}\left[#1\right]}}
\newcommand{\Exp}[1]{\Expect{#1}}
\newcommand{\cExp}[2]{\Exp{#1|#2}}
\newcommand{\samplemean}[2]{\langle{#1}\rangle_{#2}}
\newcommand{\std}[1]{\ensuremath{\sigma_{#1}}}
\newcommand{\VAR}[1]{\ensuremath{{\bf VAR}[#1]}}
\newcommand{\cVAR}[2]{\VAR{#1|#2}}
\newcommand{\STD}[1]{\ensuremath{{\bf STD}[#1]}}

\newcommand{\ep}[1]{\ensuremath{E_{#1}}}
\renewcommand{\ev}[1]{\ensuremath{A_{#1}}}

\newcommand{\borel}{\ensuremath{\mathcal{B}}}

\newcommand{\chs}[2]{\ensuremath{{{#1} \choose {#2}}}}

\newcommand{\bigcupto}[3]{\bigcup_{#1=#2}^{#3}}
\newcommand{\bigcupkton}{\bigcupto{k}{1}{n}}
\newcommand{\bigcupktoi}{\bigcupto{k}{1}{\infty}}

\newcommand{\bigcapto}[3]{\bigcap_{#1=#2}^{#3}}
\newcommand{\bigcapkton}{\bigcapto{k}{1}{n}}

\newcommand{\sumto}[3]{\sum_{#1=#2}^{#3}}
\newcommand{\sumkton}{\sumto{k}{1}{n}}
\newcommand{\sumiton}{\sumto{i}{1}{n}}
\newcommand{\sumktoi}{\sumto{k}{1}{\infty}}
\newcommand{\sumntoi}{\sumto{n}{1}{\infty}}
\newcommand{\sumkzton}{\sumto{k}{0}{n}}
\newcommand{\sumkztoi}{\sumto{k}{0}{\infty}}
\newcommand{\sumnztoi}{\sumto{n}{0}{\infty}}

\newcommand{\xcomma}[2]{\ensuremath{{#1}_1, {#1}_2, \ldots, {#1}_{#2} }}
\newcommand{\xcommai}[1]{\ensuremath{{#1}_1, {#1}_2, \ldots }}
\newcommand{\xplus}[2]{\ensuremath{{#1}_1 + {#1}_2 + \cdots + {#1}_{#2} }}
\newcommand{\xcap}[2]{\ensuremath{{#1}_1 \cap {#1}_2 \cap \cdots \cap {#1}_{#2} }}

\newcommand{\ceil}[1]{\ensuremath{\lceil #1 \rceil}}
\newcommand{\floor}[1]{\ensuremath{\lfloor #1 \rfloor}}
\newcommand{\kmax}{\ensuremath{k_\mathrm{max}}}

\newcommand{\posspmf}[2]{\frac{{#1}^{#2}}{{#2}!}e^{-{#1}}}
\newcommand{\binompmf}[3]{\chs{{#1}}{{#3}}{#2}^{#3}(1-{#2})^{{#1}-{#3}}}
\newcommand{\dbinompmf}{\binompmf{n}{p}{k}}


%\newcommand{\numbereveryeqns}{}

\ifdefined\numbereveryeqns
	\newcommand{\beq}{\begin{equation}}
	\newcommand{\eeq}{\end{equation}}
\else
	\newcommand{\beq}{\[}
	\newcommand{\eeq}{\]}
\fi

\newcommand{\beql}[1]{\begin{equation}\label{#1}}
\newcommand{\eeql}{\end{equation}}

\newcommand{\optional}{$\dagger$}

\newcommand{\pmf}[1]{p_{#1}}
\newcommand{\pmfk}[2]{\ensuremath{\pmf{#1}(#2)}}
\newcommand{\pmfxk}[1]{\pmfk{X}{#1}}
\newcommand{\cpmfk}[3]{\ensuremath{\pmf{#1}(#2|#3)}}
\newcommand{\cpmfxk}[2]{\cpmfk{X}{#1}{#2}}

\newcommand{\cdf}[1]{F_{#1}}
\newcommand{\cdfk}[2]{\ensuremath{\cdf{#1}(#2)}}
\newcommand{\cdfxk}[1]{\cdfk{X}{#1}}
\newcommand{\ccdfk}[3]{\ensuremath{\cdf{#1}(#2|#3)}}
\newcommand{\ccdfxk}[2]{\ccdfk{X}{#1}{#2}}

\newcommand{\pdf}[1]{f_{#1}}
\newcommand{\pdfk}[2]{\ensuremath{\pdf{#1}(#2)}}
\newcommand{\pdfxk}[1]{\pdfk{X}{#1}}
\newcommand{\cpdfk}[3]{\ensuremath{\pdf{#1}(#2|#3)}}
\newcommand{\cpdfxk}[2]{\cpdfk{X}{#1}{#2}}

\newcommand{\indi}[1]{I_{#1}}

\newcommand{\acapb}{\ensuremath{A\cap B}}
\newcommand{\acupb}{\ensuremath{A\cup B}}


\newcommand{\limninfty}{\lim_{n\to\infty}}
\newcommand{\intinftoinf}{\int_{-\infty}^\infty}
\newcommand{\intinfto}[1]{\int_{-\infty}^{#1}}

\newcommand{\ddx}{\frac{d}{dx}}

\newcommand{\gausspdf}[3]{\frac{1}{\sqrt{2\pi} {#3}} e^{-({#1}{#2})^2/2{#3}^2} }



\renewcommand{\emph}[1]{{\it #1}}

\begin{document}
\setlength{\headheight}{15pt}
\maketitle

\begin{enumerate}

	\item \lgprob{4.4}.
	An urn contains $8$ \$$1$ bills
	and two \$$5$ bills.
	Let \X\ be the total amount that results when two bills are drawn from the urn without replacement,
	and
	let $Y$ be the total amount that results when two bills are drawn from the urn with replacement.
	\begin{enumerate}
		\item Plot and compare the CDF's of the random variables.
		\item Use the CDF to compare the probabilities of the following events in the two problems:
		$\{X = \$2\}$,
		$\{X < \$7\}$,
		$\{X \geq 6\}$.
	\end{enumerate}


	\item \lgprob{4.7}.
	A point is selected at random inside a square defined by
	$\{(x, y): 0 \leq x \leq b, 0 \leq y \leq b\}$.
	Assume the point is equally likely to fall anywhere in the square.
	Let the random variable $Z$ be given by the minimum of the two coordinates of the point
	where the dart lands.
	\begin{enumerate}
		\item Find the sample space \sspace\
		and the sample space of $Z$, $\sspace_{Z}$.
		\item Show the mapping from \sspace\ to $\sspace_{Z}$.
		\item Find the region in the square corresponding to the event
		$\{Z \leq z\}$.
		\item Find and plot the CDF of $Z$.
		\item Use the CDF to find: $\pr{Z > 0}$, $\pr{Z > b}$, $\pr{Z \leq b/2}$, $\pr{Z > b/4}$.
	\end{enumerate}


	\item \lgprob{4.11}.
	The random variable \X\ is uniformly distributed in the interval
	$[-1, 2]$.
	\begin{enumerate}
		\item Find and plot the CDF of \X.
		\item Use the CDF to find the probabilities of the following events:
		$\{X \leq 0\}$,
		$\{|X - 0.5| < 1\}$,
		and $C = \{X > -0.5\}$.
	\end{enumerate}

	\item \lgprob{4.15}.
	For $\beta > 0$ and $\lambda > 0$,
	the Weibull random variable $X$ has CDF:
	\begin{equation}
		\cdfxk{x} = \left\{\begin{array}{ll}
		0	&\mfor x< 0
		\\1-e^{(x/\lambda)^\beta}	&\mfor x \geq 0.
		\end{array}\right.
	\end{equation}
	\begin{enumerate}
	\item Plot the CDF of $X$ for $\beta = 0.5$, $1$, and $2$.
	\item Find the probability $\pr{j\lambda < X < (j + 1)\lambda}$
	and $\pr{X < j\lambda}$.
	\item Plot $\log \pr{X > x}$ vs. $\log x$.

	\end{enumerate}

	\item \lgprob{4.20}.
	\begin{enumerate}
		\item Find and plot the PDF of the random variable $Z$ in \lgprob{4.7}.
		\item Use the PDF to find the probability that the minimum is greater than $b/3$.
	\end{enumerate}

\end{enumerate}

\section*{Bonus problems}

\begin{enumerate}
	\item \lgprob{4.29}.
	Let $C$ be an event for which $\pr{C} > 0$.
	Show that $\ccdfxk{x}{C}$
	satisfies the eight properties of a CDF.
	\ifdefined\sol
	\begin{solution}
	First, we note the continuity of probability function:
	\begin{equation}
	\label{eq-cont-prob-fcn}
		\lim_{n\to\infty} \pr{A_n} = \bpr{\lim_{n\to\infty} A_n}
	\end{equation}
	for an increasing or decreasing sequence of events in \evcl,
	$A_1$, $A_2$, \ldots.

	\renewcommand{\labelenumii}{(\roman{enumii})}
	\begin{enumerate}
		\item
		By definition,
		\[
			\ccdfxk{x}{C} = \cpr{X \leq x}{C}
			= \frac{\pr{\{X\leq x\} \cap C}}{\pr{C}}.
		\]
		Since $0\leq \pr{\{X\leq x\} \cap C} \leq \pr{C}$, we have
		\begin{equation}
			\label{eq-1}
			0 \leq \cdfxk{x}{C} \leq 1.
		\end{equation}

		\item
		Now let us consider the following increasing sequence of events
		\[
			A_n = \{X \leq n \} \cap C.
		\]
		Then by definition
		\[
			\lim_{n\to\infty} A_n
			= \bigcup_{n=1}^\infty A_n
			= \bigcup_{n=1}^\infty \left( \{X \leq n \} \cap C \right)
			= \left( \bigcup_{n=1}^\infty \{X \leq n \} \right) \cap C
			= \sspace \cap C
			= C.
		\]
		Therefore (\ref{eq-cont-prob-fcn}) implies
		\begin{eqnarray}
		\nonumber
			\lefteqn{
			\lim_{x\to\infty} \ccdfxk{x}{C}
			= \lim_{n\to\infty} \ccdfxk{n}{C}
			}
			\\
			&=& \frac{\lim_{n\to\infty} \pr{\{X\leq n\}\cap C}}{\pr{C}}
			= \frac{\lim_{n\to\infty} \pr{A_n}}{\pr{C}}
			= \frac{\pr{\lim_{n\to\infty} A_n}}{\pr{C}}
			= \frac{\pr{C}}{\pr{C}}
			= 1.
			\label{eq-2}
		\end{eqnarray}

		\item
		Similarly define the following decreasing sequence of events
		\[
			B_n = \{X \leq -n \} \cap C.
		\]
		Then
		\[
			\lim_{n\to\infty} B_n
			= \bigcap_{n=1}^\infty B_n
			= \bigcap_{n=1}^\infty \left( \{X \leq -n \} \cap C \right)
			= \left( \bigcap_{n=1}^\infty \{X \leq -n \} \right) \cap C
			= \emptyset \cap C
			= \emptyset.
		\]
		Thus (\ref{eq-cont-prob-fcn}) implies
		\begin{eqnarray}
		\nonumber
			\lefteqn{
			\lim_{x\to-\infty} \ccdfxk{x}{C}
			= \lim_{n\to-\infty} \ccdfxk{n}{C}
			}
			\\
			&=& \frac{\lim_{n\to\infty} \pr{\{X\leq -n\}\cap C}}{\pr{C}}
			= \frac{\lim_{n\to\infty} \pr{B_n}}{\pr{C}}
			= \frac{\pr{\lim_{n\to\infty} B_n}}{\pr{C}}
			= \frac{\pr{\emptyset}}{\pr{C}}
			= 0.
			\label{eq-3}
		\end{eqnarray}

		\item
		For any $a$ and $b$ such that $a<b$,
		$\{X\leq a\} \subseteq \{X\leq b\}$,
		hence
		\[
			\left( \{X\leq a\} \cap C \right) \subseteq
			\left( \{X\leq b\} \cap C \right).
		\]
		Thus $ \pr{ \{X\leq a\} \cap C } \leq \pr{ \{X\leq b\} \cap C }$
		and
		\[
		\ccdfxk{a}{C}
		=\frac{\pr{ \{X\leq a\} \cap C }}{\pr{C}} \leq
		\frac{ \pr{ \{X\leq b\} \cap C }}{\pr{C}}
		=\ccdfxk{b}{C}.
		\]
		Therefore
		\begin{equation}
		\label{eq-4}
		\ccdfxk{x}{C} \mbox{ is a nondecreasing function of } x.
		\end{equation}

		\item
		Now for some $x\in\reals$,
		we consider the following decreasing sequence of events,
		\[
			D_n = \{X\leq x + 1/n \} \cap C.
		\]
		Then
		\[
			\bigcap_{n=1}^\infty D_n
			= \left( \bigcap_{n=1}^\infty \{ X \leq x + 1/n \} \right) \cap C
			= \{X\leq x\} \cap C.
		\]
		Thus (\ref{eq-cont-prob-fcn}) implies
		\begin{eqnarray}
			\nonumber
			\lefteqn{
			\lim_{h\to+0} \ccdfxk{x+h}{C}
			= \lim_{n\to\infty} \ccdfxk{x+1/n}{C}
			= \frac{ \lim_{n\to\infty} \pr{ \{X \leq x + 1/n\} \cap C }}{\pr{C}}
			}
			\\
			&=& \frac{ \lim_{n\to\infty} \pr{ D_n }}{\pr{C}}
			= \frac{ \pr{ \lim_{n\to\infty} D_n }}{\pr{C}}
			= \frac{ \pr{ \{X\leq x\} \cap C }}{\pr{C}}
			= \ccdfxk{x}{C}
			\label{eq-5}
		\end{eqnarray}
		\ie,
		\ccdfxk{x}{C}\
		is continuous from the right.


		\item
		Again suppose that $a<b$.
		Then $\{x\leq b\} = \{x\leq a\} \cup  \{ a < x \leq b\}$.
		The distributivity implies
		\[
			\{x\leq b\} \cap C
			= \underbrace{\{x\leq a\} \cap C}_A
			\cup  \underbrace{\{ a < x \leq b\} \cap C}_B
		\]
		Since $A$ and $B$ are disjoint, \ie, $A\cap B = \emptyset$,
		\[
			\pr{\{x\leq b\} \cap C}
			= \pr{\{x\leq a\} \cap C}
			+  \pr{\{ a < x \leq b\} \cap C},
		\]
		thus
		\begin{eqnarray}
		\nonumber
			\lefteqn{
			\cpr{a<x\leq b}{C}
			= \frac{\pr{\{a<x\leq b\}\cap C }}{\pr{C}}
			}
			\\
			&=&
			 \frac{\pr{\{x\leq b\} \cap C} - \pr{\{x\leq a\} \cap C}}{\pr{C}}
			= \ccdfxk{b}{C} - \ccdfxk{a}{C}.
			\label{eq-6}
		\end{eqnarray}

		\item For any $x\in\reals$,
		(\ref{eq-6}) implies
		\begin{eqnarray*}
			\lefteqn{
			\lim_{x\to+0} (\ccdfxk{x}{C} - \ccdfxk{x-h}{C})
			= \lim_{n\to\infty} (\ccdfxk{x}{C} - \ccdfxk{x-1/n}{C})
			}
			\\
			&=& \lim_{n\to\infty} \cpr{x-1/n < X \leq x}{C}
			= \frac{ \lim_{n\to\infty}
				\pr{\{x-1/n < X \leq x\}\cap C}}{\pr{C}}
			\\&=& \frac{ \pr{
			\lim_{n\to\infty} \{x-1/n < X \leq x\}\cap C}}{\pr{C}}
			= \frac{ \pr{ \{X=x\} \cap C}}{\pr{C}}
			= \cpr{X=x}{C},
		\end{eqnarray*}
		hence
		\begin{equation}
		\label{eq-7}
			\cpr{X=x}{C} = \ccdfxk{x}{C} - \ccdfxk{x^-}{C}.
		\end{equation}

		\item Lastly,
		$\{ X \leq x \} \cup \{X > x\} = \sspace$,
		thus
		\[
			(\{ X \leq x \} \cap C)
			\cup
			(\{X > x\} \cap C) = \sspace \cap C = C.
		\]
		Since $\{ X \leq x \} \cap C$ and $\{X > x\} \cap C$ are disjoint,
		\[
			\pr{\{ X \leq x \} \cap C}
			+
			\pr{\{X > x\} \cap C} = \pr{C},
		\]
		and dividing the both side with \pr{C}\ gives
		\[
			\ccdfxk{x}{C} + \cpr{X>x}{C} = 1,
		\]
		hence
		\begin{equation}
		\label{eq-8}
			\cpr{X>x}{C} = 1 - \ccdfxk{x}{C}.
		\end{equation}
	\end{enumerate}
	\renewcommand{\labelenumii}{(\alpha{enumii})}

	Therefore \ccdfxk{x}{C}\ satisfies the eight properties of a CDF,
	(\ref{eq-1}),
	(\ref{eq-2}),
	(\ref{eq-3}),
	(\ref{eq-4}),
	(\ref{eq-5}),
	(\ref{eq-6}),
	(\ref{eq-7}),
	and (\ref{eq-8}).
	\end{solution}
	\fi


	\item \lgprob{4.46}.
	Find the mean and variance of the Gaussian random variable
	by direct integration of the following two equations, (\ref{eq-exp}) and (\ref{eq-var}).
	\begin{equation}
	\label{eq-exp}
		\Exp{X} = \intinftoinf x \pdfxk{x} \dx
	\end{equation}
	and
	\begin{equation}
	\label{eq-var}
		\VAR{X}
		= \Exp{(X-\Exp{X})^2}
		= \Exp{X^2} - \Exp{X}^2.
	\end{equation}
	\ifdefined\sol
	\begin{solution}
		The PDF of the Gaussian with $\mu$ and $\sigma>0$ as its mean and standard deviation
		is
		\begin{equation}
		\label{eq-pdf-gauss}
			\pdfxk{x} = \frac{1}{\sqrt{2\pi} \sigma} e^{-(x-\mu)^2/2\sigma^2}.
		\end{equation}
		If we substitute (\ref{eq-pdf-gauss}) for \pdfxk{x}\ in (\ref{eq-exp})
		and substitute $\mu + \sigma y$ for $x$,
		we have
		\begin{eqnarray*}
			\lefteqn{
				\Exp{X} = \frac{1}{\sqrt{2\pi} \sigma} \intinftoinf x e^{-(x-\mu)^2/2\sigma^2} \dx
				= \frac{1}{\sqrt{2\pi} \sigma } \intinftoinf (\mu + \sigma y) e^{-y^2/2} \sigma\dy
			}
			\\&=&
			\mu \underbrace{\intinftoinf \frac{1}{\sqrt{2\pi} } e^{-y^2/2} \dy}_A
			+ \sigma \underbrace{\intinftoinf \frac{1}{\sqrt{2\pi} } y e^{-y^2/2} \dy}_B
			= \mu
		\end{eqnarray*}
		where $dx = \sigma dy$, $A=1$ since it is the integral from $-\infty$ to $\infty$ of
		the PDF of the Gaussian with zero mean and unit variance,
		and $B=0$ since the integrand is symmetric about $0$.
		If we substitute (\ref{eq-pdf-gauss}) for \pdfxk{x}\ in (\ref{eq-var})
		and substitute $\mu + \sigma y$ for $x$,
		we have
		\begin{eqnarray*}
			\lefteqn{
			\VAR{X} = \Exp{(X-\Exp{X})^2}
			= \intinftoinf \frac{1}{\sqrt{2\pi} \sigma} (x-\mu)^2 e^{-(x-\mu)^2/2\sigma^2} \dx
			}
			\\&=&
			\intinftoinf \frac{1}{\sqrt{2\pi} \sigma} \sigma ^2 y^2 e^{-y^2/2} \sigma \dy
			= \sigma^ 2 \underbrace{\intinftoinf \frac{1}{\sqrt{2\pi}} y^2 e^{-y^2/2} \dy}_C
			= \sigma^2
		\end{eqnarray*}
		where $C=1$ since it is the variance of the Gaussian with zero mean and unit variance.
	\end{solution}
	\fi

	\item \lgprob{4.47}.
	Prove the following two formula, (\ref{eq-m-1}) and (\ref{eq-m-2}).
	\begin{equation}
	\label{eq-m-1}
		\Exp{X} = \int_0^\infty (1-\cdfxk{x}) \dx
	\end{equation}
	if \X\ is a continuous and nonnegative \randvar\
	and
	\begin{equation}
	\label{eq-m-2}
		\Exp{X} = \sumkztoi \pr{X>k}
	\end{equation}
	if \X\ is a nonnegative and integer-valued \randvar.
	\ifdefined\sol
	\begin{solution}
	\begin{enumerate}
		\item Suppose that \X\ is a continuous and nonnegative \randvar\
		and $\Exp{X}$ exists.
		Let $f:\preals \to \preals$ be a function such that
		\[
			f(y) = \int_0^y (1-\cdfxk{x}) \dx.
		\]
		Then
		\begin{eqnarray*}
			\lefteqn{
			f(y) = \int_0^y \left(1-\int_0^x \pdfxk{t} \dt \right) \dx
			= \int_0^y \dx - \int_0^y \int_0^x \pdfxk{t} \dt \dx
			}
			\\
			&=& y - \int_0^y \int_t^y \pdfxk{t} \dx \dt
			= y - \int_0^y \pdfxk{t} \int_t^y \dx \dt
			\\
			&=& y - \int_0^y \pdfxk{t} (y-t) \dt
			= y - y \int_0^y \pdfxk{t} \dt + \int_0^y t \pdfxk{t} \dt
			\\
			&=& y \left( 1 -  \int_0^y \pdfxk{t} \dt \right) + \int_0^y t \pdfxk{t} \dt
			\\
			&=& y \int_y^\infty \pdfxk{t} \dt + \int_0^y t \pdfxk{t} \dt
			\\
			&\leq& \int_y^\infty t \pdfxk{t} \dt + \int_0^y t \pdfxk{t} \dt
			= \int_0^\infty t \pdfxk{t} \dt = \Exp{X}
		\end{eqnarray*}
		by the existence of \Exp{X}.
		Thus we have
		\[
			\int_0^y t \pdfxk{t} \dt \leq f(y) \leq \Exp{X},
		\]
		hence
		\[
			\Exp{X} = \lim_{y\to\infty} \int_0^y t \pdfxk{t} \dt
			\leq \lim_{y\to\infty} f(y) \leq \Exp{X}.
		\]
		Therefore
		\[
			\int_0^\infty (1-\cdfxk{x}) \dx
			= \lim_{y\to\infty} f(y) = \Exp{X},
		\]
		hence the proof.

		\item Suppose that \X\ is a nonnegative and integer-valued \randvar\
		and $\Exp{X}$ exists.
		Let $f:\naturals\to\preals$ be a function such that
		\[
			f(n) = \sumkzton \pr{X>k}.
		\]
		Then
		\begin{eqnarray*}
			\lefteqn{
			f(n) = \sumkzton \sumto{i}{k+1}{\infty} \pmfxk{i}
			= \sumto{i}{1}{\infty} \sumto{k}{0}{\min\{i-1,n\}} \pmfxk{i}
			= \sumto{i}{1}{\infty} \pmfxk{i} \sumto{k}{0}{\min\{i-1,n\}} 1
			}
			\\&=&
			\sumto{i}{1}{n} i \pmfxk{i}
			+ (n+1) \sumto{i}{n+1}{\infty} \pmfxk{i}
			\leq \sumto{i}{1}{n} i \pmfxk{i}
			+ \sumto{i}{n+1}{\infty} i \pmfxk{i}
			= \Exp{X}
		\end{eqnarray*}
		by the existence of \Exp{X}.
		Therefore
		\[
			\sumto{i}{1}{n} i \pmfxk{i} \leq f(n) \leq \Exp{X}.
		\]
		Since $\lim_{n\to\infty} \sumto{i}{1}{n} i \pmfxk{i} = \Exp{X}$,
		\[
			\Exp{X} = \lim_{n\to\infty} \sumiton \pmfxk{i}
			\leq \lim_{n\to\infty} f(n)
			\leq \Exp{X}.
		\]
		Therefore
		\[
			\sumkztoi \pr{X>k} = \lim_{n\to\infty} f(n) = \Exp{X},
		\]
		hence the proof.


	\end{enumerate}
	\end{solution}
	\fi



\end{enumerate}




\end{document}

