\section{Real-Time Anomaly Detection}

\subsection{Computing Anomaly Likelihood}

At every time step, we evaluate the moving sample means and the moving sample standard deviations:
\begin{eqnarray}
\mu(t) &=& \frac{1}{w} \sum_{i=0}^{w-1} s(t-i)
\\
\sigma(t) &=& \sqrt{\frac{1}{w-1} \sum_{i=0}^{w-1} (s(t-i)-\mu(t))^2}
\end{eqnarray}
Then compute a recent short term average of raw anomaly scores,
and apply a threshold to the Gaussian tail probability (Q-function)
to decided whether or not to declare an anomaly:
\begin{equation}
L(t) = 1 - Q\left(\frac{\tilde{\mu}(t) - \mu(t)}{\sigma(t)}\right)
\end{equation}
where the short-term moving sample mean is defined by
\begin{equation}
\tilde{\mu}(t) = \frac{1}{w'} \sum_{0}^{w'-1} s(t-i)
\end{equation}
and the Q-function is defined by
\begin{equation}
Q(x) = \frac{1}{\sqrt{2\pi}} \int_x^\infty \exp\left(-\frac{\tau^2}{2}\right)  d \tau.
\end{equation}

We threshold $L(t)$ and report an anomaly if it is very close to $1$:
\begin{equation}
L(t) > 1 - \epsilon.
\end{equation}



To take longer history data into considering while simutaneously emphasizing recent data more,
we can consider the exponentially weighted moving sample mean and standard deviation:

\begin{eqnarray}
\mu(t) &=& (1-\gamma) \sum_{i=0}^{\infty} \gamma^i s(t-i)
\\
\sigma(t) &=& \sqrt{(1-\gamma) \sum_{i=0}^{\infty} \gamma^i (s(t-i)-\mu(t))^2}
\end{eqnarray}

To combine the advantages of finite window method and exponentially weighted method,
we can consider the exponentially weighted moving sample mean and standard deviation with finite window
as follows:
\begin{eqnarray}
\mu(t) &=& \frac{1}{\sum_{i=0}^{w-1} \gamma^i }\sum_{i=0}^{w-1} \gamma^i s(t-i)
\\
\sigma(t) &=& \sqrt{\frac{1}{\sum_{i=0}^{w-1} \gamma^i }\sum_{i=0}^{w-1} \gamma^i (s(t-i)-\mu(t))^2}
\end{eqnarray}
