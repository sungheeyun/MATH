\documentclass{article}

\input{../../../FrequentlyUsed/latex/mydefs}
\input{../../../FrequentlyUsed/latex/mypackages}

\title{Can an existence or entity (or living creature) in a continuous universe count?}
\author{Sunghee Yun}

\begin{document}
\maketitle

In this document,
I will show how one can define the real numbers without using the concept of natural numbers.
Then I will argue that an entity in a continuous universe, \eg, without fingers,
could possess a perception of natural numbers.

\section{What are real numbers? How can we define them?}

One approach is to define them as \href{https://en.wikipedia.org/wiki/Dedekind_cut}{Dedekind cuts}
of rational numbers, which in turn area defined by integers.
This gives an elegant construction of the real numbers from more primitive concepts and set theory.

Another way of doing this is to consider them as already given
together a set of axioms for them.
In fact, \emph{these axioms completely characterize the real numbers}.
We assume as given the set $\reals$ of real numbers,
the set $\ppreals$ of positive real numbers,
and the functions $+$ and $\cdot$ on $\reals\times\reals$ to $\reals$
and assume that these satisfy the following axioms,
which we list in three groups.

The first group describes the algebraic properties
and the second the order properties.
The third comprises the least upper bound axiom.

\paragraph{A. The Field Axioms}
For all $x,y,z \in\reals$ we have
\begin{description}
\item[A1] $x + y = y + x$. (additive commutativity)
\item[A2] $(x + y) + z = x + (y + z)$. (additive associativity)
\item[A3] There exists $0 \in\reals$ such that $x+0=x$ for all $x\in\reals$. (additive identity)
\item[A4] For each $x\in\reals$ there exists a $w \in \reals$ such that $x+w=0$.
\item[A5] $xy=yx$. (multiplicative commutativity)
\item[A6] $(xy)z=x(yz)$. (multiplicative associativity)
\item[A7] There exists $1 \in\reals$ such that $1\neq0$ and $x\cdot1=x$ for all $x\in\reals$. (multiplicative identity)
\item[A8] For each $x\in\reals$ different from $0$ there exists a $w \in \reals$ such that $xw=1$.
\item[A9] $x(y+z) = xy+xz$. (distributivity)
\end{description}

Any set that satisfies these axioms is called a \emph{field} (under $+$ and $\cdot$).
The $0$ in {\bf Axiom A3} is unique.
The $w$ in {\bf Axiom A4} is unique and denoted by $-x$.
We defined subtraction $x-y$ as $x+(-y)$.
The $1$ in {\bf Axiom A7} is also unique.
The $w$ in {\bf Axiom A8} is unique and denoted by $x^{-1}$.
(By the way, if we have any field,
\ie,
any system satisfying {\bf Axiom A1} through {\bf Axiom A9},
we can perform all the operations of elementary algebra,
including the solution of a linear system.)


The second class of properties possessed by the real numbers
have to do with the fact that the real numbers are ordered.
When we use the notation of a positive real number
as the primitive one,
we have the following second group of axioms.

\paragraph{B. Axioms of Order}
The subset $\ppreals$ of positive real numbers satisfies the following:

\begin{description}
\item[B1] $(x,y\in\ppreals) \Rightarrow x + y \in\ppreals$.
\item[B2] $(x,y\in\ppreals) \Rightarrow x y \in\ppreals$.
\item[B3] $(x\in\ppreals) \Rightarrow -x \not\in\ppreals$.
\item[B4] $(x\in\reals) \Rightarrow (x=0) \mbox{ or } (x\in\ppreals) \mbox{ or } (-x\in\ppreals)$
\end{description}

Any system satisfying the axioms of groups {\bf Axiom A} and {\bf Axiom B} is called an \emph{ordered field}.
Thus the real numbers are an ordered field.
(The rational numbers give another example of an ordered field.)

In an ordered field we defined the notion $x<y$ to mean $y-x\in\ppreals$.
We write $x\leq y$ for $x<y$ or $x=y$.
Since {\bf Axiom B1} implies that the relation $<$ is \emph{transitive}
and {\bf Axiom B4} implies that it is \emph{antisymmetric},
the real numbers are \emph{linearly ordered} by $<$.

The third axiom group consists of a single axiom,
and it is this axiom that \emph{distinguishes the real numbers from other ordered fields}.
However, as we shall see in subsequent sections, we do not need this axiom to prove our argument.

\paragraph{C. Completeness Axiom}
Every nonempty set $S$ of real numbers
which has an upper bound has a least upper bound.


\section{Principle of recursive definition}

We take for granted the principle of recursive definition
without proof.

\begin{principle}
[Principle of recursive definition]
Let $f$ be a function from a set $X$ to itself,
and let $a$ be an element of $X$.
Then there is a unique infinite sequence $\{x_i\}_{i=1}^\infty$ from $X$
such that $x_1 = a$ and $x_{i+1} = f(x_i)$ for each $i\in \naturals$.
\end{principle}


\section{How can we define in a continuous universe?}

We generally adopt the procedure of taking the natural numbers
for granted and of using them as counting numbers.
However, \emph{an entity living in a continuous universe
may not be capable of counting}.
Here I show that how an entity in such a universe
\emph{could} perceive the set of natural numbers
as a subset of real numbers
and \emph{recognize the concept of counting without discrete objects such as fingers}.
Since we have just shown that we can define the real numbers without natural numbers' help,
this stands as a valid argument!


Note here that we use the symbol $1$ not only to denote the first natural number,
but also the special real number given by {\bf Axiom A7}.
One is tempted to define the real number $3$ as $1 + 1 + 1$,
and, in a ``similar'' fashion,
we can define real numbers corresponding to any natural number.

However, we can use a tool to do this in a more precise fashion.
By the principle of recursive definition
there is a function $\varphi(1) = 1$
and $\varphi(n+1) = \varphi(n) + 1$.
\emph{Here $1$ denotes a real number on the right side and a natural number on the left.}
We show that the mapping $\varphi$ is a one-to-one mapping of $\naturals$ into $\reals$.

Let $p$ and $q$ be two different natural numbers with $p<q$.
Then $q=p+n$ with $n=q-p$,
and we shall show that $\varphi(p) < \varphi(q)$
by induction on $n$.
For $n=1$ we haev $q=p+1$ and $\varphi(q) = \varphi(p) + 1 > \varphi(p)$.
For general $n$ we have $\varphi(p+n+1) = \varphi(p+n) + 1 > \varphi(p+n) > \varphi(p)$,
thus by induction $\varphi(p+n)>\varphi(p)$,
and we see that the mapping $\varphi$ is one-to-one.
We can also prove by induction that $\varphi(p+q) = \varphi(p) + \varphi(q)$ and $\varphi(pq) = \varphi(p) \varphi(q)$.
Therefore $\varphi$ provides a one-to-one correspondence between the natural numbers and a subset of $\reals$,
and $\varphi$ preserves sums, products, and the relation $<$.
Strictly speaking, we should distinguish between the natural number $n$
and its image $\varphi(n)$ under $\varphi$.

If we do not make the distinction,
we can consider the set $\naturals$ of \emph{natural numbers} to be a subset of $\reals$.
By taking differences of natural numbers,
we obtain the \emph{integers} as a subset of $\reals$.
Taking quotients of integers provides us the \emph{rationals}.

Since {\bf Axiom C} was not used in this discussion,
\emph{the same results hold for any ordered field}.
Thus we have shown the following:

\begin{proposition}
Every ordered field contains (sets isomorphic to)
the natural numbers,
the integers,
and the rational numbers.
\end{proposition}






\end{document}

