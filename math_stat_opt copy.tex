\documentclass[10pt, twoside]{book}   	% use "amsart" instead of "article" for AMSLaTeX format

\newcommand{\theoremsandsuchchapter}{}
% theorems and such

\ifdefined\theoremsandsuchchapter
\newtheorem{definition}{Definition}[chapter]
\newtheorem{theorem}{Theorem}[chapter]
\newtheorem{lemma}{Lemma}[chapter]
\newtheorem{corollary}{Corollary}[chapter]
\newtheorem{conjecture}{Conjecture}[chapter]
\else
\newtheorem{definition}{Definition}
\newtheorem{theorem}{Theorem}
\newtheorem{lemma}{Lemma}
\newtheorem{corollary}{Corollary}
\newtheorem{conjecture}{Conjecture}
\fi

\newcommand{\definename}{Definition}
\newcommand{\theoremname}{Theorem}
\newcommand{\lemmaname}{Lemma}
\newcommand{\corollaryname}{Corollary}

\newenvironment{proof}{\begin{quote}\textit{Proof}:}{\end{quote}}
\newenvironment{solution}{\begin{quote}\textit{Solution}:}{\end{quote}}
\newenvironment{pcode}{\begin{quote}\textit{Python source code}:}{\end{quote}}
\newenvironment{names}{\begin{itshape}}{\end{itshape}}

\newcommand{\qed}{{\bf Q.E.D.}}
\renewcommand{\qed}{\rule[-.5ex]{.5em}{2ex}}
\newcommand{\textfn}[1]{\textsl{#1}}

% table

\newcommand{\tparbox}[2]{%
{\parbox[c]{#1}{\center\vspace{-.4\baselineskip}{#2}\vspace{.3\baselineskip}}}}

% smile
\renewcommand{\smile}{\ensuremath{^\wedge\&^\wedge}}

% softwares

\newenvironment{code}{\begin{quote}\begin{tt}}{\end{tt}\end{quote}}


% math


\newcommand{\bmyeq}{\[}
\newcommand{\emyeq}{\]}
\newcommand{\bmyeql}[1]{\begin{equation}\label{#1}}
\newcommand{\emyeql}{\end{equation}}
%\newenvironment{myeq}{\[}{\]}
%\newenvironment{myeql}[1]{\begin{equation}\label{#1}}{\end{equation}}

\newcommand{\onehalf}{\ensuremath{\frac{1}{2}}}
\newcommand{\onethird}{\ensuremath{\frac{1}{3}}}
\newcommand{\onefourth}{\frac{1}{4}}
\newcommand{\sumft}[2]{\sum_{#1}^{#2}}
\newcommand{\sumioneton}{\sumionetok{n}}
\newcommand{\sumionetok}[1]{\sum_{i=1}^#1}
\newcommand{\sumoneto}[2]{\sum_{#1=1}^{#2}}
\newcommand{\sumoneton}[1]{\sumoneto{#1}{n}}
\newcommand{\prodoneto}[2]{\prod_{#1=1}^{#2}}
\newcommand{\prodoneton}[1]{\prodoneto{#1}{n}}

\newcommand{\listoneto}[1]{\ensuremath{1,2,\ldots,#1}}
\newcommand{\diagxoneto}[2]{\ensuremath{\diag({#1}_1,{#1}_2,\ldots,{#1}_{#2})}}
\newcommand{\setxoneto}[2]{\ensuremath{\{\listxoneto{#1}{#2}\}}}
\newcommand{\listxoneto}[2]{\ensuremath{{#1}_1,{#1}_2,\ldots,{#1}_{#2}}}
\newcommand{\setoneto}[1]{\ensuremath{\{1,2,\ldots,#1\}}}

\newcommand{\diagmat}[2]{\diagoneto{#1}{#2}}

\newcommand{\setoneton}[1]{\setoneton{#1}}
\newcommand{\setxtok}[2]{\setxoneto{#1}{#2}}



\newcommand{\colvectwo}[2]{\ensuremath{\begin{my-matrix}{c}{#1}\\{#2}\end{my-matrix}}}
\newcommand{\colvecthree}[3]{\ensuremath{\begin{my-matrix}{c}{#1}\\{#2}\\{#3}\end{my-matrix}}}
\newcommand{\colvecfour}[4]{\ensuremath{\begin{my-matrix}{c}{#1}\\{#2}\\{#3}\\{#4}\end{my-matrix}}}
\newcommand{\rowvectwo}[2]{\ensuremath{\begin{my-matrix}{cc}{#1}&{#2}\end{my-matrix}}}
\newcommand{\rowvecthree}[3]{\ensuremath{\begin{my-matrix}{ccc}{#1}&{#2}&{#3}\end{my-matrix}}}
\newcommand{\rowvecfour}[4]{\ensuremath{\begin{my-matrix}{cccc}{#1}&{#2}&{#3}&{#4}\end{my-matrix}}}
\newcommand{\diagtwo}[2]{\ensuremath{\begin{my-matrix}{cc}{#1}&0\\0& {#2}\end{my-matrix}}}
\newcommand{\mattwotwo}[4]{\ensuremath{\begin{my-matrix}{cc}{#1}&{#2}\\{#3}&{#4}\end{my-matrix}}}
\newcommand{\bigmat}[9]{\ensuremath{\begin{my-matrix}{cccc} #1&#2&\cdots&#3\\ #4&#5&\cdots&#6\\ \vdots&\vdots&\ddots&\vdots\\ #7&#8&\cdots&#9 \end{my-matrix}}}
%\newcommand{\matdotff}[9]{\matff{#1}{#2}{\cdots}{#3}{#4}{#5}{\cdots}{#6}{\vdots}{\vdots}{\ddots}{\vdots}{#7}{#8}{\cdots}{#9}}

\newcommand{\matthreethree}[9]{%
	\begin{my-matrix}{ccc}%
	{#1}&{#2}&{#3}%
	\\{#4}&{#5}&{#6}%
	\\{#7}&{#8}&{#9}%
	\end{my-matrix}%
}
\newcommand{\matthreethreeT}[9]{%
	\matthreethree%
	{#1}{#4}{#7}%
	{#2}{#5}{#8}%
	{#3}{#6}{#9}%
}

\newenvironment{my-matrix}{\left[\begin{array}}{\end{array}\right]}

\newcommand{\mbyn}[2]{\ensuremath{#1\times #2}}
\newcommand{\realmat}[2]{\ensuremath{\reals^{\mbyn{#1}{#2}}}}
\newcommand{\realsqmat}[1]{\ensuremath{\reals^{\mbyn{#1}{#1}}}}
\newcommand{\defequal}{\triangleq}

\newcommand{\dspace}{\,}
\newcommand{\dx}{{\dspace dx}}
\newcommand{\dy}{{\dspace dy}}
\newcommand{\dt}{{\dspace dt}}
\newcommand{\intspace}{\!\!}
\newcommand{\sqrtspace}{\,}
\newcommand{\aftersqrtspace}{\sqrtspace}
\newcommand{\dividespace}{\!}

\newcommand{\one}{\mathbf{1}}
\newcommand{\arginf}{\mathop{\mathrm{arginf}}}
\newcommand{\argsup}{\mathop{\mathrm {argsup}}}
\newcommand{\argmin}{\mathop{\mathrm {argmin}}}
\newcommand{\argmax}{\mathop{\mathrm {argmax}}}

\newcommand{\reals}{{\mbox{\bf R}}}
\newcommand{\preals}{{\reals_+}}
\newcommand{\ppreals}{{\reals_{++}}}
\newcommand{\complexes}{{\mbox{\bf C}}}
\newcommand{\integers}{{\mbox{\bf Z}}}
\newcommand{\naturals}{{\mbox{\bf N}}}
\newcommand{\rationals}{{\mbox{\bf Q}}}

\newcommand{\realspace}[2]{\reals^{#1\times #2}}
\newcommand{\compspace}[2]{\complexes^{#1\times #2}}

\newcommand{\identity}{\mbox{\bf I}}
\newcommand{\nullspace}{{\mathcal N}}
\newcommand{\range}{{\mathcal R}}

\newcommand{\set}[2]{\{#1~|~#2\}}
\newcommand{\bigset}[2]{\left\{#1\left|#2\right.\right\}}


% operators

\newcommand{\Expect}{\mathop{\bf E{}}}
\newcommand{\Var}{\mathop{\bf  Var{}}}
\newcommand{\Cov}{\mathop{\bf Cov}}
\newcommand{\Prob}{\mathop{\bf Prob}}



% operators

\newcommand{\smallo}{{\mathop{\bf o}}}

\newcommand{\jac}{{\mathcal{J}}}
\newcommand{\diag}{\mathop{\bf diag}}
\newcommand{\Rank}{\mathop{\bf Rank}}
\newcommand{\rank}{\mathop{\bf rank}}
\newcommand{\dimn}{\mathop{\bf dim}}
\newcommand{\Tr}{\mathop{\bf Tr}} % trance
\newcommand{\dom}{\mathop{\bf dom}}
\newcommand{\Det}{\det}
\newcommand{\adj}{\mathop{\bf adj}}
%\newcommand{\Det}{{\mathop{\bf Det}}}
%\newcommand{\determinant}[1]{|#1|}
\newcommand{\sign}{{\mathop{\bf sign}}}
\newcommand{\dist}{{\mathop{\bf dist}}}




% probability space

\newcommand{\probsubset}{{\mathcal{P}}}
\newcommand{\eset}{{\mathcal{E}}}

\newcommand{\probspace}{{\Omega}}


% acronyms

\newcommand{\eg}{{\it e.g.}}
\newcommand{\ie}{{\it i.e.}}
\newcommand{\etc}{{\it etc.}}
\newcommand{\cf}{{\it cf.}}


% some commands & environments for making lecture notes

\newcounter{oursection}
\newcommand{\oursection}[1]{
 \addtocounter{oursection}{1}
 \setcounter{equation}{0}
 \clearpage \begin{center} {\Huge\bfseries #1} \end{center}
 {\vspace*{0.15cm} \hrule height.3mm} \bigskip
 \addcontentsline{toc}{section}{#1}
}
\newcommand{\oursectionf}[1]{  % for use with foiltex
 \addtocounter{oursection}{1}
 \setcounter{equation}{0}
 \foilhead[-.5cm]{#1 \vspace*{0.8cm} \hrule height.3mm }
 \LogoOn
}
\newcommand{\oursectionfl}[1]{  % for use with foiltex landscape
 \addtocounter{oursection}{1}
 \setcounter{equation}{0}
 \foilhead[-1.0cm]{#1}
 \LogoOn
}

\newcounter{lecture}
\newcommand{\lecture}[1]{
 \addtocounter{lecture}{1}
 \setcounter{equation}{0}
 \setcounter{page}{1}
 \renewcommand{\theequation}{\arabic{equation}}
 \renewcommand{\thepage}{\arabic{lecture} -- \arabic{page}}
 \raggedright \sffamily \LARGE
 \cleardoublepage\begin{center}
 {\Huge\bfseries Lecture \arabic{lecture} \bigskip \\ #1}\end{center}
 {\vspace*{0.15cm} \hrule height.3mm} \bigskip
 \addcontentsline{toc}{chapter}{\protect\numberline{\arabic{lecture}}{#1}}
 \pagestyle{myheadings}
 \markboth{Lecture \arabic{lecture}}{#1}
}
\newcommand{\lecturef}[1]{
 \addtocounter{lecture}{1}
 \setcounter{equation}{0}
 \setcounter{page}{1}
 \renewcommand{\theequation}{\arabic{equation}}
 \renewcommand{\thepage}{\arabic{lecture}--\arabic{page}}
 \parindent 0pt
 \MyLogo{#1}
 \rightfooter{\thepage}
 \leftheader{}
 \rightheader{}
 \LogoOff
 \begin{center}
 {\large\bfseries Lecture \arabic{lecture} \bigskip \\ #1}
 \end{center}
 {\vspace*{0.8cm} \hrule height.3mm}
 \bigskip
}
\newcommand{\lecturefl}[1]{   % use with foiltex landscape
 \addtocounter{lecture}{1}
 \setcounter{equation}{0}
 \setcounter{page}{1}
 \renewcommand{\theequation}{\arabic{equation}}
 \renewcommand{\thepage}{\arabic{lecture}--\arabic{page}}
 \addtolength{\topmargin}{-1.5cm}
 \raggedright
 \parindent 0pt
 \rightfooter{\thepage}
 \leftheader{}
 \rightheader{}
 \LogoOff
 \begin{center}
 {\Large \bfseries Lecture \arabic{lecture} \\*[\bigskipamount] {#1}}
 \end{center}
 \MyLogo{#1}
}


\usepackage{geometry}                		% See geometry.pdf to learn the layout options. There are lots.
\geometry{letterpaper}                   		% ... or a4paper or a5paper or ...
%\geometry{landscape}                		% Activate for rotated page geometry
%\usepackage[parfill]{parskip}    		% Activate to begin paragraphs with an empty line rather than an indent
\usepackage{graphicx}				% Use pdf, png, jpg, or eps§ with pdflatex; use eps in DVI mode
								% TeX will automatically convert eps --> pdf in pdflatex
\usepackage{amssymb}

\usepackage{hyperref}
\hypersetup{
    colorlinks=true,
    linkcolor=blue,
    filecolor=magenta,
    %urlcolor=cyan,
    urlcolor=blue,
}

%SetFonts

%SetFonts

\newcommand{\feasibleset}{\mathcal{F}}
\newcommand{\optsolset}{\mathcal{X}^\ast}
\newcommand{\grad}{\nabla}
\newcommand{\possemidefset}[1]{\mathcal{S}_+^{#1}}
\newcommand{\posdefset}[1]{\mathcal{S}_{++}^{#1}}
\newcommand{\covmat}[1]{{\Sigma}_{#1}}


\title{Mathematics, Statistics, Optimization, and Machine Learning}
\author{Sunghee Yun}
%\date{}							% Activate to display a given date or no date

\begin{document}
\maketitle

\tableofcontents

\newpage

\part{Mathematics}

\chapter{Calculus}

\section{Basics}

\begin{theorem}
[L'H\^opital's rule]
Let $f:\reals\to\reals$ and $f:\reals\to\reals$ be differentiable on an open interval $I\subseteq \reals$
except possibly at $c\in I$.
If $\lim_{x\to c} f(x) = \lim_{x\to c} g(x) = 0$ or $\pm \infty$,
$g'(x)=0$ for all $x\in I\backslash \{c\}$,
and $\lim_{x\to c} \frac{f'(x)}{g'(x)}$ exists,
then
\begin{equation}
\label{eq:lhopital-rule}
\lim_{x\to c} \frac{f(x)}{g(x)}
= \lim_{x\to c} \frac{f'(x)}{g'(x)}.
\end{equation}
\end{theorem}

\begin{definition}
[Taylor polynomial]
Let $n\in\integers$ be a positive integer and
let $f:\reals\to\reals$ be $n$ times differentiable at $a\in\reals$.
The $n$-th order Taylor polynomial is defined by
\begin{eqnarray}
T_{f,n}(x) &=& f(a) + f'(a)(x-a)
+ \frac{f''(a)}{2!}(x-a)^2
+ \cdots
+ \frac{f^{(n)}(a)}{n!}(x-a)^n
\nonumber
\\
&=&
\sum_{k=0}^n \frac{f^{(k)}(a)}{k!} (x-a)^k
\label{eq:taylor-poly}
\end{eqnarray}
\end{definition}


\begin{theorem}
[Taylor's theorem]
\label{theorem:taylor-peano}
Let $n\in\integers$ be a positive integer and
let $f:\reals\to\reals$ be $n$ times differentiable at $a\in\reals$.
Then there exists a function $h_n:\reals\to\reals$ such that
\begin{equation}
\label{eq:taylor-peano}
f(x) = T_{f,n}(x) + h_n(x) (x-a)^n
\end{equation}
and $\lim_{x\to a} h_n(x)=0$.
The remainder is called the Peano form of the remainder.
\end{theorem}

\begin{theorem}
[Taylor's theorem]
\label{theorem:taylor-lagrange}
Let $n\in\integers$ be a positive integer, $a,b\in\reals$,
and $I_o = (a,b) \cup (b,a)$ and $I_c = [a,b] \cup [b,a]$.
Let $f:\reals\to\reals$ be $n+1$ times differentiable on $I_o$
and $f^{(n)}$ is continuous on $I_c$.
Then for some $c \in I_o$,
\begin{equation}
\label{eq:taylor-lagrange}
f(b) = T_{f,n}(b) + \frac{f^{(n+1)}(c)}{(n+1)!}(b-a)^{n+1}.
\end{equation}
The remainder is called the Peano form of the remainder.
\end{theorem}



\section{Chain rule}

\begin{theorem}
\label{theorem:chain-rule}
Let $f:\reals\to\reals^n$ and $g:\reals^n\to\reals$ be differentiable.
Then $h:\reals\to\reals$ such that $h(t) = g(f(t))$ is also differentiable and
\[
h'(t) = \sum_{i=1}^n f_i'(t) \frac{\partial g}{\partial x_i} (f(t))
= \nabla^T g(f(t))^T D_f (t)
\]
for all $t\in\dom f$.
\end{theorem}

\begin{corollary}
\label{corollary:chain-rule-gen}
Let $f:\reals^n\to\reals^m$ and $g:\reals^m\to\reals^p$ be differentiable.
Then define a function $h:\reals^n\to\reals^p$ such that $h(x) = g(f(x))$ for all $x\in\dom f$.
Then $h$ is differentiable and
\begin{equation}
\label{eq:chain-rule-gen}
D h(x) = Dg(f(x)) Df(x)
\end{equation}
where
$D f:\reals^n\to\reals^{m\times n}$,
$D g:\reals^m\to\reals^{p\times m}$,
and $D f:\reals^n\to\reals^{p\times n}$
are the Jacobian matrix functions of $f$, $g$, and $h$ respectively.
\end{corollary}

\begin{corollary}
\label{corollary:dixu}
Let $f:\reals^n\to\reals$ be differentiable.
Then for some $A\in\reals^{n\times m}$ and $b\in\reals^n$,
define $g:\reals^m\to\reals$ such that $g(y) = f(Ay+b)$.
Then
\begin{equation}
\label{eq:dixu}
\nabla g(y) = A^T \nabla f(Ay+b).
\end{equation}
\end{corollary}


\begin{corollary}
\label{corollary:eicg}
Let $f:\reals^n\to\reals$ be twice differentiable.
Then for some $A\in\reals^{n\times m}$ and $b\in\reals^n$,
define $g:\reals^m\to\reals$ such that $g(y) = f(Ay+b)$.
Then
\begin{equation}
\label{eq:eicg}
\nabla^2 g(y) = A^T \nabla^2 f(Ay+b)A.
\end{equation}
\end{corollary}


\chapter{Convex analysis}

\section{Convex function}

A function $f:\reals^n\to\reals$ is a convex function if, for all $x,y \in \dom f$ and all $0\leq \lambda \leq 1$,
\begin{equation}
f(\lambda x + (1-\lambda)y)
\leq
\lambda f(x) + (1-\lambda)f(y).
\end{equation}

\begin{theorem}
\label{theorem:cvx-equiv-1d-fcn}
Let $f:\reals^n\to\reals$.
Then for some $x\in\dom f$ and $v\in\reals^n$,
define a function $g_{x,v}(t):\reals\to\reals$ such that $g_{x,v}(t) = f(x+tv)$
with the domain, $\set{t\in\reals}{x+tv\in \dom f}$.
Then $f$ is a convex function iff $g_{x,v}$ is a convex function for any $x\in\dom f$ and any $v\in\reals^n$.
\end{theorem}

\begin{proof}
Suppose that $f$ is a convex function.
Then for any $x\in\dom f$ and $v\in\reals^n$,
for any $s,t \in\set{t\in\reals}{x+tv\in \dom f}$ and any $\lambda\in\reals$ such that $0\leq \lambda \leq 1$,
\begin{eqnarray*}
\lefteqn{
g_{x,v}(\lambda s + (1-\lambda) t)
=f(x+(\lambda s + (1-\lambda) t)v)
}
\\
&=&
f(\lambda (x+sv) + (1-\lambda) (x+tv))
\\
&\leq&
\lambda f(x+sv) + (1-\lambda) f(x+tv)
= \lambda g_{x,v}(s) + (1-\lambda) g_{x,v}(t).
\end{eqnarray*}
Therefore $g_{x,v}$ is a convex function.

Now assume that
$g_{x,v}(t):\reals\to\reals$ is a convex function for any $x\in\dom f$ and $v\in\reals^n$.
Then for any $x,y \in \dom f$ and any $\lambda\in\reals$ such that $0\leq \lambda \leq 1$,
\begin{eqnarray*}
\lefteqn{
f((1-\lambda) x + \lambda y ) =f(x + \lambda (y-x) )
}
\\
&=&
g_{x,y-x}(\lambda)
= g_{x,y-x}((1-\lambda)\cdot 0 + \lambda \cdot 1)
\leq (1-\lambda) g_{x,y-x}(0) + \lambda g_{x,y-x}(1)
\\
&=&
(1-\lambda) f(x) + \lambda f(y),
\end{eqnarray*}
thus, $f$ is a convex function.
\end{proof}

\subsection{First order condition}

\begin{theorem}
\label{theorem:cvx-1st-order-cond-1}
If a function $f:\reals\to\reals$ is differentiable, it is a convex function iff, for all $x, y \in \dom f$,
\begin{equation}
\label{eq:cvx-1st-order-cond-1}
        f(y) \geq f(x) + f'(x) (y-x).
\end{equation}
\end{theorem}

\begin{proof}
Suppose that $f$ is a convex function.
Then assume that $y>x$. Let $h\in\reals$ be a positive number such that $h<y-x$. Then the definition of convexity implies that
\[
f(x+h) \leq (1-\lambda) f(x) + \lambda f(y)
\]
where $\lambda = h/(y-x)$ since
\[
(1-\lambda) x + \lambda y = x + \lambda (y-x) = x +h.
\]
Thus
\[
f(x+h) - f (x) \leq \lambda (f(y)-f(x)) = \frac{h}{y-x} (f(y)-f(x)),
\]
which implies
\[
f'(x) = \lim_{h\to0} \frac{f(x+h) - f (x)}{h} \leq \frac{f(y)-f(x)}{y-x}.
\]
Therefore
\[
f(y) - f(x) \geq f'(x)(y-x),
\]
hence (\ref{eq:cvx-1st-order-cond-1}) is true when $y>x$.

We can prove (\ref{eq:cvx-1st-order-cond-1}) is true when $y<x$ using the very same method.
Assume that $x>y$. Let $h\in\reals$ be a positive number such that $h<x-y$. Then the definition of convexity implies that
\[
f(x-h) \leq (1-\lambda) f(x) + \lambda f(y)
\]
where $\lambda = h/(x-y)$ since
\[
(1-\lambda) x + \lambda y = x + \lambda (y-x) = x -h.
\]
Thus
\[
f(x) - f (x-h) \geq \lambda (f(x)-f(y)) = \frac{h}{x-y} (f(x)-f(y)),
\]
which implies
\[
f'(x) = \lim_{h\to0} \frac{f(x) - f (x-h)}{h} \geq \frac{f(x)-f(y)}{x-y}  =\frac{f(y)-f(x)}{y-x}.
\]
Therefore
\[
f(y) - f(x) \geq f'(x)(y-x),
\]
hence (\ref{eq:cvx-1st-order-cond-1}) is true when $y<x$.
It is obvise that (\ref{eq:cvx-1st-order-cond-1}) is true when $y=x$.
Hence we have just proved that if $f:\reals\to\reals$ is a convex function, then (\ref{eq:cvx-1st-order-cond-1}) holds
for any $x,y\in\dom f$.

Now we prove the converse.
Suppose that (\ref{eq:cvx-1st-order-cond-1}) holds for any $x,y\in\dom f$.
Now let $x,y\in\dom f$ and $\lambda \in \reals$ suc that $0\leq \lambda \leq 1$.
Let $z=\lambda x + (1-\lambda) y$. Then (\ref{eq:cvx-1st-order-cond-1}) implies that
\begin{equation}
\label{eq:dkfj-1}
f(x) - f(z) \geq f'(z) (x-z) = (1-\lambda) f'(z) (x-y)
\end{equation}
and
\begin{equation}
\label{eq:dkfj-2}
f(y) - f(z) \geq f'(z) (y-z) = \lambda f'(z) (y-x)
\end{equation}
If we multiply $\lambda$ on both sides of (\ref{eq:dkfj-1}),
multiply $1-\lambda$ on both sides of (\ref{eq:dkfj-2}),
and add both sides, we have
\[
\lambda(f(x) - f(z)) + (1-\lambda) (f(y) - f(z))
\geq \lambda f(x) +(1-\lambda) f(y) - f(z) \geq 0,
\]
hence
\[
f(\lambda x + (1-\lambda) y)
\leq \lambda f(x) + (1-\lambda) f(y).
\]
Therefore $f$ is a convex function.
\end{proof}

\begin{corollary}
\label{corollary:cvx-deriv-non-decreasing}
Let $f:\reals\to\reals$ be differentiable. Then $f$ is a convex function iff the derivative of $f$ is a nondecreasing function.
\end{corollary}
\begin{proof}
Suppose that $f$ is a convex function.
Let $x,y\in\dom f$ such that $x<y$.
Then \theoremname~\ref{theorem:cvx-1st-order-cond-1} implies
\[
f(y) \geq f(x) + f'(x)(y-x)
\]
and
\[
f(x) \geq f(y) + f'(y)(x-y),
\]
thus
\[
f'(x) \leq \frac{f(y)-f(x)}{y-x}
= \frac{f(x)-f(y)}{x-y} \leq f'(y)
\]
since $y>x$.
Therefore $f'$ is a nondecreasing function.

Now we prove the converse. Suppose that $f'$ is a nondecreasing function.
Then if $y>x$, the mean value theorem implies that there exists some $z \in (x,y)$ such that
\[
\frac{f(y)-f(x)}{y-x} = f'(z).
\]
Since $f'$ is nondecreasing, we have
\[
f'(x) \leq \frac{f(y)-f(x)}{y-x} \leq f'(y),
\]
thus
\begin{equation}
\label{eq:cias-1}
f(y) \geq f(x) + f'(x)(y-x)
\end{equation}
and
\begin{equation}
\label{eq:cias-2}
f(y) \leq f(x) + f'(y)(y-x).
\end{equation}
Therefore (\ref{eq:cias-1}) implies that $f$ satisfies (\ref{eq:cvx-1st-order-cond-1}).
Now if $x>y$, (\ref{eq:cias-2}) implies that
\[
f(x) \leq f(y) + f'(x)(x-y)
\Leftrightarrow
f(y) \geq f(x) + f'(x)(y-x)
\]
which again implies that $f$ satisfies (\ref{eq:cvx-1st-order-cond-1}).
Therefore (\ref{theorem:cvx-1st-order-cond-1}) implies that $f$ is a convex function.
\end{proof}

\begin{corollary}
\label{corollary:cvx-1st-order-cond}
If a function $f:\reals^n\to\reals$ is differentiable, it is a convex function iff, for all $x,y \in \dom f$,
\begin{equation}
\label{eq:cvx-1st-order-cond}
        f(y) \geq f(x) + \nabla f(x)^T (y-x).
\end{equation}
\end{corollary}

\begin{proof}
Suppose that $f$ is a convex funciton.
Let $x,y\in\dom f$.
If we let $g_{x,y-x}:\reals\to\reals$ be a funciton such that $g_{x,y-x}(t) = f(x+t(y-x))$,
\theoremname~\ref{theorem:cvx-equiv-1d-fcn} implies $g_{x,y-x}$ is a convex function.
Therefore \theoremname~\ref{theorem:cvx-1st-order-cond-1} together with \corollaryname~\ref{corollary:dixu}
implies
\[
f(y) = g_{x,y-x}(1) \geq g_{x,y-x}(0) + g_{x,y-x}'(0) (1-0)
= f(x) + \nabla f(x) ^T (y-x)
\]
for any $x,y \in \dom f$.

Now suppose that (\ref{eq:cvx-1st-order-cond}) holds for any $x,y \in \dom f$.
Then \corollaryname~\ref{corollary:dixu} implies that,
for any $r,s\in\reals$ and $v\in\reals^n$ such that $r,s\in\set{t\in\reals}{x+tv\in\dom f}$,
\[
g_{x,v}(r) = f(x+rv) \geq f(x+sv) + (r-s) \nabla f(x+sv)^T v = g_{x,v}(s) + g_{x,v}'(s)(r-s).
\]
Thus \theoremname~\ref{theorem:cvx-1st-order-cond-1} implies
$g_{x,v}$ is a convex function for any $x\in\dom f$ and $v\in\reals^n$.
Therefore by \theoremname~\ref{theorem:cvx-equiv-1d-fcn}, $f$ is a convex function.
\end{proof}

\subsection{Second order condition}

\begin{theorem}
\label{theorem:cvx-2nd-order-cond-1}
If a function $f:\reals\to\reals$ is twice differentiable, it is a convex function iff, for all $x \in \dom f$,
\begin{equation}
\label{eq:cvx-2nd-order-cond-1}
        f''(x) \geq 0.
\end{equation}
\end{theorem}

\begin{proof}
Suppose that $f$ is a convex function.
Then \corollaryname~\ref{corollary:cvx-deriv-non-decreasing} implies that $f'$ is a nondecreasing function,
hence
\[
f''(x)
= \lim_{h\to0} \frac{f'(x+h) - f'(x)}{h}
= \lim_{h\to0^+} \frac{f'(x+h) - f'(x)}{h} \geq 0.
\]
Now if $f''(x)\geq0$ for all $x\in\dom f$, the mean value theorem implies that $f'$ is a nondecreasing function.
\end{proof}


\begin{theorem}
\label{theorem:cvx-2nd-order-cond}
If a function $f:\reals^n\to\reals$ is twice differentiable, it is a convex function iff, for all $x \in \dom f$,
\begin{equation}
\label{eq:cvx-2nd-order-cond}
        \nabla^2 f(x) \succeq 0.
\end{equation}
\end{theorem}

\begin{proof}
Suppose that $f$ is a convex function.
Then \theoremname~\ref{theorem:cvx-equiv-1d-fcn} implies that
for any $x\in\dom f$ and any $v\in\reals^n$,
the function $g_{x,v}:\reals\to\reals$ such that $g_{x,v}(t) = f(x+tv)$
is a convex function in $\set{t\in\reals}{x+tv \in \dom f}$.
Then \theoremname~\ref{theorem:cvx-2nd-order-cond-1} together with \corollaryname~\ref{corollary:eicg}
implies that
\[
v^T \nabla^2 f(x) v = g_{x,v}''(0) \geq0.
\]
Therefore $\nabla^2 f(x) \succeq 0$ for any $x \in \dom f$.

Now if $\nabla^2 f(x) \succeq 0$ for all $x\in\dom f$,
then
\corollaryname~\ref{corollary:eicg}
implies that
$g_{x,v}''(t) = v^T \nabla^2 f(x+tv) v \geq0$ for any $x \in \dom f$ and any $v\in\reals^n$.
Then \theoremname~\ref{theorem:cvx-2nd-order-cond-1} implies
$g_{x,v}$ is a convex function for any $x \in \dom f$ and any $v\in\reals^n$,
hence by \theoremname~\ref{theorem:cvx-equiv-1d-fcn},
$f$ is a convex function.
\end{proof}



\chapter{Linear Algebra}
\section{Eigenvalues}

\subsection{Basic definitions}

Given a square matrix $A\in\reals^{n\times n}$,
if there exist $\lambda \in \complexes$ and nonzero $v \in \complexes^n$ such that
\begin{equation}
        A v = \lambda v
\end{equation}
then $\lambda$ is called an eigenvalue of $A$ and $v$ is called an eigenvector associated with $\lambda$.

If there exist $n$ linearly independent eigenvectors, we have
\begin{equation}
A \begin{my-matrix}{ccc} v_1 & \cdots & v_n \end{my-matrix}
= \begin{my-matrix}{ccc} v_1 & \cdots & v_n \end{my-matrix} \diag(\lambda_1,\ldots,\lambda_n)
\end{equation}
or
\begin{equation}
\label{eq:v8dy}
A V = V \Lambda
\end{equation}
where
\begin{equation}
V = \begin{my-matrix}{ccc} v_1 & \cdots & v_n \end{my-matrix}
\in\complexes^{n\times n}
\end{equation}
and
\begin{equation}
\Lambda = \diag(\lambda_1,\ldots,\lambda_n)
= \begin{my-matrix}{cccc}
\lambda_1 & 0 & \cdots & 0
\\
0 & \lambda_2 & \cdots & 0
\\
\vdots & \vdots & \ddots & \vdots
\\
0 & 0 & \cdots & \lambda_n
\end{my-matrix}
\in\complexes^{n\times n}.
\end{equation}
In this case, $A$ is said to be diagonalizable.

Since $V$ is nonsingular, \ie, invertible, we can rewrite (\ref{eq:v8dy}) as
\begin{equation}
\label{eq:2}
A = V \Lambda V^{-1} \Leftrightarrow V^{-1} A V = \Lambda.
\end{equation}


\subsection{Symmetric matrices}

Given a symmetric matrix $A = A^T\in\reals^{n\times n}$,
all the eigenvalues are real and we can choose $n$ real orthonormal eigenvectors,
\ie,
we can find $n$ eigenvectors $v_1, \ldots, v_n\in\reals^n$
associated with $n$ eigenvectors, $\lambda_1, \ldots, \lambda_n \in \reals$
such that
\begin{equation}
    \|v_i\| = 1
\end{equation}
for $i=1,\ldots,n$
and
\begin{equation}
    v_i^T v_j = 0
\end{equation}
for $1\leq i\neq j\leq n$.
Thus, all symmetric matrices are diagonalizable.

Now (\ref{eq:2}) becomes
\begin{equation}
\label{eq:sym-eigen-decomp}
A = V \Lambda V^T \Leftrightarrow V^T A V = \Lambda
\end{equation}
since
\begin{equation}
V^T V = I_n
\end{equation}
where $I_n\in\reals^{n\times n}$ is the indentity matrix.
We can rewrite (\ref{eq:sym-eigen-decomp}) as
\begin{equation}
\label{eq:sym-eigen-decomp-1}
A =
\begin{my-matrix}{ccc} v_1 & \cdots & v_n \end{my-matrix}
\diag(\lambda_1,\ldots,\lambda_n)
\begin{my-matrix}{c} v_1^T \\ \vdots \\ v_n^T \end{my-matrix}
= \sum_{i=1}^n \lambda_i v_i v_i^T.
\end{equation}

\section{Positive definiteness}

\begin{itemize}

\item
A symmetric matrix $A=A^T\in\reals^{n\times n}$ is called positive semidefinite if for all $x\in\reals^n$,
\begin{equation}
x^T A x \geq 0.
\end{equation}

\item
A symmetric matrix $A=A^T\in\reals^{n\times n}$ is called positive definite if for all nonzero $x\in\reals^n$,
\begin{equation}
x^T A x > 0.
\end{equation}

\item The set of all the $n$-by-$n$ positive semidefinite matrices is (sometimes) denoted by $\possemidefset{n}$,
\ie,
\begin{equation}
\possemidefset{n} = \set{A=A^T\in\reals^{n\times n}}{x^T A x \geq 0 \mbox{ for all } x \in \reals^n}.
\end{equation}

\item The set of all the $n$-by-$n$ positive definite matrices is (sometimes) denoted by $\posdefset{n}$,
\ie,
\begin{equation}
\posdefset{n} = \set{A=A^T\in\reals^{n\times n}}{x^T A x > 0 \mbox{ for all nonzero } x \in \reals^n}.
\end{equation}

\item $A=A^T\in\reals^{n\times n}$ is positive semidefinite if and only if all the eigenvalues of $A$ are nonnegative.

\item $A=A^T\in\reals^{n\times n}$ is positive definite if and only if all the eigenvalues of $A$ are positive.

\begin{proof}
For symmetric $A=A^T$, there exist orthgonal $V\in\reals^{n\times n}$ and diagonal $\Lambda\in\reals^{n\times n}$
such that
\[
A = V \Lambda V^T = \sum_{i=1}^n \lambda_i v_i v_i^T,
\]
thus for any $x\in\reals^n$,
\[
x^T A x = x^T \left(\sum_{i=1}^n \lambda_i v_i v_i^T \right) x
= \sum_{i=1}^n \lambda_i x^T v_i v_i^T x
= \sum_{i=1}^n \lambda_i (v_i^T x)^2.
\]
Therefore if all $\lambda_i$ are nonnegative, $x^T A x\geq0$ for any $x\in\reals^n$, hence $A\in\possemidefset{n}$.
Now assume $A\in\possemidefset{n}$, but $\lambda_j < 0$ for some $j\in\{1,\ldots,n\}$.
Then
\begin{equation}
v_j^T A v_j
= \sum_{i=1}^n \lambda_i (v_i^T v_j)^2
= \sum_{i=1}^n \lambda_i \delta_{i,j}
= \lambda_j < 0
\end{equation}
since $v_1$, \ldots, $v_n$ are orthonormal
where
$\delta_{i,j}$ is the \href{https://en.wikipedia.org/wiki/Kronecker_delta}{Kronecker delta function},
hence $A\not \in \possemidefset{n}$.
Therefore if $A\in\possemidefset{n}$, all $\lambda_i$ are nonnegative.

Therefore $A\in\possemidefset{n}$ if and only if all $\lambda_i$ are nonnegative.

Now assume that all $\label_i$ are positive.
Then for all nonzero $x\in\reals^n$,
there exists $i\in\{1,\ldots,n\}$ such tat $v_i^Tx$
since if $v_i^Tx=0$ for all $i$, then
$V^T x = 0$, hence $x=0$ since $V^T$ is nonnsigular.
Therefore
\begin{equation}
x^T A x = \sum_{i=1}^n \lambda_i (v_i^T x)^2
\geq \lambda_j (v_j^T x)^2 > 0.
\end{equation}
Thus, $A\in\posdefset{n}$.

Now assume that $A\in\posdefset{n}$.
If $\lambda_j \leq 0$ for some $j\in\{1,\ldots,n\}$,
then
\begin{equation}
v_j^T A v_j
= \sum_{i=1}^n \lambda_i \delta_{i,j}
= \lambda_j \leq 0,
\end{equation}
hence $A\not\in\posdefset{n}$. Therefore if $A\in\posdefset{n}$, all $\lambda_i$ are positive.

Therefore $A\in\posdefset{n}$ if and only if all $\lambda_i$ are positive.

\end{proof}



\end{itemize}

\part{Optimization}
\chapter{Optimization}

\section{Mathematical optimization problem}

A mathematical optimization problem can be expressed as
\begin{equation}
\label{eq:opt-prob}
\begin{array}{ll}
\mbox{minimize} & f_0(x)
\\
\mbox{subject to} & f_i(x) \leq 0 \mbox{ for } i = 1, \ldots, m
\\
& h_i(x) = 0 \mbox{ for } i = 1, \ldots, p
\end{array}
\end{equation}
where
$x\in\reals^n$ is the optimization variable,
$f_0:\reals^n\to\reals$ is the objective function,
$f_i:\reals^n\to\reals$ for $i=1,\ldots,n$ are the inequality constraint functions,
and
$h_i:\reals^n\to\reals$ for $i=1,\ldots,p$ are the equality constraint functions.

The conditions, $f_i(x) \leq 0$ for $ i = 1, \ldots, m$, are called inequality constraints
and the conditions, $ h_i(x) = 0 $ for $ i = 1, \ldots, p$ are called equation constraints.

Note that this formulation covers pretty much every single-objective optimization problem.
For example,
consider the following optimization problem.

\begin{equation}
\begin{array}{ll}
\mbox{maximize} & f(x_1,x_2)
\\
\mbox{subject to} & x_1 \geq x_2
\\
& x_1 + x_2 = 2
\end{array}
\end{equation}
This problem can be cast into an equivalent problem as follows.
\begin{equation}
\begin{array}{ll}
\mbox{minimize} & -f(x_1,x_2)
\\
\mbox{subject to} & - x_1 + x_2 \leq 0
\\
& x_1 + x_2 - 2 = 0
\end{array}
\end{equation}


The feasible set for (\ref{eq:opt-prob}) is defined by the set of $x\in\reals^n$ which satisfies all the contraints.
Also, the optimal value for (\ref{eq:opt-prob}) is the infimum of $f_0(x)$ while $x$ is in the feasible set.
When the infimum is achievable, we define the optimal solution set as the set of all feasible $x$ achieving
the infimum value.
These are defined in mathematically rigorous terms below.

\begin{itemize}

\item
The feasible set for (\ref{eq:opt-prob}) is defined by
\begin{equation}
\feasibleset
=
\set{x\in \mathcal{D}}
{ f_i(x)\leq 0 \mbox{ for } i =0, \ldots, m,\ h_j(x) = 0 \mbox{ for } j = 1,\ldots,p}
\subseteq \reals^n
\end{equation}
where
\begin{equation}
\mathcal{D} = \left( \bigcap_{0\leq i\leq m} \dom f_i \right) \cap \left( \bigcap_{1\leq i\leq p} \dom h_i \right).
\end{equation}

\item
The optimal value for (\ref{eq:opt-prob}) is defined by
\begin{equation}
p^\ast = \inf_{x\in\feasibleset} f_0(x)
\end{equation}

We use the conventions that $p^\ast = -\infty$ if $f_0(x)$ is unbounded below for $x\in \feasibleset$
and that $p^\ast = \infty$ if $\feasibleset = \emptyset$.

\item
The optimal solution set for (\ref{eq:opt-prob}) is defined by
\begin{equation}
\optsolset = \set{x\in\feasibleset}{f_0(x) = p^\ast}.
\end{equation}


\end{itemize}


\section{Convex optimization problem}

A mathematical optimization problem is called a convex optimization problem
if the objective function and all the inequality constraint functions are convex functions
and all the equality constraint functions are affine functions.

Hence, a convex optimization problem can be expressed as
\begin{equation}
\label{eq:cvx-opt-prob}
\begin{array}{ll}
\mbox{minimize} & f_0(x)
\\
\mbox{subject to} & f_i(x) \leq 0 \mbox{ for } i = 1, \ldots, m
\\
& A x = b
\end{array}
\end{equation}
where
$x\in\reals^n$ is the optimization variable,
$f_i:\reals^n\to\reals$ for $i=0,\ldots,n$ are convex functions,
$h_i:\reals^n\to\reals$ for $i=1,\ldots,p$ are the equality constraint functions,
$A \in \reals^{p\times n}$, and $b\in\reals^p$.

A function, $f:\reals^n \to \reals$, is called a convex function if
$\dom f \subseteq \reals^n$ is a convex set
and
for all $x, y\in\dom f$
and all $0\leq \lambda \leq 1$,
\begin{equation}
    f( \lambda x + (1-\lambda) y) \leq
    \lambda f(x) + (1-\lambda) f(y)
\end{equation}
where $\dom f \subseteq \reals^n$ denotes the domain of $f$.

A convex optimization enjoys a number of nice theoretical and practical properties.

\begin{itemize}
\item A local minimum of a convex optimization problem is a global minimum,
\ie,
if for some $R>0$ and $x_0\in\feasibleset$, $\|x-x_0\|<R$ and $x\in\feasibleset$ imply $f_0(x_0) \leq f_0(x)$,
then $f_0(x_0) \leq f_0(x)$ for all $x\in\feasibleset$.
\begin{proof}
Assume that $x_0\in\feasibleset$ is a local minimum, \ie,
for some $R>0$, $\|x-x_0\|<R$ and $x\in\feasibleset$ imply $f_0(x_0) \leq f_0(x)$.

Now assume that $x_0$ is not a global minimum, \ie, there exists $y\in\feasibleset$
such that $y\neq x_0$ and $f_0(y) < f_0(x_0)$.
Then for $z = \lambda y + (1-\lambda) x_0$ with $\lambda = \min\{ R/\|y-x_0\|, 1\}/2$,
the convexity of $f_0$ implies
\begin{equation}
\label{eq:4}
f_0(z) \leq \lambda f_0(y) + (1-\lambda) f_0(x_0)
\end{equation}
since $0 < \lambda \leq 1/2 < 1$.
Furthermore
\begin{equation}
\|z - x_0\| = \lambda \|y-x_0\| \leq R/2,
\end{equation}
hence $f_0(z) \geq f_0(x_0)$, which together with (\ref{eq:4}) implies
\begin{equation}
f_0(x_0) \leq f_0(z)
\leq \lambda f_0(y) + (1-\lambda) f_0(x_0)
< \lambda f_0(x_0) + (1-\lambda) f_0(x_0)
= f_0(x_0),
\end{equation}
which is a contradiction.
Therefore there is no $y\in\feasibleset$ such that $y\neq x_0$ and $f_0(y) < f_0(x_0)$.
Therefore $x_0$ is a global minimum.
\end{proof}



\item For a unconstrained problem, \ie, the problem (\ref{eq:cvx-opt-prob}) with $m=p=0$, with differential objective function,
$x\in\dom f_0$ is an optimal solution if and only if $\grad f_0(x)= 0 \in \reals^n$.

\begin{proof}
The Taylor theorem implies that for any $x,y\in\dom f_0$,
\begin{equation}
\label{eq:second-order-taylor}
f_0(y) = f(x) + \grad f_0(x) ^T (y-x) + \frac{1}{2} (y-x)^T \grad^2 f_0(z) (y-x)
\end{equation}
for some $z$ on the line segment having $x$ and $y$ as its end points,
\ie, $z = \alpha x + (1-\alpha) y$ for some $0\leq \alpha \leq 1$.
Since $\grad^2 f(x) \succeq0$ for any $z \in \dom f_0$, we have
\begin{equation}
f_0(y) \geq f_0(x) + \grad f_0(x) ^T (y-x)
\end{equation}

Thus, if for some $x_0 \in \reals^n$, $\grad f_0(x_0) = 0$, for any $x\in\dom f_0$,
\begin{equation}
f_0(x) \geq f_0(x_0) + \grad f_0(x_0) ^T (x-x_0) = f_0(x_0),
\end{equation}

hence $x_0$ is an optimal solution.
Now assume that $x_0$ is an optimal solution, but $\grad f_0(x_0) \neq 0$.
Then for any $k>0$, if we let $x=x_0$ and $y = x_0 - k \grad f_0(x_0) $,
(\ref{eq:second-order-taylor}) becomes
\begin{eqnarray*}
\lefteqn{
f_0(y) = f(x_0) + \grad f_0(x_0) ^T (-k \grad f_0(x_0)) + \frac{k^2}{2} \grad f_0(x_0) ^T \grad^2 f_0(z) \grad f_0(x_0)
}
\\
&=&
f(x_0) - k \|\grad f_0(x_0)\|^2 + \frac{k^2}{2} \grad f_0(x_0) ^T \grad^2 f_0(z) \grad f_0(x_0)
\end{eqnarray*}
for all $y = x_0 - k \grad f_0(x_0) \in \dom f_0$.

Since for $k< 2 \|\grad f_0(x_0)\|^2 / \grad f_0(x_0) ^T \grad^2 f_0(z) \grad f_0(x_0)$,
$-k \|\grad f_0(x_0)\|^2 + \frac{k^2}{2} \grad f_0(x_0) ^T \grad^2 f_0(z) \grad f_0(x_0) < 0$,
thus
$f_0(y) < f(x_0)$,
hence the constradiction.
Therefore, if $x_0$ is an optimal solution for the unconstrained problem, $\grad f_0(x_0) = 0$.

\end{proof}
\end{itemize}


\newpage
\chapter{Portfolio optimization}

\section{Problem formulation}

Suppose that we have $n$ assets to invest on
and that the return of each asset per unit invest is modeled by random variables $R_i$ for $i=1,\ldots,n$.
Then we want to decide the amount of investment on each asset, $x_i\in\reals$ for $i=1,\ldots,n$,
so that it optimizes the overall investment (in certain senses).

For formulization, we use the following definitions.

\begin{itemize}
\item Define a vector random variable $R\in\reals^n$ such that
\begin{equation}
R = \begin{my-matrix}{c}
R_1
\\
\vdots
\\
R_n
\end{my-matrix}
\in\reals^n.
\end{equation}

\item Let $r\in\reals^n$ be the expected value of $R$,
\ie,
\begin{equation}
r
= \Expect(R)
= \begin{my-matrix}{c}
\Expect(R_1)
\\
\vdots
\\
\Expect(R_n)
\end{my-matrix}
= \begin{my-matrix}{c}
r_1
\\
\vdots
\\
r_n
\end{my-matrix}
\in\reals^n.
\end{equation}

\item Define a vector $x\in\reals$ which is an aggregate of the investments:
\begin{equation}
x = \begin{my-matrix}{c}
x_1
\\
\vdots
\\
x_n
\end{my-matrix}
\in\reals^n.
\end{equation}

\item Define a feasible set $\mathcal{X}\subseteq \reals^n$ for $x$.
For example, if we have a limit on the total investment,
\begin{equation}
\label{eq:cnst-on-cost}
\mathcal{X} = \set{x\in\reals^n}{\sum_{i=1}^n c_i x_i \leq c_\mathrm{max}},
\end{equation}
or if we have the minimum and maximum amount to invest for each asset,
we'd have
\begin{equation}
\label{eq:cnst-on-each-amount}
\mathcal{X} = \set{x\in\reals^n}{d_\mathrm{min} \leq x_i \leq d_\mathrm{max} \mbox{ for } i=1,\ldots,n}.
\end{equation}
Generally, we'd prefer $\mathcal{X}$ to be a convex set, \ie,
for any $x,y\in\mathcal{X}$ and $0\leq \lambda \leq 1$,
\begin{equation}
\lambda x + (1-\lambda) y \in \mathcal{X}.
\end{equation}

\end{itemize}

\subsection{A portfolio optimization problem}

A portfolio optimization problem can be formulized by
\begin{equation}
\label{eq:opt-port-prob}
\begin{array}{ll}
\mbox{maximize} & f(x) = \Expect(Z)
\\
\mbox{minimize} & g(x) = \Var(Z)
\\
\mbox{subject to} & x \in \mathcal{X}
\end{array}
\end{equation}
where the optimization variable is $x\in\reals^n$
and
\begin{equation}
Z = \sum_{i=1}^n x_i R_i = x^T R
\end{equation}
where $\Expect(\cdot)$ and $\Var(\cdot)$ refer to the expected value and the variance operators respectively.

This problem formulation tries to \emph{maximize the expected return}
while \emph{minimizing the variance or uncertainty or risk}, which generally makes sense.

(\ref{eq:cnst-on-cost})
(\ref{eq:cnst-on-each-amount})


Note that
\begin{equation}
\Expect(Z)
= \Expect(x^T R)
= \Expect \left(\sum_{i=1}^n x_i R_i \right)
= \sum_{i=1}^n x_i \Expect(R_i)
= \sum_{i=1}^n x_i r_i
= r^T x
\end{equation}
and
\begin{eqnarray*}
\lefteqn{
\Var(Z) = \Expect(Z-\Expect(Z))^2 = \Expect \left(x^TR-x^Tr \right)^2
}
\\
&=&
\Expect \left(x^T(R-r) \right)^2
= \Expect \left(x^T(R-r)(R-r)^T x \right)
\\
&=&
x^T \Expect(R-r)(R-r)^T x
=
x^T \covmat{R} x
\end{eqnarray*}
where $\covmat{R} = \Expect(R-r)(R-r)^T$ is the \href{https://en.wikipedia.org/wiki/Covariance_matrix}{covariance matrix} of $R$.
Note that $\covmat{R}\in\possemidefset{n}$
since for any $y\in\reals^n$,
\begin{equation}
y^T \covmat{R} y = y^T \Expect(R-r)(R-r)^T y = \Expect(x^T(R-r))^2 \geq 0.
\end{equation}



Thus, (\ref{eq:opt-port-prob}) can be rewritten as
\begin{equation}
\label{eq:opt-port-prob-vec}
\begin{array}{ll}
\mbox{maximize} & f(x) = r^T x
\\
\mbox{minimize} & g(x) = x^T \covmat{R} x
\\
\mbox{subject to} & x \in \mathcal{X}.
\end{array}
\end{equation}



\part{Statistics}

\chapter{Statistics Basics}

\section{Correlation coefficients}

The correlation coefficients of two random variables, $X$ and $Y$, is defined by

\begin{equation}
\rho_{X,Y} = \frac{\Expect (X-\mu_X)(Y-\mu_Y)} {\sqrt{\Expect (X-\mu_X)^2 \Expect(Y-\mu_Y)^2}}
\end{equation}


\section{Probability density function transformation}

Assume two random variables, $X\in\reals$ and $Y\in\reals$, which are related by
a smooth function $g:\reals \to \reals$
such that
\begin{equation}
\label{eq:g8cx}
Y = g(X).\footnote{We assume that $g\in C^1$.}
\end{equation}

If $f_X:\reals\to\preals$ is the probability density function (PDF), what would be the PDF of $Y$?

First note that we should have
\begin{equation}
        f_X(x) \Delta x \sim f_Y(y) \Delta y
\end{equation}
for $x$ and $y$ such that $y=g(x)$
where $g(x)$ is increasing at $x$.

Suppose that $g$ is an injective function (or one-to-one function).
Then for every $y \in g(\mathcal{D})$ where $\mathcal{D}$ is the domain of $g$,
if we make $\Delta x$ or $\Delta y$ infinitesimally small,
(\ref{eq:g8cx}) becomes
\begin{equation}
\label{eq:pdf-transform}
f_Y(y)
= \lim_{\Delta x \to 0}\frac{\Delta x}{\Delta y} f_X(x)
= \frac{1}{\lim_{\Delta x \to 0} \frac{\Delta y}{\Delta x}} f_X(x)
= \frac{1}{g'(x)} f_X(x).
\end{equation}

This equation can be generalized for any smooth function $g$
and any $y\in g(\mathcal{D})$,
\begin{equation}
f_Y(y) = \sum_{g(x)=y} \frac{1}{|g'(x)|} f_X(x)
\end{equation}

\section{Log-normal distribution}

 We say $Y$ is log-normally distributed, if, for $X\sim\mathcal{N}(\mu_X,\sigma_X^2)$,
 \begin{equation}
 Y = \exp(X).
 \end{equation}

Then (\ref{eq:pdf-transform}) implies that
\begin{eqnarray}
f_Y(y) &=& \frac{1}{\exp(\log(y))} \cdot \frac{1}{\sqrt{2\pi} \sigma_X}  \exp\left(-\frac{(\log(y)-\mu_X)^2}{2\sigma_X^2}\right)
\nonumber
\\
&=& \frac{1}{\sqrt{2\pi} \sigma_X y}  \exp\left(-\frac{(\log(y)-\mu_X)^2}{2\sigma_X^2}\right).
\label{eq:log-normal-pdf}
\end{eqnarray}


\subsection{Some statistics}

The definition of the expected value implies
\begin{eqnarray*}
\lefteqn{
\Expect Y = \int_{0}^{\infty} y f_Y(y) \, dy
= \int_{0}^{\infty} \frac{1}{\sqrt{2\pi} \sigma_X }  \exp\left(-\frac{(\log(y)-\mu_X)^2}{2\sigma_X^2}\right) dy
}
\\
&=&
\int_{-\infty}^{\infty} \frac{1}{\sqrt{2\pi} \sigma_X }  \exp\left(-\frac{(x-\mu_X)^2}{2\sigma_X^2}\right) \exp(x) \, dx
\\
&=&
\int_{-\infty}^{\infty} \frac{1}{\sqrt{2\pi} \sigma_X }  \exp\left(-\frac{(x-\mu_X)^2}{2\sigma_X^2}+x\right) dx
\\
&=&
\int_{-\infty}^{\infty} \frac{1}{\sqrt{2\pi} \sigma_X }  \exp\left(-\frac{x^2 - 2(\mu_X +\sigma_X^2)x + \mu_X^2}{2\sigma_X^2}\right) dx
\\
&=&
\int_{-\infty}^{\infty} \frac{1}{\sqrt{2\pi} \sigma_X }  \exp\left(-\frac{(x - (\mu_X +\sigma_X^2))^2 + \mu_X^2 -(\mu_X+\sigma_X^2)^2}{2\sigma_X^2}\right) dx
\\
&=&
\int_{-\infty}^{\infty} \frac{1}{\sqrt{2\pi} \sigma_X }  \exp\left(-\frac{(x - (\mu_X +\sigma_X^2))^2 - 2\mu_X\sigma_X^2 - \sigma_X^4}{2\sigma_X^2}\right) dx
\\
&=&
\int_{-\infty}^{\infty} \frac{1}{\sqrt{2\pi} \sigma_X }  \exp\left(-\frac{(x - (\mu_X +\sigma_X^2))^2 - \sigma_X^2(2\mu_X+ \sigma_X^2)}{2\sigma_X^2}\right) dx
\\
&=&
\exp\left(\frac{2\mu_X+ \sigma_X^2}{2}\right)
\int_{-\infty}^{\infty} \frac{1}{\sqrt{2\pi} \sigma_X }  \exp\left(-\frac{(x - (\mu_X +\sigma_X^2))^2 }{2\sigma_X^2}\right) dx
\\
&=&
\exp\left(\frac{2\mu_X+ \sigma_X^2}{2}\right)
\end{eqnarray*}
since $dy = \exp(x) dx$
and $\frac{1}{\sqrt{2\pi} \sigma_X }  \exp\left(-\frac{(x - (\mu_X +\sigma_X^2))^2 }{2\sigma_X^2}\right)$
is the PDF of a random variable $\sim$ $\mathcal{N}(\mu_X+\sigma_X^2,\sigma_X^2)$,
thus
\begin{equation}
\label{eq:log-normal-mean}
\mu_Y
= \Expect Y = \exp\left(\frac{2\mu_X+ \sigma_X^2}{2}\right).
\end{equation}

Similarly,
\begin{eqnarray*}
\lefteqn{
\Expect Y^2 = \int_{0}^{\infty} y^2 f_Y(y) \, dy
= \int_{0}^{\infty} \frac{1}{\sqrt{2\pi} \sigma_X }  y \exp\left(-\frac{(\log(y)-\mu_X)^2}{2\sigma_X^2}\right) dy
}
\\
&=&
\int_{-\infty}^{\infty} \frac{1}{\sqrt{2\pi} \sigma_X }  \exp(x) \exp\left(-\frac{(x-\mu_X)^2}{2\sigma_X^2}\right) \exp(x) \, dx
\\
&=&
\int_{-\infty}^{\infty} \frac{1}{\sqrt{2\pi} \sigma_X }  \exp\left(-\frac{(x-\mu_X)^2}{2\sigma_X^2}+2x\right) dx
\\
&=&
\int_{-\infty}^{\infty} \frac{1}{\sqrt{2\pi} \sigma_X }  \exp\left(-\frac{x^2 - 2(\mu_X +2\sigma_X^2)x + \mu_X^2}{2\sigma_X^2}\right) dx
\\
&=&
\int_{-\infty}^{\infty} \frac{1}{\sqrt{2\pi} \sigma_X }  \exp\left(-\frac{(x - (\mu_X +2\sigma_X^2))^2 + \mu_X^2 -(\mu_X+2\sigma_X^2)^2}{2\sigma_X^2}\right) dx
\\
&=&
\int_{-\infty}^{\infty} \frac{1}{\sqrt{2\pi} \sigma_X }  \exp\left(-\frac{(x - (\mu_X +2\sigma_X^2))^2 - 4\mu_X\sigma_X^2 - 4\sigma_X^4}{2\sigma_X^2}\right) dx
\\
&=&
\int_{-\infty}^{\infty} \frac{1}{\sqrt{2\pi} \sigma_X }  \exp\left(-\frac{(x - (\mu_X +2\sigma_X^2))^2 - 4\sigma_X^2(\mu_X+ \sigma_X^2)}{2\sigma_X^2}\right) dx
\\
&=&
\exp\left({2(\mu_X+ \sigma_X^2)}\right)
\int_{-\infty}^{\infty} \frac{1}{\sqrt{2\pi} \sigma_X }  \exp\left(-\frac{(x - (\mu_X +2\sigma_X^2))^2 }{2\sigma_X^2}\right) dx
\\
&=&
\exp\left({2(\mu_X+ \sigma_X^2)}\right),
\end{eqnarray*}
thus
\begin{equation}
\label{eq:log-normal-var}
\sigma_Y^2 =
\Var(Y) = \Expect Y^2 - (\Expect Y)^2 = \exp(2(\mu_X+\sigma_X^2)) - \exp(2\mu_X+\sigma_X^2)
= (\exp(\sigma_X^2)-1) \exp(2\mu_X+\sigma_X^2)).
\end{equation}

Note that (\ref{eq:log-normal-mean}) implies that
\begin{equation}
\mu_Y^2
= \exp(2\mu_X+\sigma_X^2)),
\end{equation}
hence
\begin{equation}
\sigma_Y^2
= (\exp(\sigma_X^2)-1) \mu_Y^2.
\end{equation}
This multiplicative dependency of the standard deviation on the expected value
is attributed to the fact that $\log(Y) \sim \mathcal{N}(\mu_X,\sigma_X^2)$,
\ie,
the log-scale of $Y$ follows the normal distribution.

Now if we differentiate the PDF with respect to $y$,
(\ref{eq:log-normal-pdf}) implies that
\begin{eqnarray*}
\lefteqn{
\frac{d}{dy} f_Y(y)
= \frac{d}{dy} \left(\frac{1}{\sqrt{2\pi} \sigma_X y}  \exp\left(-\frac{(\log(y)-\mu_X)^2}{2\sigma_X^2}\right)\right)
}
\\
&=&
-\frac{1}{\sqrt{2\pi} \sigma_X y^2}  \exp\left(-\frac{(\log(y)-\mu_X)^2}{2\sigma_X^2}\right)
\\
&&
+
\frac{1}{\sqrt{2\pi} \sigma_X y}  \exp\left(-\frac{(\log(y)-\mu_X)^2}{2\sigma_X^2}\right)
\left(-\frac{(\log(y)-\mu_X)}{\sigma_X^2}\right)
\frac{1}{y}
\\
&=&
-\frac{1}{\sqrt{2\pi} \sigma_X y^2}  \exp\left(-\frac{(\log(y)-\mu_X)^2}{2\sigma_X^2}\right)
\left(1+\frac{(\log(y)-\mu_X)}{\sigma_X^2} \right)
\\
&=&
-\frac{1}{\sqrt{2\pi} \sigma_X^3 y^2}  \exp\left(-\frac{(\log(y)-\mu_X)^2}{2\sigma_X^2}\right)
\left(\log(y)-(\mu_X-\sigma_X^2) \right).
\end{eqnarray*}
Equating the derivative to zero yields
\begin{equation}
y = \exp(\mu_X-\sigma_X^2),
\end{equation}
which is the mode of $Y$.


\subsection{Parameter estimation}

Now assume that we have a log-normally distributed random variable, $Y\in\ppreals$,
with $\mu_Y$ and $\sigma_Y^2$ as its mean and variance.
We derived the parameters of the source distribution, $\mu_X$ and $\sigma_X$.

The two equations, (\ref{eq:log-normal-mean}) and (\ref{eq:log-normal-var}), imply
\begin{eqnarray*}
\mu_Y &=& \exp(\mu_X+\sigma_X^2/2),
\\
\sigma_Y^2 &=& (\exp(\sigma_X^2)-1)\exp(2\mu_X+\sigma_X^2) = (\exp(\sigma_X^2)-1) \mu_Y^2,
\end{eqnarray*}
thus
\begin{eqnarray*}
\sigma_X^2 &=& \log(1+{\sigma_Y^2}/{\mu_Y^2}),
\\
\mu_X &=& \log(\mu_Y) - \sigma_X^2/2 = \log(\mu_Y) - \log(1+{\sigma_Y^2}/{\mu_Y^2})/2
= \frac{1}{2} \log\left(\frac{\mu_Y^2}{1+{\sigma_Y^2}/{\mu_Y^2}}\right).
\end{eqnarray*}


\chapter{Information Theory}

\section{Basics}


\subsection{Entropy}

\subsection{Mutual Information}

The mutual information (MI) is defined by

\begin{equation}
I(X;Y) = \Expect \log \frac{f_{X,Y}(X,Y)}{f_{X}(X)f_{Y}(Y)}
= \int_{-\infty}^{\infty} \int_{-\infty}^{\infty} \log \frac{f_{X,Y}(x,y)}{f_{X}(x)f_{Y}(y)} \, dx dy
\end{equation}





\part{Machine Learning}

\chapter{Bayesian Network}


\begin{equation}
\Pr\{X_1,\ldots,X_n\}
= \prod_{i=1}^n \Pr\set{X_i}{X_1,\ldots,X_{i-1}}
= \prod_{i=1}^n \Pr\set{X_i}{\mathbf{parent}(X_i)}
\end{equation}



\end{document}
