\section{Vector space}

Cauchy–Schwarz inequality: for $a,b\in\complexes^n$,
\begin{equation}
|a^H b| \leq \|a\|_2 \|b\|_2.
\end{equation}

The generalized form:
for $f:[0,1] \to \complexes$ and $g:[0,1] \to \complexes$,
\begin{equation}
\left|\int_{0}^1 \overline{f(t)} g(t) \, dt \right|
\leq
\left(\int_0^1 |f(t)|^2 dt \, \right)^{1/2}
\left(\int_0^1 |g(t)|^2 dt \, \right)^{1/2}.
\end{equation}

H\"older's inequality: for $a,b\in\complexes^n$, $p>1$, and $q>1$ such that $1/p+1/q=1$,
\begin{equation}
|a^H b| \leq \|a\|_p \|b\|_q.
\end{equation}

The generalized form:
for $f:[0,1] \to \complexes$, $g:[0,1] \to \complexes$,
$p>1$, and $q>1$ such that $1/p+1/q=1$,
\begin{equation}
\left|\int_{0}^1 \overline{f(t)} g(t) \, dt \right|
\leq
\left(\int_0^1 |f(t)|^p dt \, \right)^{1/p}
\left(\int_0^1 |g(t)|^q dt \, \right)^{1/q}.
\end{equation}

When $b=\ones$, the Cauchy-Schwarz inequality implies
\begin{equation}
\|a\|_1 \leq n \|a\|_2
\end{equation}



\section{Eigenvalues}

\subsection{Basic definitions}

Given a square matrix $A\in\reals^{n\times n}$,
if there exist $\lambda \in \complexes$ and nonzero $v \in \complexes^n$ such that
\begin{equation}
        A v = \lambda v
\end{equation}
then $\lambda$ is called an eigenvalue of $A$ and $v$ is called an eigenvector associated with $\lambda$.

If there exist $n$ linearly independent eigenvectors, we have
\begin{equation}
A \begin{my-matrix}{ccc} v_1 & \cdots & v_n \end{my-matrix}
= \begin{my-matrix}{ccc} v_1 & \cdots & v_n \end{my-matrix} \diag(\lambda_1,\ldots,\lambda_n)
\end{equation}
or
\begin{equation}
\label{eq:v8dy}
A V = V \Lambda
\end{equation}
where
\begin{equation}
V = \begin{my-matrix}{ccc} v_1 & \cdots & v_n \end{my-matrix}
\in\complexes^{n\times n}
\end{equation}
and
\begin{equation}
\Lambda = \diag(\lambda_1,\ldots,\lambda_n)
= \begin{my-matrix}{cccc}
\lambda_1 & 0 & \cdots & 0
\\
0 & \lambda_2 & \cdots & 0
\\
\vdots & \vdots & \ddots & \vdots
\\
0 & 0 & \cdots & \lambda_n
\end{my-matrix}
\in\complexes^{n\times n}.
\end{equation}
In this case, $A$ is said to be diagonalizable.

Since $V$ is nonsingular, \ie, invertible, we can rewrite (\ref{eq:v8dy}) as
\begin{equation}
\label{eq:2}
A = V \Lambda V^{-1} \Leftrightarrow V^{-1} A V = \Lambda.
\end{equation}


\subsection{Symmetric matrices}

Given a symmetric matrix $A = A^T\in\reals^{n\times n}$,
all the eigenvalues are real and we can choose $n$ real orthonormal eigenvectors,
\ie,
we can find $n$ eigenvectors $v_1, \ldots, v_n\in\reals^n$
associated with $n$ eigenvectors, $\lambda_1, \ldots, \lambda_n \in \reals$
such that
\begin{equation}
    \|v_i\| = 1
\end{equation}
for $i=1,\ldots,n$
and
\begin{equation}
    v_i^T v_j = 0
\end{equation}
for $1\leq i\neq j\leq n$.
Thus, all symmetric matrices are diagonalizable.

Now (\ref{eq:2}) becomes
\begin{equation}
\label{eq:sym-eigen-decomp}
A = V \Lambda V^T \Leftrightarrow V^T A V = \Lambda
\end{equation}
since
\begin{equation}
V^T V = I_n
\end{equation}
where $I_n\in\reals^{n\times n}$ is the indentity matrix.
We can rewrite (\ref{eq:sym-eigen-decomp}) as
\begin{equation}
\label{eq:sym-eigen-decomp-1}
A =
\begin{my-matrix}{ccc} v_1 & \cdots & v_n \end{my-matrix}
\diag(\lambda_1,\ldots,\lambda_n)
\begin{my-matrix}{c} v_1^T \\ \vdots \\ v_n^T \end{my-matrix}
= \sum_{i=1}^n \lambda_i v_i v_i^T.
\end{equation}

\section{Positive definiteness}

\begin{itemize}

\item
A symmetric matrix $A=A^T\in\reals^{n\times n}$ is called positive semidefinite if for all $x\in\reals^n$,
\begin{equation}
x^T A x \geq 0.
\end{equation}

\item
A symmetric matrix $A=A^T\in\reals^{n\times n}$ is called positive definite if for all nonzero $x\in\reals^n$,
\begin{equation}
x^T A x > 0.
\end{equation}

\item The set of all the $n$-by-$n$ positive semidefinite matrices is (sometimes) denoted by $\possemidefset{n}$,
\ie,
\begin{equation}
\possemidefset{n} = \set{A=A^T\in\reals^{n\times n}}{x^T A x \geq 0 \mbox{ for all } x \in \reals^n}.
\end{equation}

\item The set of all the $n$-by-$n$ positive definite matrices is (sometimes) denoted by $\posdefset{n}$,
\ie,
\begin{equation}
\posdefset{n} = \set{A=A^T\in\reals^{n\times n}}{x^T A x > 0 \mbox{ for all nonzero } x \in \reals^n}.
\end{equation}

\item $A=A^T\in\reals^{n\times n}$ is positive semidefinite if and only if all the eigenvalues of $A$ are nonnegative.

\item $A=A^T\in\reals^{n\times n}$ is positive definite if and only if all the eigenvalues of $A$ are positive.

\begin{proof}
For symmetric $A=A^T$, there exist orthgonal $V\in\reals^{n\times n}$ and diagonal $\Lambda\in\reals^{n\times n}$
such that
\[
A = V \Lambda V^T = \sum_{i=1}^n \lambda_i v_i v_i^T,
\]
thus for any $x\in\reals^n$,
\[
x^T A x = x^T \left(\sum_{i=1}^n \lambda_i v_i v_i^T \right) x
= \sum_{i=1}^n \lambda_i x^T v_i v_i^T x
= \sum_{i=1}^n \lambda_i (v_i^T x)^2.
\]
Therefore if all $\lambda_i$ are nonnegative, $x^T A x\geq0$ for any $x\in\reals^n$, hence $A\in\possemidefset{n}$.
Now assume $A\in\possemidefset{n}$, but $\lambda_j < 0$ for some $j\in\{1,\ldots,n\}$.
Then
\begin{equation}
v_j^T A v_j
= \sum_{i=1}^n \lambda_i (v_i^T v_j)^2
= \sum_{i=1}^n \lambda_i \delta_{i,j}
= \lambda_j < 0
\end{equation}
since $v_1$, \ldots, $v_n$ are orthonormal
where
$\delta_{i,j}$ is the \href{https://en.wikipedia.org/wiki/Kronecker_delta}{Kronecker delta function},
hence $A\not \in \possemidefset{n}$.
Therefore if $A\in\possemidefset{n}$, all $\lambda_i$ are nonnegative.

Therefore $A\in\possemidefset{n}$ if and only if all $\lambda_i$ are nonnegative.

Now assume that all $\label_i$ are positive.
Then for all nonzero $x\in\reals^n$,
there exists $i\in\{1,\ldots,n\}$ such tat $v_i^Tx$
since if $v_i^Tx=0$ for all $i$, then
$V^T x = 0$, hence $x=0$ since $V^T$ is nonnsigular.
Therefore
\begin{equation}
x^T A x = \sum_{i=1}^n \lambda_i (v_i^T x)^2
\geq \lambda_j (v_j^T x)^2 > 0.
\end{equation}
Thus, $A\in\posdefset{n}$.

Now assume that $A\in\posdefset{n}$.
If $\lambda_j \leq 0$ for some $j\in\{1,\ldots,n\}$,
then
\begin{equation}
v_j^T A v_j
= \sum_{i=1}^n \lambda_i \delta_{i,j}
= \lambda_j \leq 0,
\end{equation}
hence $A\not\in\posdefset{n}$. Therefore if $A\in\posdefset{n}$, all $\lambda_i$ are positive.

Therefore $A\in\posdefset{n}$ if and only if all $\lambda_i$ are positive.

\end{proof}

\end{itemize}

\section{Matrix norms}

\begin{equation}
\mathbf{dist}( C_\mathrm{org 1}, C_\mathrm{org 2} ) = 
\|C_\mathrm{org 1} - C_\mathrm{org 2} \|
= |\lambda_\mathrm{max}(C_\mathrm{org 1} - C_\mathrm{org 2})|
\end{equation}
