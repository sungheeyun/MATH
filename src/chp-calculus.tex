
\section{Basics}

\begin{theorem}
[L'H\^opital's rule]
Let $f:\reals\to\reals$ and $f:\reals\to\reals$ be differentiable on an open interval $I\subseteq \reals$
except possibly at $c\in I$.
If $\lim_{x\to c} f(x) = \lim_{x\to c} g(x) = 0$ or $\pm \infty$,
$g'(x)=0$ for all $x\in I\backslash \{c\}$,
and $\lim_{x\to c} \frac{f'(x)}{g'(x)}$ exists,
then
\begin{equation}
\label{eq:lhopital-rule}
\lim_{x\to c} \frac{f(x)}{g(x)}
= \lim_{x\to c} \frac{f'(x)}{g'(x)}.
\end{equation}
\end{theorem}

\begin{definition}
[Taylor polynomial]
Let $n\in\integers$ be a positive integer and
let $f:\reals\to\reals$ be $n$ times differentiable at $a\in\reals$.
The $n$-th order Taylor polynomial is defined by
\begin{eqnarray}
T_{f,n}(x) &=& f(a) + f'(a)(x-a)
+ \frac{f''(a)}{2!}(x-a)^2
+ \cdots
+ \frac{f^{(n)}(a)}{n!}(x-a)^n
\nonumber
\\
&=&
\sum_{k=0}^n \frac{f^{(k)}(a)}{k!} (x-a)^k
\label{eq:taylor-poly}
\end{eqnarray}
\end{definition}


\begin{theorem}
[Taylor's theorem]
\label{theorem:taylor-peano}
Let $n\in\integers$ be a positive integer and
let $f:\reals\to\reals$ be $n$ times differentiable at $a\in\reals$.
Then there exists a function $h_n:\reals\to\reals$ such that
\begin{equation}
\label{eq:taylor-peano}
f(x) = T_{f,n}(x) + h_n(x) (x-a)^n
\end{equation}
and $\lim_{x\to a} h_n(x)=0$.
The remainder is called the Peano form of the remainder.
\end{theorem}

\begin{theorem}
[Taylor's theorem]
\label{theorem:taylor-lagrange}
Let $n\in\integers$ be a positive integer, $a,b\in\reals$,
and $I_o = (a,b) \cup (b,a)$ and $I_c = [a,b] \cup [b,a]$.
Let $f:\reals\to\reals$ be $n+1$ times differentiable on $I_o$
and $f^{(n)}$ is continuous on $I_c$.
Then for some $c \in I_o$,
\begin{equation}
\label{eq:taylor-lagrange}
f(b) = T_{f,n}(b) + \frac{f^{(n+1)}(c)}{(n+1)!}(b-a)^{n+1}.
\end{equation}
The remainder is called the Peano form of the remainder.
\end{theorem}


\section{Multivariate functions}

\begin{definition}[Jacobian matrix]
Let $f:\reals^n \to \reals^m$ be a differentiable function,
\ie, the partial derivative $\partial f_j(x) / \partial x_i$ exists for every $1\leq i\leq n$ and $1\leq j\leq m$.
Then the Jacobian matrix of $f$ at $x$ is defined by the function $D_f:\reals^{n} \to \reals^{m \times n}$
such that
\begin{equation}
\label{eq:jacobian-matrix}
D_f(x) = \begin{my-matrix}{cccc}
\partial f_1(x) / \partial x_1
& \partial f_1(x) / \partial x_2
& \cdots
& \partial f_1(x) / \partial x_n
\\
\partial f_2(x) / \partial x_1
& \partial f_2(x) / \partial x_2
& \cdots
& \partial f_2(x) / \partial x_n
\\
\vdots & \vdots & \ddots & \vdots
\\
\partial f_m(x) / \partial x_1
& \partial f_m(x) / \partial x_2
& \cdots
& \partial f_m(x) / \partial x_n
\end{my-matrix}
\in\reals^{m \times n}.
\end{equation}
\end{definition}



\section{Chain rule}

\begin{theorem}
\label{theorem:chain-rule}
Let $f:\reals\to\reals^n$ and $g:\reals^n\to\reals$ be differentiable.
Then $h:\reals\to\reals$ such that $h(t) = g(f(t))$ is also differentiable and
\[
h'(t) = \sum_{i=1}^n f_i'(t) \frac{\partial g}{\partial x_i} (f(t))
= \nabla g(f(t))^T D_f (t)
\]
for all $t\in\dom f$.
\end{theorem}

\begin{corollary}
\label{corollary:chain-rule-gen}
Let $f:\reals^n\to\reals^m$ and $g:\reals^m\to\reals^p$ be differentiable.
Then define a function $h:\reals^n\to\reals^p$ such that $h(x) = g(f(x))$ for all $x\in\dom f$.
Then $h$ is differentiable and
\begin{equation}
\label{eq:chain-rule-gen}
D h(x) = Dg(f(x)) Df(x)
\end{equation}
where
$D f:\reals^n\to\reals^{m\times n}$,
$D g:\reals^m\to\reals^{p\times m}$,
and $D f:\reals^n\to\reals^{p\times n}$
are the Jacobian matrix functions of $f$, $g$, and $h$ respectively.
\end{corollary}

\begin{corollary}
\label{corollary:chain-rule-grad}
Let $f:\reals^n\to\reals^m$ and $g:\reals^m\to\reals$ be differentiable.
Then define a function $h:\reals^n\to\reals$ such that $h(x) = g(f(x))$ for all $x\in\dom f$.
Then $h$ is differentiable and
\begin{equation}
\label{eq:chain-rule-gen}
\nabla h(x) = Df(x)^T \nabla g(f(x))
\end{equation}
where
$D f:\reals^n\to\reals^{m\times n}$,
$D g:\reals^m\to\reals^{p\times m}$,
and $D f:\reals^n\to\reals^{p\times n}$
are the Jacobian matrix functions of $f$, $g$, and $h$ respectively.
\end{corollary}

\begin{corollary}
\label{corollary:dixu}
Let $f:\reals^n\to\reals$ be differentiable.
Then for some $A\in\reals^{n\times m}$ and $b\in\reals^n$,
define $g:\reals^m\to\reals$ such that $g(y) = f(Ay+b)$.
Then
\begin{equation}
\label{eq:dixu}
\nabla g(y) = A^T \nabla f(Ay+b).
\end{equation}
\end{corollary}


\begin{corollary}
\label{corollary:eicg}
Let $f:\reals^n\to\reals$ be twice differentiable.
Then for some $A\in\reals^{n\times m}$ and $b\in\reals^n$,
define $g:\reals^m\to\reals$ such that $g(y) = f(Ay+b)$.
Then
\begin{equation}
\label{eq:eicg}
\nabla^2 g(y) = A^T \nabla^2 f(Ay+b)A.
\end{equation}
\end{corollary}

\section{Integration}


\begin{lemma}
Let $A\in\reals^{n\times n}$ be a nonsingular matrix. Suppose that the following integral exists for some $C\subseteq \reals^n$.
\begin{equation}
\int_{C} f(x) dx
\end{equation}
\end{lemma}





