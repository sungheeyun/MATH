\newcommand{\numcus}{\ensuremath{{n_C}}}
\newcommand{\numitem}{\ensuremath{{n_I}}}
\newcommand{\numfeat}{\ensuremath{{n_F}}}
\newcommand{\numrating}{\ensuremath{{n_R}}}
\newcommand{\nan}{\ensuremath{\mathrm{NaN}}}
\newcommand{\ratmat}{\ensuremath{R}}
\newcommand{\ratvec}{r}
\newcommand{\ratmataggr}{\ensuremath{\bar{\ratmat}}}
\newcommand{\ratmatnorm}{\ensuremath{\tilde{\ratmat}}}
\newcommand{\ratmatpred}{\hat{\ratmat}}
\newcommand{\datsourceset}{\ensuremath{\mathcal{S}}}
\newcommand{\dir}{\mathrm{Dir}}
\newcommand{\simil}{s}
\newcommand{\bmscore}{\mathrm{bm}_{25}}
\newcommand{\avgcl}{\ensuremath{\bar{c}}}

\newcommand{\numfactors}{\ensuremath{{n_L}}}
\newcommand{\lfcn}{\ensuremath{l}}
\newcommand{\confvar}{c}
\newcommand{\confmat}{C}
\newcommand{\binvar}{p}
\newcommand{\binmat}{P}

\newcommand{\Xsvd}{\XYsvd{X}}
\newcommand{\Ysvd}{\XYsvd{Y}}
\newcommand{\XYsvd}[1]{{#1}_\mathrm{svd}}

\newcommand{\XY}[2]{{{#1}_{#2}}}
\newcommand{\XX}[1]{\XY{X}{#1}}
\newcommand{\YY}[1]{\XY{Y}{#1}}

\newcommand{\initX}{\tilde{X}}
\newcommand{\initY}{\tilde{Y}}

%\newcommand{\Xsim}{\ensuremath{X_\mathrm{sim}^\ast}}
%\newcommand{\Ysim}{\ensuremath{Y_\mathrm{sim}^\ast}}
\newcommand{\Xsim}{\ensuremath{X^\mathrm{sim}}}
\newcommand{\Ysim}{\ensuremath{Y^\mathrm{sim}}}

\newcommand{\popest}{{\hat{\theta}}}
\newcommand{\popestk}[1]{\ensuremath{\popest_{#1}}}


\newpage

\section{Item-based Collaborative Filtering}

The purpose of the item-based collaborative filtering is to recommend items using the information obatined from similar items.
Since the information from other items help provide better recommendation to customers,
it is called an \emph{item-based collaborative filtering}.

We formulate an item-based collaborative filtering as follows.
We assume that there are \numcus\ customers and \numitem\ items.
We further assume that we have partial information as to the taste of each customer for each item.
In typical cases,
we have customers can leave reviews with ratings for $n$ items where $n \ll \numitem$.
When we have hundreds of thousands of items, this makes the data structure extremely sparse.
This fact plays a critical role when we calculate similarities among items or customers,
or learn the matrix factorization-based collaborative filtering models,
e.g., using gradient descent (GD) method or alternating least squares (ALS) method.\footnote{
    These models will be discussed in later sections.
}

Now suppose that we have the rating matrix
\begin{equation}
\label{eq:rating-matrix}
\ratmat \in (\reals\cup\{\nan\})^{\numcus \times \numitem}
\end{equation}
where $\ratmat_{ij} \in (\reals\cup\{\nan\})$ denotes \emph{certain score} corresponding to $i$th customer and $j$th item
and \nan\ denotes we do not know the values.
Note that this {certain score} can refer to rating of movies, but can mean virtually anything related to customers'
taste or activities. For example, it can mean the frequency of customers' clicks on certain menus
where the items mean the menu items for this example.

This matrix is sparse, but not in a traditional way. Generally, we say a matrix is sparse if the number of nonzero entries
is much less than $\numcus \times \numitem$. We say \ratmat\ is sparse because there are huge number of entries
for which we \emph{do not know} the true values for them.
In a general recommendation problem, the purpose is to guess or estimate the true values,
\eg, what rating the customer would give if she did, or how frequently a customer would click on the menu item if she knew
there exist such menu item.

Now we want to evaluate the similarity (or distance) among items.
In many places in the recommendation systems and information retrieval literature,
it is shown that applying some transformation or weighting on the values
before calculating the similarity measure,
\eg, \href{https://en.wikipedia.org/wiki/Cosine_similarity}{cosine similarity}
or \href{https://en.wikipedia.org/wiki/Correlation_coefficient}{correlation coefficient}\footnote{
    The correlation coefficient is sometimes called the Pearson correlation coefficient
    after \href{https://en.wikipedia.org/wiki/Karl_Pearson}{Karl Pearson}, who was an English mathematician and biostatistician.
}.
There weighting methods and similarity measures will be discussed in later sections.

\subsection{Rating matrix modeling for menu personalization for mobile shopping app}

Here we consider a problem of using a collaborative filtering for efficient menu personalization problem for an mobile shoppiong app.
Suppose that we have some history data of the number of clicks or tabs on each menu item by each customer for certain period.
Therefore we can define the matrix in (\ref{eq:rating-matrix})
where
\numcus\ denotes the number of customers,
\numitem\ denotes the number of menu items,
and
$\ratmat_{ij}$ denotes the number of clicks or tabs on $j$th menu item by $i$th customer
with $1\leq i\leq \numcus$ and $1\leq j\leq \numitem$.
recommendation system design cases where each component represents the rating of an item,
these numbers vary depending on the time interval for which we collect the data.
So here we model this matrix as a function of time.
Moreover, we can have more than one source for the customers' data.
Thus we assume multi-source model.

\subsubsection{Rating matrix modeling}

Suppose that $\ratmat_s: \reals \to \reals^{\numcus \times \numitem}$ for $s\in\datsourceset$ is the rating matrix dependent on time,
\ie,
\begin{equation}
\label{eq:rating-matrix-temporal}
\ratmat_s(t)  \in \reals^{\numcus \times \numitem}
\end{equation}
where
\datsourceset\ refers to the set of data sources
and $\ratmat_s(t)_{ij}$ refers to the number of clicks or tabs for $j$th item by $i$th customer
which was collected from data source, $s \in \datsourceset$.
For example, $\datsourceset = \{\mathrm{mobile}, \mathrm{web}\}$.

For our purpose, we want to have one matrix which represents the customers' activity or behavior with their taste
by properly aggregating the history data.
Among many possible choices, we choose exponentially decaying weight method together with proper weight on data sources.
We define the following aggregate rating matrix
\begin{equation}
\label{eq:ratmat-aggr}
\ratmataggr(t) = \sum_{s\in\datsourceset} \sum_{\tau=0}^\infty \gamma^\tau w_s \ratmat_s(t-\tau)
\end{equation}
where $\gamma$ is a positive constant less than $1$ and $w_s$ are the weights on each data source such that
$\sum_{s\in\datsourceset} w_s = 1$.


This aggregate matrix, however, has one problem.
Suppose that the $1$st customer and the $2$nd customer have the very same activity pattern on menu items.
However, the $1$st customer happens to have actively used the shopping app or website for this week,
but the $2$nd customer has not used the app or website for the past few days or weeks.
Because of the decay factor, the $2$nd customer's activity would appear much less than those of the $1$st customer
even ideally we need to have the very same pattern for those two customers.
Therefore we need to normalize the values so that these two customers' preferences are considered the same.
For this we normalize each row, so that the sum of each row is always one.
\begin{equation}
\label{eq:ratmat-norm}
\ratmatnorm(t) = \diag(\ratmataggr(t) \ones_{\numitem})^{-1}\ratmataggr(t)
\end{equation}
where $\ones_{\numitem}\in\reals^{\numitem}$ is the vector where every entry is $1$
and $\diag(x)$ for $x\in\reals^n$ refers to a diagonal matrix whose diagonal entries are the entries of $x$ in the same order.
We can readily see that
\begin{equation}
\ratmatnorm(t) \ones_{\numitem} = \diag(\ratmataggr(t) \ones_{\numitem})^{-1}\ratmataggr(t) \ones_{\numitem} = \ones_{\numcus},
\end{equation}
\ie, the sum of each row is $1$.
From this point on, we will remove the subscript for $\ones$ unless it can cause confusion as to the dimension of the vector.

One problem of this approach is that $\ratmatnorm(t)$ cannot be defined if some row has all zero entries.
Even if the sum of every row is nonzero, if the values is very small,
\eg, a customer has not shown any activity except for a few clicks or tabs on a handful of menu items,
that doesn't mean that the corresponding row represents the customer's preference.
Hence, we can augment the values of $\ratmatnorm(t)$ by prior distribution of menu items
as in the following section.


\subsection{Value augmentation based on Bayesian MAP}
\label{subsec:bayes-prior}

To prevent the divide-by-zero errors from occurring while evaluating (\ref{eq:ratmat-norm}),
we consider maximum a posteriori (MAP) estimation for all the rows of $\ratmataggr(t)$ in (\ref{eq:ratmat-aggr}).

We can think of a problem of filling out each entry in $\ratmataggr(t)$
as performing $N$ multinomial experiments on each item $j\in\{1,\ldots,\numitem\}$
and count the occurrences for each item and fill in $\ratmataggr(t)$ for each customer $i\in\{1,\ldots,\numcus\}$
after normalization.


Now assume that we the prior as Dirichlet-multinomial model,
\ie,
\begin{equation}
\dir(\theta|\alpha) = \frac{1}{B(\alpha)} \prod_{k=1}^{\numitem} \theta_k^{\alpha_k-1} I(x = k).
\end{equation}
Then since the likelihood for the multinomial distribution has the form
\begin{equation}
p(\mathcal{D}|\theta) = \prod_{k=1}^{\numitem} \theta_k^{N_k},
\end{equation}
the posterior is
\begin{equation}
p(\theta|\mathcal{D})
\propto p(\mathcal{D}|\theta)  p(\theta)
\propto \prod_{k=1}^{\numitem} \theta_k^{\alpha_k+N_k-1}
= \dir(\theta|\alpha_1+N_1, \ldots, \alpha_{\numitem} + N_{\numitem}).
\end{equation}

It can be proved that the maximum a posteriori (MAP) estimate for $\theta_1, \ldots, \theta_{\numitem}$ is
\begin{equation}
\label{eq:map-estimate}
\popestk{k} = \frac{N_k + \alpha_k - 1}{N + \alpha_0 -{\numitem}}
\end{equation}
where $\alpha_0 = \sum_{k=1}^{\numitem} \alpha_k$ (K.P. Murphy).
By choosing proper values for $\alpha_k > 1$, we can make every entry nonzero in $\ratmataggr(t)$.

Adding this information to the rating matrix is equivalent to Bayesian inference
since this uses a priori distribution.
Therefore, it will play a regularization role in our inference.

\subsection{Similarity measure among items}

Suppose that we have the transformed matrix, $\ratmatnorm\in\reals^{\numcus \times \numitem}$.
For the case of mobile shopping app menu personalization, $\ratmatnorm = \ratmatnorm(t)$ for some $t$.
For general cases, \ratmatnorm\ equals to \ratmat\ with all \nan s replaced by $0$.

Suppose that $c_1, \ldots, c_{\numcus} \in \reals^{\numitem}$ are the row vectors of \ratmatnorm\
and $d_1, \ldots, d_{\numitem} \in \reals^{\numcus}$ are the column vectors of \ratmatnorm,
\ie,
\begin{equation}
\ratmatnorm
= \begin{my-matrix}{c}
c_1^T
\\
\vdots
\\
c_{\numcus}^T
\end{my-matrix}
= \begin{my-matrix}{ccc}
d_1
&
\cdots
&
d_{\numitem}
\end{my-matrix}
\in\reals^{\numcus \times \numitem}.
\end{equation}

\subsubsection{Cosine similarity}

The cosine similarity measures the cosine of the angle between the two vectors,
\ie, it is defined by
\begin{equation}
\label{eq:cos-simil}
\simil(d_i, d_j) = \frac{d_i^T d_j}{\|d_i\|_2 \|d_j\|_2}
\end{equation}
where $x^Ty$ denotes the inner product of two vectors $x$ and $y$,
and $\|\cdot\|_2$ denotes the $2$-norm of a vector.
The \href{https://en.wikipedia.org/wiki/Jensen\%27s_inequality}{Jensen's inequality}
guarantees that $-1 \leq \simil(d_i,d_j) \leq 1$.
It can be easily shown that we have $0 \leq \simil(d_i,d_j) \leq 1$
when every entry in \ratmatnorm\ is nonzero.

\subsubsection{Cosine similarity when prior distribution is used}

When a prior distribution is added to \ratmatnorm\ as in \S\ref{subsec:bayes-prior},
the sparsity breaks and the matrix becomes a dense matrix.
Therefore it seems that we lose huge advantage in computation efforts
when calculating the similarities.
However, because the added matrix is rank-one matrix, we can still exploit the sparsity and calculate the similarities
at almost the same cost as before.

Assume that \popestk{k}\ in (\ref{eq:map-estimate}) has been calculated and is added to \ratmatnorm.
Thus we have a new rating matrix.
\begin{equation}
\ratmatnorm_\popest
= \ratmatnorm + \lambda \ones_{\numcus\times\numitem}\diag(\popest)
= \ratmatnorm + \lambda \rowvecthree{\popestk{1} \ones}{\cdots}{\popestk{\numitem} \ones}
\in \reals^{\numcus \times \numitem}
\end{equation}
where $\popest = \rowvecthree{\popestk{1}}{\cdots}{\popestk{\numitem}}^T \in \reals^\numitem$.
Here $\lambda$ plays a similar role as the coefficient for the regularization
when we assume that
\begin{equation}
        p(\popest) \sim N(0,\lambda I_\numitem).
\end{equation}
Now let $\tilde{d}_1$, \ldots, $\tilde{d}_\numitem$ are the column vectors of $\ratmatnorm_\popest$.
Then the cosine similarity of $i$th item and $j$th item is
\begin{equation}
\simil(\tilde{d}_i, \tilde{d}_j) = \frac{\tilde{d}_i^T \tilde{d}_j}{\|\tilde{d}_i\|_2 \|\tilde{d}_j\|_2}.
\end{equation}
Now note that
\begin{equation}
\label{eq:vusg}
\tilde{d}_i^T \tilde{d}_j = (d_i + \lambda \popestk{i} \ones)^T(d_j + \lambda \popestk{j} \ones)
= d_i ^Td_j
+ \lambda \popestk{j} \ones^T d_i^T
+ \lambda \popestk{i} \ones^T d_j
+ \lambda^2 \popestk{i} \popestk{j} \numcus
\end{equation}
hence
\begin{equation}
\label{eq:ubms}
\|\tilde{d}_i\|_2^2 = \tilde{d}_i^T \tilde{d}_i
= \|d_i\|_2^2
+ 2 \lambda \popestk{i} \ones^T d_i^T
+ \lambda^2 \popestk{i}^2 \numcus
\end{equation}

We can pre-compute $\ones^T d_i$ for $i=1,\ldots, \numitem$
which takes less than \numrating\ where \numrating\ refers to the number of nonzero entries in \ratmatnorm.
Hence, the additional computational cost is negligible.

We can also solve this problem at a different angle. In order to calculate the inner projects and the norms in
(\ref{eq:vusg}) and (\ref{eq:ubms}), we can do the following matrix multiplication.
\begin{eqnarray}
\ratmatnorm_\popest^T
\ratmatnorm_\popest
&=&
(\ratmatnorm + \lambda \ones_{\numcus\times\numitem}\diag(\popest))^T
(\ratmatnorm + \lambda \ones_{\numcus\times\numitem}\diag(\popest))
\\
&=&
\ratmatnorm^T \ratmatnorm
+ \lambda \ratmatnorm^T \ones_{\numcus\times\numitem}\diag(\popest)
+\lambda \diag(\popest)  \ones_{\numitem\times\numcus} \ratmatnorm
\\
&&+\lambda^2 \diag(\popest) \ones_{\numitem\times\numcus} \ones_{\numcus\times\numitem}\diag(\popest)
\\
&=&
\ratmatnorm^T \ratmatnorm
+\lambda \popest \tilde{r}^T
+\lambda \tilde{r} \popest^T
+\lambda^2 \numcus \popest \popest^T
\end{eqnarray}

\subsubsection{Correlation coefficient similarity}
The correlation coefficient is defined by
\begin{equation}
\label{eq:corr-coef-simil}
\simil(d_i, d_j) = \frac{(d_i-\ones^Td_i/\numcus)^T (d_j-\ones^T d_j/\numcus)}{\|d_i-\ones^Td_i/\numcus\|_2 \|d_j-\ones^Td_j/\numcus\|_2}.
\end{equation}
Again the \href{https://en.wikipedia.org/wiki/Jensen\%27s_inequality}{Jensen's inequality}
guarantees that $-1 \leq \simil(d_i,d_j) \leq 1$,
but the fact that every entry in \ratmatnorm\ is nonzero does not make this similarity nonnegative.


\subsection{Data value transformation}

When there are popular items across many customers,
those values can dominate in the similarity measure evaluation.
For example,
the aggregate rating matrix, \ratmataggr, looks like the following:
\begin{equation}
\begin{array}{c}
\begin{my-matrix}{ccccc}
\cdots & 100 & \cdots & 130 & \cdots
\\
\cdots & 2 & \cdots & 5 & \cdots
\\
\cdots & 3 & \cdots & 3 & \cdots
\\
\cdots & 10 & \cdots & 1 & \cdots
\end{my-matrix}
\\
\begin{array}{ccccc}
& \uparrow & \ \ \ \ \ \   & \uparrow &
\\
& d_i &  & d_j &
\end{array}
\end{array}
\end{equation}

Note here that the scores in the first row are much larger than the other terms.
This can be caused by the fact that some customers are way more active than other customers,
\eg, they can listen to some musics many times or uses an shopping app very frequently.
Now the cosine similarity and the correlation coefficient similarity between $d_i$ and $d_j$
are
\begin{equation}
0.9956 \mbox{ and } 0.9951
\end{equation}
respectively.
However, if we remove the first customer and recalculate both similarities,
it yields
\begin{equation}
0.4611 \mbox{ and } -0.9176
\end{equation}
respectively.

Note that both similarity measures give very different measures.
The correlation coefficient similarity, which also tells whether both have positive or negative correlations
by its sign, gives opposite signs.
This shows how \emph{some} customers activity or behavior can change the item-to-item similarities drastically.

However, this doesn't seem to be right.
The item-to-time similarities are supposed to represent the nature of the relation among items,
hence should not be decided by a small number of extremely active customers.

For this reason, in many recommendation systems literature,
it has been reported that the similarity measures mentioned above do not show good performance.
To address this issue,
various value transformation methods have been introduced.
We will discuss some of these methods in the following sections.

\subsubsection{TFIDF (or tf-idf)}

The \href{https://en.wikipedia.org/wiki/Tf-idf}{term frequency–inverse document frequency} (TFIDF or tf-idf)
is a numerical statistic which has been widely used in the field of
\href{https://en.wikipedia.org/wiki/Information_retrieval}{information retrieval}.
It was orginally designed to reflect how important a word is to a document in a collection or corpus.
It is often used as a weighting factor in searches of information retrieval,
text mining, and user modeling.
The tf–idf value increases proportionally to the number of times a word appears in the document
and is offset by the number of documents in the corpus that contain the word,
which helps to adjust for the fact that some words appear more frequently in general.
Tf–idf is one of the most popular term-weighting schemes today;
83\% of text-based recommender systems in digital libraries use tf–idf.

\subsubsection{Okapi BM25 transformation}

Instead, the \href{https://en.wikipedia.org/wiki/Okapi_BM25}{Okapi BM25} score has shown much better performance when applied to the entries in \ratmatnorm\
before calculating similarity measures.
It is defined by
\begin{equation}
\bmscore(i,j) = \log\left(\frac{\numcus - \|d_j\|_0 + 0.5}{\|d_j\|_0 + 0.5}\right)
\frac{\ratmatnorm_{ij}(k_1+1)}{\ratmatnorm_{ij} + k_1 \left(1-b+b\frac{\|c_i\|_0}{\avgcl}\right)}
\end{equation}
where
$\|\cdot\|_0$ denotes the number of nonzero entries of a vector
and
$k_{1}$ and $b$ are some parameters usually with $k_{1}\in [1.2,2.0]$ and $b \sim 0.75$.


\subsection{Recommendation based on item similarities}

Now finally, we can evaluate new (menu) item scores for each (menu) items for each customer.
Here we suggest two methods.

\begin{itemize}
\item For customer $i$ and item $j$, we calculate new recommendation as
\begin{equation}
\ratmatpred_{ij} = \frac{\sum_{i=1}^{\numitem}\simil(i,j) \ratmatnorm_{i,j}}{\sum_{i=1}^{\numitem}\simil(i,j)}
\end{equation}
where $\simil(i,j)$ is the cosine similarity between $d_i$ and $d_j$.

\item The second candidate considers the deviation from the average, \ie,
\begin{equation}
\ratmatpred_{ij} =
\ones^Tc_i/\numitem
+
\frac{\sum_{i=1}^{\numitem}\simil(i,j)(\ratmatnorm_{ij}-\ones^Tc_i/\numitem) }{\sum_{i=1}^{\numitem}\simil(i,j)}
\end{equation}
where $\simil(i,j)$ is the correlation coefficient similarity between $d_i$ and $d_j$.
\end{itemize}


These new values are the ones obtained from the history data for that particular customer together
with those for neighboring customers. Hence, this will give better recommendation.


\newpage
\section{Collaborative Filtering using Matrix Factorization}

Here we discuss the collaborative filtering using matrix factorization,
which uses latent factor models.
There are two types of latent factors; one for customers and one for items.

The customer latent factors represent customers' taste or tendency.
In psychological perspective, even customers themselves may not realize these,
but they implicitly express these by their activities.
They can be
\begin{itemize}
\item the degree of pursuing economical life
\item the degree of interest in tableware
\item the degree of caring children
\item the degree of interest in books
\item the degree of putting values on family
\end{itemize}

On the other hand, the item latent factors represent items' attributes.
Again, customers may not realize these, but they tend to click on some menus or items
according items attributes and their inclination.
They can be
\begin{itemize}
\item the probability of leading to inexpensive items
\item the probability of leading to tableware
\item the probability of leading to items related to children
\item the probability of leading to book purchase
\item the probability of leading to items related to family
\end{itemize}

We will discuss how we can utilize these latent factors to design our collaborative filtering below.
Note, however, that it is generally impossible to identify the actual latent factors
a ML algorithm yields with any of the aforementioned factors.
These latent factors are \emph{not predefined or decided apriori},
but are learned or revealed through the values coming out of the algorithms.
What's more important is that these methods work very well in practice.

We will also describe different types of problem formulations
and different approaches to solve them,
\eg,
stochastic gradient descent (SGD) method,
alternating SGD,
and alternating least squares (ALS).


\subsection{Problem definition and formulations}


As before, suppose that there are \numcus\ customers and \numitem\ items.
We assume that we have a (sparse) rating matrix,
$\ratmat \in (\reals\cup\{\nan\})^{\numcus \times \numitem}$, in (\ref{eq:rating-matrix}).
Sometimes we need to deal with the aggregate rating matrix, $
\ratmataggr(t) \in (\reals\cup\{\nan\})^{\numcus \times \numitem}$, in (\ref{eq:ratmat-aggr}).
To simplify notations, we will refer to both types of rating matrix as \ratmat.

Now we assume that there are \numfactors\ latent factors.
This means that every customer \numfactors\ latent factors and every item has \numfactors\ latent factors.

Let $x_i \in \reals^\numfactors$ be a column vector representing \numfactors\ latent factors for the $i$th customer
and let $y_j \in \reals^\numfactors$ be a column vector representing \numfactors\ latent factors for the $j$th item
with
\begin{equation}
x_i = \begin{my-matrix}{c} x_{i,1} \\ \vdots \\ x_{i,\numfactors} \end{my-matrix}
\in\reals^\numfactors
\end{equation}
and
\begin{equation}
y_j = \begin{my-matrix}{c} y_{j,1} \\ \vdots \\ y_{j,\numfactors} \end{my-matrix}
\in\reals^\numfactors.
\end{equation}
Then $x_{i,1}, \ldots, x_{i,\numfactors}$ are the \numfactors\ tendency or taste factors of the $i$th customer
and let $y_{j,1}, \ldots, y_{j,\numfactors}$ are the \numfactors\ attributes of the $j$th item.

Then we assume that the rating of the $j$ item by the $i$ customer can be inferenced (or estimated)
by the sum of the corresponding factors,
\ie,
\begin{equation}
\label{eq:dixk}
\ratmat_{i,j} \simeq \hat{\ratmat}_{i,j} = x_{i,1} y_{j,1} + \cdots + x_{i,\numfactors} y_{j,\numfactors}.
\end{equation}
\emph{This is the most critical assumption in the matrix factorization.}

The purpose of the collaborative filtering is to find these latent factors so as to make these inference
as accurate as possible,
\ie,
to solve the following optimization problem
\begin{equation}
\label{eq:mat-fac-form-1}
\begin{array}{ll}
\mbox{minimize} & \sum_{1\leq i\leq \numcus, 1\leq j\leq \numitem:\ratmat_{i,j} \in \reals} \lfcn({\ratmat}_{i,j}, \hat{\ratmat}_{i,j})
\end{array}
\end{equation}
where the $\numfactors(\numcus + \numitem)$ optimization variables are $x_{i,k}$ and $y_{j,k}$
for $1\leq i\leq \numcus$, $1\leq j\leq \numitem$, and $1\leq k\leq \numfactors$,
and $l:\reals\times\reals\to\preals$ is a loss function measuring the distance between the true value and the estimate,
hoping that $\hat{\ratmat}_{i,j}$ can accurately predict the rating for $1\leq i\leq \numcus$ and $1\leq j\leq \numitem$
where $\ratmat_{i,j}$ is not given.

In most cases, we use squared loss for \lfcn, so we will also sue the squared loss (except some special cases).
\begin{equation}
\lfcn(y_1, y_2) = (y_1 - y_2)^2.
\end{equation}

Now let us come up with more compact notation to describe our problem.
Let $X \in\reals^{\numcus\times\numfactors}$ be the customer latent factor matrix whose $i$th row is $x_i^T$, \ie,
\begin{equation}
X = \begin{my-matrix}{c}
x_1^T \\ \vdots \\ x_\numcus^T
\end{my-matrix}
= \begin{my-matrix}{cccc}
x_{1,1} & x_{1,2} & \cdots & x_{1,\numfactors}
\\
x_{2,1} & x_{2,2} & \cdots & x_{2,\numfactors}
\\
\vdots & \vdots & \ddots & \vdots
\\
x_{\numcus,1} & x_{\numcus,2} & \cdots & x_{\numcus,\numfactors}
\end{my-matrix}
\in
\reals^{\numcus\times\numfactors}.
\end{equation}
Likewise, let $Y \in\reals^{\numitem\times\numfactors}$ be the item latent factor matrix whose $j$th row is $y_j^T$, \ie,
\begin{equation}
Y = \begin{my-matrix}{c}
y_1^T \\ \vdots \\ y_\numitem^T
\end{my-matrix}
= \begin{my-matrix}{cccc}
y_{1,1} & y_{1,2} & \cdots & y_{1,\numfactors}
\\
y_{2,1} & y_{2,2} & \cdots & y_{2,\numfactors}
\\
\vdots & \vdots & \ddots & \vdots
\\
y_{\numitem,1} & y_{\numitem,2} & \cdots & y_{\numitem,\numfactors}
\end{my-matrix}
\in
\reals^{\numitem\times\numfactors}.
\end{equation}
Note that these two matrices are extremely skinny meaning that we have much more rows then columns in general
since \numcus\ could be hundreds of millions and \numitem\ could be hundreds of thousands,
but \numfactors\ is $100$ or so at most.

Now using this notation, we can say
\begin{equation}
\hat{\ratmat} = XY^T,
\end{equation}
which is mathematically equivalent to
\begin{equation}
\hat{\ratmat}_{i,j} = x_i^T y_j = x_{i,1} y_{j,1} + \cdots + x_{i,\numfactors} y_{j,\numfactors},
\end{equation}
which again is equivalent to (\ref{eq:dixk}).

Then the optimization problem (\ref{eq:mat-fac-form-1}) can be rewritten as
\begin{equation}
\label{eq:mat-fac-form-2}
\begin{array}{ll}
\mbox{minimize} & \sum_{1\leq i\leq \numcus, 1\leq j\leq \numitem:\ratmat_{i,j} \in \reals} \lfcn({\ratmat}_{i,j}, x_i^Ty_j)
\end{array}
\end{equation}
where the optimization variables are
$X\in\reals^{\numcus\times\numfactors}$
and
$Y\in\reals^{\numitem\times\numfactors}$.

If we use the squared loss for \lfcn, then we can write this equation more compactly as follows.
\begin{equation}
\label{eq:mat-fac-form-3}
\begin{array}{ll}
\mbox{minimize} & \|\ratmat - XY^T\|_{F,\ratmat}^2
\end{array}
\end{equation}
where $\|Z\|_{F,\ratmat}$ refers to \href{http://mathworld.wolfram.com/FrobeniusNorm.html}{Frobenius norm}
calculated for those entries $Z_{i,j}$ with $\ratmat_{i,j} \in\reals$.

Note that the number of real variables we need to optimize is the same for all the formulations
(\ref{eq:mat-fac-form-1}),
(\ref{eq:mat-fac-form-2}),
and
(\ref{eq:mat-fac-form-3}) (because they are equivalent optimization problems).


\subsection{Solution methods}

\subsubsection{Matrix factorization via singular value decomposition (SVD)}


First we discuss obtaining the latent factors, \ie, $X$ and $Y$,
using sparse singular value decomposition (SVD).

For any rank-$k$ matrix $A\in\reals^{m\times n}$, we always have a SVD
\begin{equation}
A = U \Sigma V^T
\end{equation}
where
$U\in\reals^{m\times k}$,
$V\in\reals^{n\times k}$,
and
$\Sigma\in\reals^{k\times k}$.
(Refer to \S\ref{subsubsec:svd}.)
The basic idea of obtaining the latent factors using SVD is
to apply SVD to the rating matrix.

However, the matrix $\ratmat \in (\reals\cup\{\nan\})^{\numcus \times \numitem}$ can have many unknown values,
hence we cannot apply SVD directly.
Instead, we employ a certain type of missing data imputation to replace the unknowns with real values,
then apply SVD to the matrix.
There can be many options for the missing data imputation.
Here we list some oft them.
\begin{itemize}
\item replacing every unknown with zeros
\item replacing unknowns with global item average ratings
\item replacing unknowns with global customer average ratings
\item use some model based methods (\eg, the item-based collaborative filtering itself can be considered as such a method)
\end{itemize}


Now suppose that $\tilde{\ratmat}$ is \ratmat\ with missing values replaced by proper real values.
We compute the largest \numfactors\ singular values with corresponding singular vectors.
Then we can obtain the approximation of $\tilde{\ratmat}$ by
\begin{equation}
\label{eq:voxa}
\tilde{\ratmat} \simeq U_\numfactors \Sigma_{\numfactors} V_\numfactors^T
\end{equation}
where
\begin{equation}
U_\numfactors = \rowvecthree{u_1}{\cdots}{u_\numfactors} \in \reals^{\numcus\times \numfactors}
\end{equation}
\begin{equation}
V_\numfactors = \rowvecthree{v_1}{\cdots}{v_\numfactors} \in \reals^{\numitem\times \numfactors}
\end{equation}
and
\begin{equation}
\Sigma_\numfactors = \diag(\sigma_1, \cdots, \sigma_\numfactors) \in \reals^{\numfactors\times \numfactors}.
\end{equation}
Refer to \S\ref{subsubsec:svd} for more details.
This approximation is the best approximation to $\tilde{\ratmat}$
in Frobenius norm sense. (Refer to \S\ref{subsubsec:svd-approx} for further details.)



Now since (\ref{eq:voxa}) can be rewritten as
\begin{equation}
\tilde{\ratmat} \simeq U_\numfactors \Sigma_{\numfactors} V_\numfactors^T
= U_\numfactors \Sigma_{\numfactors}^{1/2} \Sigma_{\numfactors}^{1/2} V_\numfactors^T
= (U_\numfactors \Sigma_{\numfactors}^{1/2}) (V_\numfactors \Sigma_{\numfactors}^{1/2})^T
\end{equation}
where
\begin{equation}
\Sigma_\numfactors^{1/2} = \diag(\sigma_1^{1/2}, \cdots, \sigma_\numfactors^{1/2}) \in \reals^{\numfactors\times \numfactors}
\end{equation}
we can obtain the customer and item latent factor matrices as follows.
\begin{equation}
\Xsvd = U_\numfactors \Sigma_{\numfactors}^{1/2} = \rowvecthree{\sigma_1^{1/2} u_1}{\cdots}{\sigma_\numfactors^{1/2} u_\numfactors}
\in\reals^{\numcus\times \numfactors}
\end{equation}
\begin{equation}
\Ysvd = V_\numfactors \Sigma_{\numfactors}^{1/2} = \rowvecthree{\sigma_1^{1/2} v_1}{\cdots}{\sigma_\numfactors^{1/2} v_\numfactors}
\in\reals^{\numcus\times \numfactors}
\end{equation}

\subsubsection{Matrix factorization via gradient descent (GD) method}

One obvious way to obtain the latent factor matrices is to directly apply
\href{https://en.wikipedia.org/wiki/Gradient_descent}{gradient descent method}
to the following optimization problem directly.
\begin{equation}
\label{eq:mat-fac-form-4}
\begin{array}{ll}
\mbox{minimize} & f(X,Y) = \sum_{i, j:\ratmat_{i,j} \in \reals} \lfcn({\ratmat}_{i,j}, x_i^T y_j)
\end{array}
\end{equation}
which is equivalent to (\ref{eq:mat-fac-form-2}).
Here we denote the objective function of the optimization problem by
$f:\reals^{\numcus \times \numfactors}
\times
\reals^{\numitem \times \numfactors}
\to
\preals$.

In order to apply gradient descent, we need to evaluate the partial derivative of the objective function
with respect to
\begin{itemize}
\item $x_{i,k}$ for all $1\leq k\leq \numfactors$ and all $i \in \set{1\leq i\leq \numcus}{\ratmat_{i,j} \in\reals \mbox{ for some } 1\leq j\leq \numitem}$
\item $y_{j,k}$ for all $1\leq k\leq \numfactors$ and all $j \in \set{1\leq j\leq \numitem}{\ratmat_{i,j} \in\reals \mbox{ for some } 1\leq i\leq \numcus}$
\end{itemize}
For simplicity of the equation derivation,
let us assume that
\begin{eqnarray}
\set{1\leq i\leq \numcus}{\ratmat_{i,j} \in\reals \mbox{ for some } 1\leq j\leq \numitem}
&=&
\{1,\ldots, \numcus\}
\\
\set{1\leq j\leq \numitem}{\ratmat_{i,j} \in\reals \mbox{ for some } 1\leq i\leq \numcus}
&=&
\{1,\ldots, \numitem\}
\end{eqnarray}
\ie,
every item has at least one rating and every customer has at least one rating.


Now the gradient of $f(X,Y)$ with respect to each $x_i$ is
\begin{equation}
\label{eq:mat-fac-grad-x}
\nabla_{x_i}  f(X,Y)
= \sum_{j:\ratmat_{i,j}\in\reals} \frac{\partial}{\partial y_2} \lfcn(\ratmat_{i,j}, x_i^T y_j) y_j \in\reals^\numfactors
\mbox{ for } 1\leq i\leq \numcus
\end{equation}
and that with respect to $y_j$ is
\begin{equation}
\label{eq:mat-fac-grad-y}
\nabla_{y_j}  f(X,Y)
= \sum_{i:\ratmat_{i,j}\in\reals} \frac{\partial}{\partial y_2} \lfcn(\ratmat_{i,j}, x_i^T y_j) x_i \in\reals^\numfactors
\mbox{ for } 1\leq j\leq \numitem
\end{equation}

If we use the squared loss for \lfcn,
the optimization problem becomes
\begin{equation}
\label{eq:mat-fac-form-5}
\begin{array}{ll}
\mbox{minimize} & f(X,Y) = \|\ratmat - XY^T\|_{F,\ratmat}^2
\end{array}
\end{equation}
which is equivalent to (\ref{eq:mat-fac-form-3}).
Then (\ref{eq:mat-fac-grad-x}) and (\ref{eq:mat-fac-grad-y})
imply that the gradients of $f(X,Y)$ can be calculated by
\begin{equation}
\label{eq:mat-fac-sq-grad-x}
\nabla_{x_i}  f(X,Y)
= - \sum_{j:\ratmat_{i,j}\in\reals} (\ratmat_{i,j} -x_i^T y_j) y_j \in\reals^\numfactors
\mbox{ for } 1\leq i\leq \numcus
\end{equation}
and that with respect to $y_j$ is
\begin{equation}
\label{eq:mat-fac-sq-grad-y}
\nabla_{y_j}  f(X,Y)
= - \sum_{i:\ratmat_{i,j}\in\reals} (\ratmat_{i,j}- x_i^T y_j) x_i \in\reals^\numfactors
\mbox{ for } 1\leq j\leq \numitem
\end{equation}
For notational convenience, we stack these vectors to form derivative matrices as below.

If $\ratmat\in\reals^{\numcus \times \numitem}$,
\ie, there are no missing values,
we can write the gradients more compact form,
\ie, as the derivatives of $X$ and $Y$.
\begin{equation}
\label{eq:mat-fac-sq-deriv-x}
D_X f(X,Y) = -2 (\ratmat-XY^T) Y
\in\reals^{\numcus\times\numfactors}
\end{equation}
and
\begin{equation}
\label{eq:mat-fac-sq-deriv-x}
D_Y f(X,Y) = -2 (\ratmat^T-YX^T) X
\in\reals^{\numitem\times\numfactors}
\end{equation}


Finally, we describe the gradient descent method.

\begin{itemize}
\item choose learning rate strategy $\eta_k>0$
\item choose initial $\initX$ and $\initY$
\item let $\XX{0}:=\initX$ and $\YY{0}:=\initY$
\item let $k:=0$
\item for each iteration, update $X$ and $Y$
\begin{eqnarray}
    \XX{k+1} &=& \XX{k} - \eta_k D_X f(\XX{k},\YY{k})
\\
\label{eq:btai}
    \YY{k+1} &=& \YY{k} - \eta_k D_Y f(\XX{k},\YY{k})
\end{eqnarray}
\item stops if certain stopping criterion is satisfied
\item update $k:=k+1$ and repeat iterations
\end{itemize}

\subsubsection{Matrix factorization via alternating gradient descent (GD) method}

Instead of updating $X$ and $Y$ simultaneously,
we can update $X$ and $Y$ after the other is updated.
The algorithm can be described as follows.

\begin{itemize}
\item choose learning rate strategy $\eta_k>0$
\item choose initial $\initX$ and $\initY$
\item let $\XX{0}:=\initX$ and $\YY{0}:=\initY$
\item let $k:=0$
\item for each iteration, update $X$ and $Y$
\begin{eqnarray}
    \XX{k+1} &=& \XX{k} - \eta_k D_X f(\XX{k},\YY{k})
\\
\label{eq:bhio}
    \YY{k+1} &=& \YY{k} - \eta_k D_Y f(\XX{k+1},\YY{k})
\end{eqnarray}
\item stops if certain stopping criterion is satisfied
\item update $k:=k+1$ and repeat iterations
\end{itemize}

Note the difference in (\ref{eq:btai}) and (\ref{eq:bhio}).
This method updates $Y$ with most recent $X$ values.

\subsubsection{Matrix factorization via stochastic gradient descent (SGD) method}

Theoretically both GD and alternating GD converge (to local minima) with proper learning rate strategies.
However, the purpose of the collaborative filtering is not minimize (the sum of) errors (or loss function values),
but accurately predict missing ratings, \ie, the ratings that a customer has never given,
or sometimes accurately predict the ranking of the items.
In this case, what we want to achieve is the solution to the following stochastic optimization problem.
\begin{equation}
\label{eq:mat-fac-form-stoch}
\begin{array}{ll}
\mbox{minimize} & \Expect \sum_{i,j:\ratmat_{i,j} = \nan} \lfcn(\tilde{\ratmat}_{i,j}, \x_i^T y_j)
\end{array}
\end{equation}
where $\Expect(\cdot)$ denotes the expected value (or mean/average) of a random variable
and $\tilde{\ratmat}_{i,j}$ denotes the rating that the $i$th customer would give to $j$th item in the future.

Unfortunately there is no direct way to solve this problem
because this stochastic optimization problem is not (exactly) solvable.
Most importantly, we do not have the future data (some of them would be never available).
Therefore solving (\ref{eq:mat-fac-form-4}) is not what we want.

There is also another problem with solving (\ref{eq:mat-fac-form-4}); computational cost per iteration.
The gradient evaluation takes $3\numfactors \numrating$ multiplications and $3\numfactors \numrating$ additions
where \numrating\ refers to the number of known ratings.
Thus if the density of \ratmat\ is $\alpha$, the number of multiplications and additions required to evaluate the gradient
is $3\alpha \numfactors \numcus \numitem$.
Therefore, when \numcus\ or \numitem\ or both are huge, the cost for the gradient calculation can be huge.


To resolve these two problems at the same time, we can use stochastic gradient descent (SGD) method with mini-batch methods.
The mini-batch method is to use fixed size of training sets for each gradient descent iteration.
This can save the computational cost considerably while approximately solving the stochastic optimization problem (\ref{eq:mat-fac-form-stoch}).



\subsubsection{Matrix factorization via alternating least-squares (ALS)}

When we use the squared loss, we solve the problem (\ref{eq:mat-fac-form-5}).
The objective function of this problem is
\begin{equation}
\label{eq:vuwj}
f(X,Y) = \|\ratmat - XY^T\|_{F,\ratmat}^2
\end{equation}

This is a quadratic function of a \href{https://en.wikipedia.org/wiki/Bilinear_map}{bilinear function} of $X$ and $Y$.
Unfortunately, it is not a convex function. If it were a convex function, probably we would not need to use (stochastic) gradient descent method
because we have much more efficient methods to solve it (\eg, Newton's method).
And if it were a convex function, we would be able to obtain the global minimum (even with gradient descent method).


However, if we fix either $X$ or $Y$, it becomes the convex function of the other. Indeed, the problem (\ref{eq:mat-fac-form-5}) becomes
a \href{https://en.wikipedia.org/wiki/Least_squares}{least-squares} problem when one of $X$ and $Y$ is fixed,
which leads to the alternating least-squares algorithm.

\begin{itemize}
\item choose initial $\initX$ and $\initY$
\item let $\XX{0}:=\initX$ and $\YY{0}:=\initY$
\item let $k:=0$
\item for each iteration, update $X$ and $Y$
\begin{eqnarray}
\label{eq:als-1}
    \XX{k+1} &=& \argmin_X \|\ratmat-X{\YY{k}}^T\|_{F,\ratmat}
\\
\label{eq:als-2}
    \YY{k+1} &=& \argmin_Y \|\ratmat-\XX{k+1}Y^T\|_{F,\ratmat}
\end{eqnarray}
\item stops if certain stopping criterion is satisfied
\item update $k:=k+1$ and repeat iterations
\end{itemize}

The ALS is named so since solving (\ref{eq:als-1}) or (\ref{eq:als-2}) is equivalent to solving a least-squares problem.
Note that the functions in (\ref{eq:als-1}) and (\ref{eq:als-2}) can be separated by $x_i$s or $y_j$s,
\eg, the function in (\ref{eq:als-1}) is
\begin{equation}
\label{eq:bian}
f_Y(X) = \|\ratmat - XY^T\|_{F,\ratmat}^2
= \sum_{i=1}^\numcus \|\tilde{\ratvec}_i^T-x_i^T Y^T\|_{F,\tilde{\ratvec}_i^T}^2
= \sum_{i=1}^\numcus \|\tilde{\ratvec}_i-Yx_i\|_{F,\tilde{\ratvec}_i}^2
\end{equation}
where $\tilde{r}_i \in\reals^\numitem$ are the row vectors of \ratmat,
thus solving (\ref{eq:als-1}) is equivalent to solving \numcus\ uncorrelated problems separately.
This is another great advantage of ALS since we can use parallelism to save the training time considerably.
For example, if we have $N$ processing units, the training time per iteration would be equivalent to
that for solving each (small) least-squares $\numcus/N$ times.
Note that we cannot separate the objective function in (\ref{eq:vuwj}) when we consider $X$ and $Y$ simultaneously
since each of \numrating\ terms in (\ref{eq:vuwj}) are intertwined through $x_i$s and $y_j$s.

When $\ratmat\in\reals^{\numcus\times\numitem}$, it can be easily shown that
\begin{equation}
x_i^\ast =  Y^\dagger \tilde{\ratvec}_i
\end{equation}
where
\begin{equation}
\label{eq:pseudo-inverse-y}
Y^\dagger = (Y^TY)^{-1}Y^T
\end{equation}
is the pseudo-inverse of $Y$.
Thus (\ref{eq:als-1}) becomes
\begin{equation}
\label{eq:als-1-sol}
\XX{k+1} = \ratmat {{\YY{k}}^\dagger}^T
= \ratmat {\YY{k}} \left({\YY{k}}^T{\YY{k}} \right)^{-1}.
\end{equation}
Likewise, (\ref{eq:als-2}) becomes
\begin{equation}
\label{eq:als-2-sol}
\YY{k+1} = \ratmat^T {{\XX{k+1}}^\dagger}^T
= \ratmat^T {\XX{k+1}} \left({\XX{k+1}}^T{\XX{k+1}} \right)^{-1}.
\end{equation}

Note that in practice, we do not evaluate the pseudo-inverse as in (\ref{eq:pseudo-inverse-y})
because it causes catastrophic numerical instability especially when the matrix is huge
(as in most recommendation system cases).
Instead, we use \href{https://en.wikipedia.org/wiki/QR_decomposition}{QR decomposition},
\ie, if $Y = QR$ is the QR decomposition of $Y$, $Y^\dagger = R^{-1} Q^T$.

We now discuss the interpretation of (\ref{eq:als-1-sol}) and (\ref{eq:als-2-sol}).
Let $X^\ast$ and $Y^\ast$ be the optimal solutions for the problem (\ref{eq:mat-fac-form-3}).
Then these should satisfy (\ref{eq:als-1}) and (\ref{eq:als-2}),
\ie,
\begin{eqnarray}
X^\ast &=& \argmin_{X} \|\ratmat - X{Y^\ast}^T \|_{F,\ratmat}
\\
Y^\ast &=& \argmin_{Y} \|\ratmat - X^\ast Y^T\|_{F,\ratmat}
\end{eqnarray}
Then (\ref{eq:als-1-sol}) and (\ref{eq:als-2-sol}) imply that
\begin{eqnarray}
X^\ast {Y^\ast}^T &=& \ratmat Y^\ast ({Y^\ast}^T Y^\ast)^{-1} {Y^\ast}^T = \ratmat \Ysim
\\
&=& X^\ast ({X^\ast}^T X^\ast)^{-1} {X^\ast}^T\ratmat = \Xsim \ratmat
\end{eqnarray}
where
$\Xsim$
and
$\Ysim$
\begin{eqnarray}
\Xsim &=& X^\ast ({X^\ast}^T X^\ast)^{-1} {X^\ast}^T\in \reals^{\numcus \times \numcus}
\\
\Ysim &=& Y^\ast ({Y^\ast}^T Y^\ast)^{-1} {Y^\ast}^T\in \reals^{\numitem \times \numitem}
\end{eqnarray}

These yield very interesting interpretations for \Xsim\ and \Ysim.
These equations imply that the optimal prediction for the rating of the $j$th item by the $i$th customer is
\begin{eqnarray}
\label{eq:als-item-simil}
{x_i^\ast}^T y_j^\ast &=& \sum_{k=1}^\numitem \Ysim_{k,j} \ratmat_{i,k}
\\
\label{eq:als-cust-simil}
&=& \sum_{k=1}^\numcus \Xsim_{i,k} \ratmat_{k,j}
\end{eqnarray}

The equation (\ref{eq:als-item-simil}) tells us that
the optimal prediction is the weighted sum of the $i$th customer's ratings
where the weights are determined by \Ysim.
This is \emph{exactly the same as the prediction by item-based collaborative filtering}
where item-to-item similarity matrix is given by \Ysim.
On the other hand,
the equation (\ref{eq:als-cust-simil}) tells us that
the optimal prediction is the weighted sum of the $j$th item ratings
where the weights are determined by \Xsim.
This is \emph{exactly the same as the prediction by user-based collaborative filtering}
where user-to-user similarity matrix is given by \Xsim
(which we have not covered in this document).



\subsubsection{Weighted matrix factorization via alternating least-squares (ALS)}

Here we consider the weighted norm for the objective function.
\begin{equation}
\label{eq:vuwj}
f_W(X,Y) = \|W\bullet(\ratmat - XY^T)\|_{F,\ratmat}^2
\end{equation}
where $W\in\preals^{\numcus\times\numitem}$
and $\bullet$ denotes the component-wise multiplication.

This is also a quadratic function of a \href{https://en.wikipedia.org/wiki/Bilinear_map}{bilinear function} of $X$ and $Y$,
thus not a convex function.
Therefore, as before, we can use ALS as follows.

\begin{itemize}
\item choose initial $\initX$ and $\initY$
\item let $\XX{0}:=\initX$ and $\YY{0}:=\initY$
\item let $k:=0$
\item for each iteration, update $X$ and $Y$
\begin{eqnarray}
\label{eq:als-1}
    \XX{k+1} &=& \argmin_X \|\ratmat-X{\YY{k}}^T\|_{F,\ratmat}
\\
\label{eq:als-2}
    \YY{k+1} &=& \argmin_Y \|\ratmat-\XX{k+1}Y^T\|_{F,\ratmat}
\end{eqnarray}
\item stops if certain stopping criterion is satisfied
\item update $k:=k+1$ and repeat iterations
\end{itemize}

The ALS is named so since solving (\ref{eq:als-1}) or (\ref{eq:als-2}) is equivalent to solving a least-squares problem.
Note that the functions in (\ref{eq:als-1}) and (\ref{eq:als-2}) can be separated by $x_i$s or $y_j$s,
\eg, the function in (\ref{eq:als-1}) is
\begin{equation}
\label{eq:bian}
f_Y(X) = \|\ratmat - XY^T\|_{F,\ratmat}^2
= \sum_{i=1}^\numcus \|\tilde{\ratvec}_i^T-x_i^T Y^T\|_{F,\tilde{\ratvec}_i^T}^2
= \sum_{i=1}^\numcus \|\tilde{\ratvec}_i-Yx_i\|_{F,\tilde{\ratvec}_i}^2
\end{equation}
where $\tilde{r}_i \in\reals^\numitem$ are the row vectors of \ratmat,
thus solving (\ref{eq:als-1}) is equivalent to solving \numcus\ uncorrelated problems separately.
This is another great advantage of ALS since we can use parallelism to save the training time considerably.
For example, if we have $N$ processing units, the training time per iteration would be equivalent to
that for solving each (small) least-squares $\numcus/N$ times.
Note that we cannot separate the objective function in (\ref{eq:vuwj}) when we consider $X$ and $Y$ simultaneously
since each of \numrating\ terms in (\ref{eq:vuwj}) are intertwined through $x_i$s and $y_j$s.

When $\ratmat\in\reals^{\numcus\times\numitem}$, it can be easily shown that
\begin{equation}
x_i^\ast =  Y^\dagger \tilde{\ratvec}_i
\end{equation}
where
\begin{equation}
\label{eq:pseudo-inverse-y}
Y^\dagger = (Y^TY)^{-1}Y^T
\end{equation}
is the pseudo-inverse of $Y$.
Thus (\ref{eq:als-1}) becomes
\begin{equation}
\label{eq:als-1-sol}
\XX{k+1} = \ratmat {{\YY{k}}^\dagger}^T
= \ratmat {\YY{k}} \left({\YY{k}}^T{\YY{k}} \right)^{-1}.
\end{equation}
Likewise, (\ref{eq:als-2}) becomes
\begin{equation}
\label{eq:als-2-sol}
\YY{k+1} = \ratmat^T {{\XX{k+1}}^\dagger}^T
= \ratmat^T {\XX{k+1}} \left({\XX{k+1}}^T{\XX{k+1}} \right)^{-1}.
\end{equation}

Note that in practice, we do not evaluate the pseudo-inverse as in (\ref{eq:pseudo-inverse-y})
because it causes catastrophic numerical instability especially when the matrix is huge
(as in most recommendation system cases).
Instead, we use \href{https://en.wikipedia.org/wiki/QR_decomposition}{QR decomposition},
\ie, if $Y = QR$ is the QR decomposition of $Y$, $Y^\dagger = R^{-1} Q^T$.


We now discuss the interpretation of (\ref{eq:als-1-sol}) and (\ref{eq:als-2-sol}).
Let $X^\ast$ and $Y^\ast$ be the optimal solutions for the problem (\ref{eq:mat-fac-form-3}).
Then these should satisfy (\ref{eq:als-1}) and (\ref{eq:als-2}),
\ie,
\begin{eqnarray}
X^\ast &=& \argmin_{X} \|\ratmat - X{Y^\ast}^T \|_{F,\ratmat}
\\
Y^\ast &=& \argmin_{Y} \|\ratmat - X^\ast Y^T\|_{F,\ratmat}
\end{eqnarray}
Then (\ref{eq:als-1-sol}) and (\ref{eq:als-2-sol}) imply that
\begin{eqnarray}
X^\ast {Y^\ast}^T &=& \ratmat Y^\ast ({Y^\ast}^T Y^\ast)^{-1} {Y^\ast}^T = \ratmat \Ysim
\\
&=& X^\ast ({X^\ast}^T X^\ast)^{-1} {X^\ast}^T\ratmat = \Xsim \ratmat
\end{eqnarray}
where
$\Xsim$
and
$\Ysim$
\begin{eqnarray}
\Xsim &=& X^\ast ({X^\ast}^T X^\ast)^{-1} {X^\ast}^T\in \reals^{\numcus \times \numcus}
\\
\Ysim &=& Y^\ast ({Y^\ast}^T Y^\ast)^{-1} {Y^\ast}^T\in \reals^{\numitem \times \numitem}
\end{eqnarray}

These yield very interesting interpretations for \Xsim\ and \Ysim.
These equations imply that the optimal prediction for the rating of the $j$th item by the $i$th customer is
\begin{eqnarray}
\label{eq:als-item-simil}
{x_i^\ast}^T y_j^\ast &=& \sum_{k=1}^\numitem \Ysim_{k,j} \ratmat_{i,k}
\\
\label{eq:als-cust-simil}
&=& \sum_{k=1}^\numcus \Xsim_{i,k} \ratmat_{k,j}
\end{eqnarray}

The equation (\ref{eq:als-item-simil}) tells us that
the optimal prediction is the weighted sum of the $i$th customer's ratings
where the weights are determined by \Ysim.
This is \emph{exactly the same as the prediction by item-based collaborative filtering}
where item-to-item similarity matrix is given by \Ysim.
On the other hand,
the equation (\ref{eq:als-cust-simil}) tells us that
the optimal prediction is the weighted sum of the $j$th item ratings
where the weights are determined by \Xsim.
This is \emph{exactly the same as the prediction by user-based collaborative filtering}
where user-to-user similarity matrix is given by \Xsim
(which we have not covered in this document).





\iffalse

\subsection{Regularization}

XXX: to be filled soon

\subsection{Generalized loss function}

XXX: to be filled soon


\subsection{Training, validation, and test strategy}

XXX: to be filled soon


\fi


\subsection{Collaborative filtering for implicit feedback dataset}

\begin{itemize}

\item binarized variables

\begin{equation}
\binvar_{i,j} = \left\{\begin{array}{ll}
1 & \ratmat_{i,j} \in \reals
\\
0 & \ratmat_{i,j} \not\in \reals
\end{array}\right.
\end{equation}

\item confidence variables

\begin{equation}
\confvar_{i,j} = 1 + \alpha \ratmat_{i,j}
\end{equation}
or
\begin{equation}
\confvar_{i,j} = 1 + \alpha \log(1 + \ratmat_{i,j}/\epsilon)
\end{equation}

\item objective function

\begin{equation}
f(X,Y; \lambda) = \sum_{i=1}^\numcus \sum_{j=1}^\numitem \confvar_{i,j} (\binvar_{i,j} - x_i^Ty_j)^2
+ \lambda_X \|X\|_F^2
+ \lambda_Y \|Y\|_F^2
\end{equation}
where $\lambda = (\lambda_X, \lambda_Y) \in \ppreals^2$
is the coefficient for the regularization.

\item gradients

\begin{eqnarray}
\nonumber
\nabla_{x_i} f(X,Y; \lambda)
&=& -2 \sum_{j=1}^\numitem \confvar_{i,j} (\binvar_{i,j} - x_i^Ty_j) y_j + 2 \lambda_X x_i
\\
\nonumber
&=& 2 \left( \left(
\sum_{j=1}^\numitem \confvar_{i,j} y_j y_j^T + \lambda_X I_\numfactors\right)x_i  - \sum_{j=1}^\numitem \confvar_{i,j} \binvar_{i,j} y_j
\right)
\\
\label{eq:implicit-grad-x}
&=& 2 \left(\left( Y^T \diag(\tilde{\confvar}_i)Y + \lambda_X I_\numfactors \right) x_i - Y^T \diag(\tilde{\confvar}_i) \tilde{\binvar}_i \right)
\end{eqnarray}

Likewise,
\begin{equation}
\label{eq:implicit-grad-x}
\nabla_{y_j} f(X,Y; \lambda)
= 2 \left(\left( X^T \diag(\confvar_j)X + \lambda_Y I_\numfactors \right) y_i - X^T \diag(\confvar_j)  \binvar_j \right)
\end{equation}

Here $\tilde{\confvar}_i \in \reals^\numitem$ and $\confvar_j \in \reals^\numcus$
are the $i$th row vector and the $j$th column vector of $\confmat\in\reals^{\numcus\times\numitem}$
respectively,
and
$\tilde{\binvar}_i \in \reals^\numitem$ and $\binvar_j \in \reals^\numcus$
are the $i$th row vector and the $j$th column vector of $\binmat\in\reals^{\numcus\times\numitem}$
respectively,
\ie,
\begin{eqnarray}
\confmat
&=& \colvecfour
{\tilde{\confvar}_1}
{\tilde{\confvar}_2}
{\vdots}
{\tilde{\confvar}_\numcus}
= \rowvecfour
{{\confvar}_1}
{{\confvar}_2}
{\cdots}
{{\confvar}_\numcus}
\in\reals^{\numcus\times\numitem}
\\
\binmat
&=& \colvecfour
{\tilde{\binvar}_1}
{\tilde{\binvar}_2}
{\vdots}
{\tilde{\binvar}_\numcus}
= \rowvecfour
{{\binvar}_1}
{{\binvar}_2}
{\cdots}
{{\binvar}_\numcus}
\in\reals^{\numcus\times\numitem}
\end{eqnarray}


\end{itemize}


\subsubsection{Regularization coefficient conversion}

\begin{equation}
f(X,Y; \lambda) = \sum_{i=1}^\numcus \sum_{j=1}^\numitem \confvar_{i,j} (\binvar_{i,j} - x_i^Ty_j)^2
+ \lambda_X \|X\|_F^2
+ \lambda_Y \|Y\|_F^2
\end{equation}
where $\lambda = (\lambda_X, \lambda_Y) \in \preals^2$.

Suppose that $(X_\lambda^\ast, Y_\lambda^\ast)$ is the optimal solution for the following problem:
\begin{equation}
\label{eq:als-min-prob-diff-reg}
\begin{array}{ll}
\mbox{minimize} &
f(X,Y; \lambda) = \sum_{i=1}^\numcus \sum_{j=1}^\numitem \confvar_{i,j} (\binvar_{i,j} - x_i^Ty_j)^2 + \lambda_X \|X\|_F^2 + \lambda_Y \|Y\|_F^2
\end{array}
\end{equation}
for some $\lambda$.
Since for any $X$, $Y$, and $a\neq0$,
\begin{equation}
f(aX, (1/a) Y; (1/a^2) \lambda_X, a^2\lambda_Y)
= f(X, Y; \lambda_X, \lambda_Y),
\end{equation}
for any $X$ and $Y$,
\begin{eqnarray*}
\lefteqn{
f(aX_\lambda^\ast, (1/a) Y_\lambda^\ast; (1/a^2) \lambda_X, a^2\lambda_Y)
= f(X_\lambda^\ast, Y_\lambda^\ast; \lambda_X, \lambda_Y)
}
\\
&\leq& f((1/a)X, aY; \lambda_X, \lambda_Y)
= f(X, Y; (1/a^2) \lambda_X, a^2\lambda_Y).
\end{eqnarray*}
Therefore 
$(aX_\lambda^\ast, (1/a) Y_\lambda^\ast)$ is the optimal solution for the following problem:
\begin{equation}
\begin{array}{ll}
\mbox{minimize} & f(X,Y; (1/a^2) \lambda_X, a^2\lambda_Y)
\end{array}
\end{equation}

Therefore if we obtain an optimal solution for a certain $(\tilde{\lambda}_X, \tilde{\lambda}_Y)$, we have readily obtained optimal solutions
for all $(\lambda_X, \lambda_Y)$ pairs such that $\lambda_X \lambda_Y = \tilde{\lambda}_X \tilde{\lambda}_Y$.

Therefore solving the problem (\ref{eq:als-min-prob-diff-reg}) is equivalent to solving the following optimization problem:
\begin{equation}
\begin{array}{ll}
\mbox{minimize} &
f(X,Y; \lambda) = \sum_{i=1}^\numcus \sum_{j=1}^\numitem \confvar_{i,j} (\binvar_{i,j} - x_i^Ty_j)^2 + \sqrt{\lambda_X \lambda_Y} (\|X\|_F^2 + \|Y\|_F^2)
\end{array}
\end{equation}


Now if assume that we have an optimal regularization coefficient $\lambda\in\preals$ for the following ML problem:
\begin{equation}
\label{eq:nefk}
\begin{array}{ll}
\mbox{minimize} &
f(X,Y; \lambda) = \sum_{i=1}^\numcus \sum_{j=1}^\numitem \confvar_{i,j} (\binvar_{i,j} - x_i^Ty_j)^2 + \lambda (\|X\|_F^2 + \|Y\|_F^2)
\end{array}
\end{equation}

If we consider the normalization of each of the three terms in the objective function,
we can formulate the problem as follows.
\begin{equation}
\begin{array}{ll}
\mbox{minimize} &
f(X,Y; \lambda) = \sum_{i=1}^\numcus \sum_{j=1}^\numitem \confvar_{i,j} (\binvar_{i,j} - x_i^Ty_j)^2 / {\numcus\numitem}
+ \lambda (\|X\|_F^2/{\numcus\numfactors} + \|Y\|_F^2/{\numitem \numfactors})
\end{array}
\end{equation}
which is equivalent to
\begin{equation}
\begin{array}{ll}
\mbox{minimize} &
f(X,Y; \lambda) = \sum_{i=1}^\numcus \sum_{j=1}^\numitem \confvar_{i,j} (\binvar_{i,j} - x_i^Ty_j)^2
+ \lambda ((\numitem/\numfactors)\|X\|_F^2 + (\numcus / \numfactors) \|Y\|_F^2)
\end{array}
\end{equation}
which again is equivalent to
\begin{equation}
\label{eq:buysjh}
\begin{array}{ll}
\mbox{minimize} &
f(X,Y; \lambda) = \sum_{i=1}^\numcus \sum_{j=1}^\numitem \confvar_{i,j} (\binvar_{i,j} - x_i^Ty_j)^2
+ (\lambda \sqrt{\numcus \numitem} / \numfactors) ( \|X\|_F^2 + \|Y\|_F^2)
\end{array}
\end{equation}



Now assume that $\lambda^\ast(\numcus, \numitem, \numfactors)$
is the optimal values for $\lambda$ in (\ref{eq:nefk}) (which, for example, can be approximately obtained
by hyperparameter optimization with the formulation (\ref{eq:nefk})).
Then, for some other values $(\numcus', \numitem', \numfactors')$,
(\ref{eq:buysjh}) implies that the approximate optimal $\lambda$ can be found by
\begin{eqnarray*}
&&
\lambda^\ast(\numcus, \numitem, \numfactors) \sqrt{\numcus \numitem} / \numfactors
= \lambda^\ast(\numcus', \numitem', \numfactors') \sqrt{\numcus' \numitem'} / \numfactors'
\\
&\Leftrightarrow&
\lambda^\ast(\numcus', \numitem', \numfactors')
= \lambda^\ast(\numcus, \numitem, \numfactors) \numfactors' \sqrt{\numcus \numitem} / \numfactors\sqrt{\numcus' \numitem'} 
\end{eqnarray*}

So for example, if we have $\numitem = \numitem'$ and $\numfactors = \numfactors'$,
then
\begin{equation}
\lambda^\ast(\numcus', \numitem', \numfactors')
= \sqrt{\frac{\numcus}{\numcus'}} \lambda^\ast(\numcus, \numitem, \numfactors).
\end{equation}







\newpage
\section{Appendix}

\subsection{Linear algebra}

\subsubsection{Singular value decomposition (SVD)}
\label{subsubsec:svd}

For any rank-$k$ matrix $A\in\reals^{m\times n}$,
there exist three matrix
$U\in\reals^{m\times k}$,
$\Sigma\in\reals^{k\times k}$,
and
$V\in\reals^{n\times k}$,
such that
\begin{equation}
A = U \Sigma V^T
\end{equation}
where
the columns of $U$ are orthonormal,
the columns of $V$ are orthonormal,
and
$\Sigma$ is a diagonal matrix with nonincreasing positive diagonal entries,
\ie,
\begin{equation}
U^T U = V^T V = I_k
\in\reals^{k\times k}
\end{equation}
and
\begin{equation}
\Sigma = \diag(\sigma_1, \ldots, \sigma_k) =
\bigmat
{\sigma_1}
{0}
{0}
{0}
{\sigma_2}
{0}
{0}
{0}
{\sigma_k}
\in\reals^{k\times k}
\end{equation}
with
\begin{equation}
\sigma_1 \geq \sigma_2 \geq \cdots \geq \sigma_k > 0
\end{equation}
where $I_k$ referst to $k$-by-$k$ identity matrix.

If we let $u_1, \ldots, u_k \in\reals^n$ be the $k$ column vectors of $U$,
\ie,
\begin{equation}
U = \rowvecthree{u_1}{\cdots}{u_k} \in \reals^{n\times k},
\end{equation}
$U^T U =I_k$ holds if and only if
\begin{equation}
u_i^T u_j = \delta_{i,j} = \left\{\begin{array}{ll}
1 & \mbox{if } i=j
\\
0 & \mbox{otherwise}
\end{array}\right.
\end{equation}
where $\delta_{i,j}$ denotes the \href{https://en.wikipedia.org/wiki/Kronecker_delta}{Kronecker delta},
which means the length of each $u_i$ is $1$
and all of them are orthogonal to each other.
Likewise,
if we let $v_1, \ldots, v_k \in\reals^m$ be the $k$ column vectors of $V$,
\ie,
\begin{equation}
V = \rowvecthree{v_1}{\cdots}{v_k} \in \reals^{m\times k},
\end{equation}
$V^T V =I_k$ holds if and only if
\begin{equation}
v_i^T v_j = \delta_{i,j}
\end{equation}
which means the length of each $v_i$ is $1$
and all of them are orthogonal to each other.
Note that
\begin{itemize}
\item $\sigma_1, \ldots, \sigma_k$ are called \emph{singular values} of $A$.
\item $u_1, \ldots, u_k$ are called \emph{left singular vectors} of $A$.
\item $v_1, \ldots, v_k$ are called \emph{right singular vectors} of $A$.
\end{itemize}

Note also that $A$ can be expressed as
\begin{equation}
A = \sum_{i=1}^k \sigma_i u_i v_i^T,
\end{equation}
\ie, $A$ can be express as a linear combination of $k$ rank-$1$ matrices.



\subsubsection{Singular value decomposition as rank-$k$ approximation}
\label{subsubsec:svd-approx}

Given a matrix $A\in\reals^{m\times n}$
where $\rank(A)=k$,
consider the following optimization problem with $r\leq k$.
\begin{equation}
\label{eq:prob:norm-min-prob}
\begin{array}{ll}
\mbox{minimize} & \|A-B\|_F
\\
\mbox{subject to} & \rank(B) = r
\end{array}
\end{equation}
where the optimization variable is $B\in\reals^{m\times n}$
and $\|\cdot\|_F$ denotes the \href{http://mathworld.wolfram.com/FrobeniusNorm.html}{Frobenius norm}.

It turns out that
\begin{equation}
B^\ast = U_r \Sigma_r V_r^T
= \sum_{i=1}^r \sigma_i u_i v_i^T,
\end{equation}
is an optimal solution for the problem (\ref{eq:prob:norm-min-prob}),
\ie,
\begin{equation}
\|A- U_r \Sigma_r V_r^T\|_F \leq \|A-B\|_F
\end{equation}
for every $B$ with $\rank(B) = r$
where
$U_r \in \reals^{m\times r}$,
$V_r \in \reals^{n\times r}$,
and
$\Sigma_r \in \reals^{r\times r}$
are defined as
\begin{eqnarray}
U_r &=& \rowvecthree{u_1}{\cdots}{u_r} \in \reals^{n\times r}
\\
V_r &=& \rowvecthree{v_1}{\cdots}{v_r} \in \reals^{m\times r}
\\
\Sigma_r &=& \diag(\sigma_1, \ldots, \sigma_r) \in \reals^{r\times r}
\end{eqnarray}


This means the close rank-$r$ matrix to $A$ (in Frobenius norm sense) can be obtained
with $k$ largest singular values together with corresponding $k$ left and right singular vectors.
