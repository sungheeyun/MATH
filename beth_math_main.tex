\documentclass[12pt]{article}   	% use "amsart" instead of "article" for AMSLaTeX format

%\newcommand{\theoremsandsuchchapter}{}

\input{/Users/sunyun/mytex/mydefs}

\usepackage{appendix}
\usepackage{geometry}                		% See geometry.pdf to learn the layout options. There are lots.
\geometry{letterpaper}                   		% ... or a4paper or a5paper or ...
%\geometry{landscape}                		% Activate for rotated page geometry
\usepackage[parfill]{parskip}    		% Activate to begin paragraphs with an empty line rather than an indent
\usepackage{graphicx}				% Use pdf, png, jpg, or eps§ with pdflatex; use eps in DVI mode
								% TeX will automatically convert eps --> pdf in pdflatex
\usepackage{amssymb}

\usepackage{hyperref}
\hypersetup{
    colorlinks=true,
    linkcolor=blue,
    filecolor=magenta,
    %urlcolor=cyan,
    urlcolor=blue,
}

\usepackage{amsmath}

%SetFonts

%SetFonts

\newcommand{\feasibleset}{\mathcal{F}}
\newcommand{\optsolset}{\mathcal{X}^\ast}
\newcommand{\grad}{\nabla}
\newcommand{\possemidefset}[1]{\mathcal{S}_+^{#1}}
\newcommand{\posdefset}[1]{\mathcal{S}_{++}^{#1}}
\newcommand{\covmat}[1]{{\Sigma}_{#1}}


\title{Math for Beth}
%\author{}
%\date{}							% Activate to display a given date or no date


\begin{document}

\maketitle

%\setcounter{secnumdepth}{4}
%\setcounter{tocdepth}{3}
%\tableofcontents


\section{Algebra}

\subsection{Commutative law, associative law, distributive law}

\begin{itemize}
\item Commutative law
\begin{eqnarray*}
a + b &=& b + a
\\
a \times b &=& b \times a
\end{eqnarray*}

Examples:
\begin{itemize}
\item $2 + 4 = 4 + 2$
\item $3 \times 4 = 4 \times 3$
\end{itemize}

However,
\begin{eqnarray*}
a - b &\neq& b - a
\\
a \div b &\neq& b \div a
\end{eqnarray*}
\ie, commutative law does not hold for operations, such as, subtraction and division.

Counter examples:
\begin{itemize}
\item $ 2 - 4 = -2 \neq 2 = 4 - 2 $
\item $2 \div 4 = 1/2 \neq 2 = 4 \div 2$
\end{itemize}


\item Associative law
\begin{eqnarray*}
(a + b) + c  &=&  a + (b + c)
\\
(a \times b) \times c  &=&  a \times (b \times c)
\end{eqnarray*}

Examples:
\begin{itemize}
    \item $(2 + 3) + 1  =  2 + (3 + 1)$
    \item $(2 \times 4) \times 3  =  2 \times (4 \times 3)$
\end{itemize}

However, 
\begin{eqnarray*}
(a - b) - c  &\neq&  a - (b - c)
\\
(a \div b) \div c  &\neq&  a \div (b \div c)
\end{eqnarray*}

\item Distributive law
\begin{eqnarray*}
a \times (b + c)  &=& a \times b + a \times c
\\
(a + b) \times c  &=&  a \times c + b \times c
\end{eqnarray*}

Examples:
\begin{itemize}
    \item $3 \times (4 + 2)  = 3 \times 4 + 3 \times 2$
    \item $(4 + 3) \times 5  =  4 \times 5 + 3 \times 5$
\end{itemize}

\end{itemize}


\section{Others}

\subsection{Pigeonhole principle}

If $n$ pigeons are put to into $m$ pigeonholes with $n>m$,
then at least one pigeonhole must contain more than one pigeon.

For example, assume that there are $5$ pigeons and $4$ pigeonholes.
If $5$ pigeons go into one of the $4$ pigeonholes,
then there exists at least one pigeonhole with more than one pigeon.


\end{document}
